\documentclass[notheorems,compress,mathserif,table]{beamer}

\useoutertheme{tree}
\usecolortheme{whale}      % Outer color themes, 其他选择: whale, seahorse, dolphin . 换一个编译看看有什么不同.
\usecolortheme{orchid}     % Inner color themes, 其他选择: lily, orchid
\useinnertheme[shadow]{rounded} % 对 box 的设置: 圆角、有阴影.
\setbeamercolor{sidebar}{bg=blue!50} % sidebar的颜色, 50%的蓝色.
%\setbeamercolor{background canvas}{bg=blue!9} % 背景色, 9%的蓝色. 去掉下一行, 试一试这个.
\setbeamertemplate{background canvas}[vertical shading][bottom=white,top=structure.fg!25] %%背景色, 上25%的蓝, 过渡到下白.
\usefonttheme{serif}  % 字体. 个人偏好有轮廓的字体. 去掉这个设置编译, 就看到不同了.
\setbeamertemplate{navigation symbols}{}   %% 去掉页面下方默认的导航条.
%%------------------------常用宏包---------------------------------------------------------------------
%%注意, beamer 会默认使用下列宏包: amsthm, graphicx, hyperref, color, xcolor, 等等
%\usepackage{CJK}
\usepackage{ctex}
\usepackage{amsmath,amsthm,amsfonts,amssymb,bm}
\usepackage{mathrsfs}
\usepackage{subfigure} %%图形或表格并排排列
\usepackage{xmpmulti}  %%支持文中的 \multiinclude 等命令, 使 mp 文件逐帧出现. 具体讨论见 beamer 手册.
\usepackage{colortbl,dcolumn}     %% 彩色表格
%\logo{\includegraphics[height=0.09\textwidth]{ajln.jpg}}   %左上角科大logo
%%%%%%%%%%%%%%%%%%%%%%%%%%%%%%%%%%%%%%重定义字体、字号命令 %%%%%%%%%%%%%%%%%%%%%%%%%%%%%%%%%%%%%%%%%%%%%%
%\newcommand{\songti}{\CJKfamily{song}}        % 宋体
%\newcommand{\fangsong}{\CJKfamily{fs}}        % 仿宋体
%\newcommand{\kaishu}{\CJKfamily{kai}}         % 楷体
%\newcommand{\heiti}{\CJKfamily{hei}}          % 黑体
%\newcommand{\lishu}{\CJKfamily{li}}           % 隶书
\newcommand{\youyuang}{\CJKfamily{you}}       % 幼圆
\newcommand{\sanhao}{\fontsize{16pt}{\baselineskip}\selectfont}     % 字号设置
\newcommand{\sihao}{\fontsize{14pt}{\baselineskip}\selectfont}      % 字号设置
\newcommand{\xiaosihao}{\fontsize{12pt}{\baselineskip}\selectfont}  % 字号设置
\newcommand{\wuhao}{\fontsize{10.5pt}{\baselineskip}\selectfont}    % 字号设置
\newcommand{\xiaowuhao}{\fontsize{9pt}{\baselineskip}\selectfont}   % 字号设置
\newcommand{\liuhao}{\fontsize{7.875pt}{\baselineskip}\selectfont}  % 字号设置
\newcommand{\qihao}{\fontsize{5.25pt}{\baselineskip}\selectfont}    % 字号设置
%%%%%%%%%%%%%%%%%%%%%%%%%%%%%%%%%%%%%%%%%%%%%%%%%%%%%%%%%%%%%%%%%%%%%%%%%%%%%%%%%%%%%%%%%%%%%%%%%%%%%%%%
%%----------------------- Theorems ---------------------------------------------------------------------
\newtheorem{theorem}{定理}
\newtheorem{definition}{定义}
\newtheorem{lemma}{引理}
\newtheorem{example}{例题}
\newtheorem{answer}{解:}
\newtheorem{dablock}{}
\newtheorem{jytg}{提纲}
\newtheorem{daproof}{证明}
\newtheorem{explain}{说明}
\newtheorem{summary}{小结}

\newtheorem{zhuyi}{注意}
\newtheorem{shuoming}{说明}
\newtheorem{wenti}{问题}
\newtheorem{jielun}{结论}
\newtheorem{yinli}{引理}
%%----------------------------------------------------------------------------------------------------
\title{\heiti 第2章\quad 时域离散信号和系统的频域分析}
\author[\textcolor{blue}]{{\sihao\kaishu  笪邦友}}
\institute{\sihao\lishu  \textcolor{violet}{中南民族大学~~ 电子信息工程学院}}
\date{\fangsong\today} 

\begin{document}
%  \begin{CJK*}{GBK}{kai}
\kaishu
\frame{ \titlepage }
%%---------------------------------------------------------------------------------------------------
\section*{目录}
\frame{\kaishu\frametitle{\kaishu 目录}\tableofcontents}
\section*{前言}
%%%%%%%%%%%%%%%%%%%%%%%%%%%%%%%%%%%%%%%%%%%%%%%%%%%%%%%%%%%%%%%%%%%%%%%%%%%%%%%%%%%%%%%%%%%%%%
\begin{frame}[shrink]\frametitle{\kaishu 傅里叶变换的回顾}%[allowframebreaks][shrink]
回顾一下模拟信号的傅里叶变换

\begin{enumerate}
	\item [(1)] 周期连续信号的傅里叶级数
	\item [(2)] 非周期连续信号的傅里叶变换
\end{enumerate}
%例如,我们曾学习过模拟信号的傅里叶变换公式:
%\begin{dablock}
%\begin{align*}
%  F(j\Omega) &= \int_{-\infty}^{\infty}f(t)e^{-j\Omega t}dt \\
%  f(t)\quad  &= \frac{1}{2\pi}\int_{-\infty}^{\infty}F(j\Omega)e^{j\Omega t}d\Omega \qquad\qquad\qquad\qquad
%\end{align*}
%\end{dablock}
\pause
\begin{wenti}
	对于离散信号,是否有傅里叶变换和傅里叶级数的相关概念?
	\pause
	\begin{enumerate}
		\item 非周期离散信号的傅里叶变换
		\item 周期离散信号的傅里叶级数和傅里叶变换
	\end{enumerate}
\end{wenti}
\end{frame}
%%%%%%%%%%%%%%%%%%%%%%%%%%%%%%%%%%%%%%%%%%%%%%%%%%%%%%%%%%%%%%%%%%%%%%%%%%%%%%%%%%%%%%%%%%%%%%%
\section{2.1 序列的傅里叶变换的定义}

%\subsection*{序列的傅里叶变换的定义}

%%%%%%%%%%%%%%%%%%%%%%%%%%%%%%%%%%%%%%%%%%%%%%%%%%%%%%%%%%%%%%%%%%%%%%%%%%%%%%%%%%%%%%%%%%%%%%
\begin{frame}[shrink]\frametitle{时域离散信号的傅里叶变换的定义}%[allowframebreaks][shrink]


\begin{definition}
设序列$x(n)$满足绝对可和条件,即$\sum_{n=-\infty}^{\infty}|x(n)|<\infty$,则定义
\begin{equation*}
	X(e^{j\omega}) = \sum_{n=-\infty}^{\infty}x(n)e^{-j\omega n}
\end{equation*}
为序列的傅里叶变换,可用$FT$表示,即$X(e^{j\omega}) = FT\big[x(n)\big]$。
\end{definition}
\pause
\par 显然有:
\begin{enumerate}
\item [(1)] $X(e^{j\omega})$ 是$\omega$ 的连续函数,被称为信号$x(n)$的频谱密度函数,其含义与连续信号的频谱密度函数相同。
\item [(2)] $X(e^{j\omega})$ 是$\omega$ 的周期函数,周期为$T=2\pi$。
\end{enumerate}
%\newpage

\end{frame}




\begin{frame}[shrink]\frametitle{问题:FT 的反变换}%[allowframebreaks][shrink]

%  \begin{enumerate}
已知序列$x(n)$的傅里叶变换为:
\begin{equation*}
X(e^{j\omega}) = \sum_{n=-\infty}^{\infty}x(n)e^{-j\omega n}
\end{equation*}
% \item
求$ X(e^{j\omega})$的反变换$x(n)$的表达式。%,也叫傅里叶反变换。
\begin{equation*}
x(n) = \frac{1}{2\pi}\int_{-\pi}^{\pi}X(e^{j\omega})e^{j\omega n}d\omega
\end{equation*}

\end{frame}

\begin{frame}[allowframebreaks]\frametitle{证明:}%[allowframebreaks][shrink]
%证明:\par

\begin{enumerate}
\item [1]
因为$X(e^{j\omega})$为连续函数,且$T=2\pi$,用$e^{j\omega m}$乘以定义式两边,
并在$(-\pi,\pi)$内对$\omega$ 进行积分,这里$m,n\in Z$,
可得:
\begin{equation*}
\begin{split}
\int_{-\pi}^{\pi}X(e^{j\omega})e^{j\omega m}d\omega &= \int_{-\pi}^{\pi}\left[\sum_{n=-\infty}^{\infty}x(n)e^{-j\omega n}\right]e^{j\omega m}d\omega \\
&= \sum_{n=-\infty}^{\infty}x(n)\int_{-\pi}^{\pi}e^{j\omega(m-n)}d\omega
\end{split}
\end{equation*}
\item [2]
显然,关键是下式
$$\int_{-\pi}^{\pi}e^{j\omega(m-n)}d\omega$$
的求解。
\newpage
显然有
\begin{equation*}
\int_{-\pi}^{\pi}e^{j\omega(m-n)}d\omega = 2\pi\delta(n-m) =
\left\{\begin{array}{r@{,\quad}l}
2\pi  &n=m\\
0     &\mbox{其他}
\end{array} \right.
\end{equation*}

简单说明:
\begin{enumerate}
\item [(1)] 当$m-n=0$ 时,显然有$$\int_{-\pi}^{\pi}e^{j\omega\cdot 0}d\omega = 2\pi$$
\item [(2)] 当$m-n\neq 0$时,不妨令$m-n=k$,$k$为整数,有
\begin{equation*}
\begin{split}
	\int_{-\pi}^{\pi}e^{j\omega(m-n)}d\omega
	&= \int_{-\pi}^{\pi}e^{j\omega k}d\omega \\
	&= \int_{-\pi}^{\pi}cos(\omega k)d\omega + j\int_{-\pi}^{\pi}sin(\omega k)d\omega =0
\end{split}
\end{equation*}
\end{enumerate}
\item [3]那么有:
\begin{equation*}
\begin{split}
\int_{-\pi}^{\pi}X(e^{j\omega})e^{j\omega m}d\omega
&= \int_{-\pi}^{\pi}\left[\sum_{n=-\infty}^{\infty}x(n)e^{-j\omega n}\right]e^{j\omega m}d\omega \\
&= \sum_{n=-\infty}^{\infty}x(n)\int_{-\pi}^{\pi}e^{j\omega(m-n)}d\omega \\
&= 2\pi\sum_{n=-\infty}^{\infty}x(n)\delta(n-m) \\
&= 2\pi\cdot x(m)
\end{split}
\end{equation*}

%                $$\int_{-\pi}^{\pi}X(e^{j\omega})e^{j\omega m}d\omega
%                  = 2\pi\sum_{n=-\infty}^{\infty}x(n)\delta(n-m) = 2\pi\cdot x(m)
%                $$
因此:
\begin{equation*}
x(m) = \frac{1}{2\pi}\int_{-\pi}^{\pi}X(e^{j\omega})e^{j\omega m}d\omega
\end{equation*}
\newpage
前述有:
\begin{equation*}
x(m) = \frac{1}{2\pi}\int_{-\pi}^{\pi}X(e^{j\omega})e^{j\omega m}d\omega
\end{equation*}
\newline
\item [4] 将$m$换为$n$,则可得到:
\begin{equation*}
x(n) = \frac{1}{2\pi}\int_{-\pi}^{\pi}X(e^{j\omega})e^{j\omega n}d\omega
\end{equation*}
上式即为傅里叶变换的逆变换,
\end{enumerate}
\end{frame}

\begin{frame}[shrink]\frametitle{FT成立的充分条件}%[allowframebreaks][shrink]
\begin{zhuyi}
FT成立的充分条件为:
\begin{equation*}
\sum_{n=-\infty}^{\infty}|x(n)| <\infty
\end{equation*}
有些序列并不满足上述绝对可和条件,如周期函数,但如果引入冲激函数$\delta(\cdot)$,则其傅里叶变换也可得到。
\end{zhuyi}



\end{frame}
%%%%%%%%%%%%%%%%%%%%%%%%%%%%%%%%%%%%%%%%%%%%%%%%%%%%%%%%%%%%%%%%%%%%%%%%%%%%%%%%%%%%%%%%%%%%%%%



%%%%%%%%%%%%%%%%%%%%%%%%%%%%%%%%%%%%%%%%%%%%%%%%%%%%%%%%%%%%%%%%%%%%%%%%%%%%%%%%%%%%%%%%%%%%%%
\begin{frame}[allowframebreaks]\frametitle{}%[allowframebreaks][shrink]
\begin{example}
设$x(n)= R_{N}(n)$,求$x(n)$的傅里叶变换。
\end{example}

\par\textbf{解}:
\begin{equation*}
\begin{split}
X(e^{j\omega})
&= \sum_{n=-\infty}^{\infty}R_{N}(n)e^{-j\omega n} \\
&= \sum_{n=0}^{N-1}e^{-j\omega n}
= \frac{1-e^{-j\omega N}}{1-e^{-j\omega}}\\
&= \frac{e^{-j\omega N/2}(e^{j\omega N/2}-e^{-j\omega N/2})}{e^{-j\omega/2}(e^{j\omega/2}-e^{-j\omega/2})}\\
&= e^{-j\frac{N-1}{2}\omega}\quad\cdot \quad \frac{sin(\omega N/2)}{sin(\omega/2)}
\end{split}
\end{equation*}



%$$X(e^{j\omega}) = \sum_{n=-\infty}^{\infty}R_{N}(n)e^{-j\omega n} = \sum_{n=0}^{N-1}e^{-j\omega n}= \frac{1-e^{-j\omega N}}{1-e^{-j\omega}}$$
%\quad\quad\quad$$=\frac{e^{-j\omega N/2}(e^{j\omega N/2}-e^{-j\omega N/2})}{e^{-j\omega/2}(e^{j\omega/2}-e^{-j\omega/2})}%$$\quad\quad\quad$$
%=e^{-j(N-1)\omega/2}\frac{sin(\omega N/2)}{sin(\omega/2)}$$
\newpage
$$\mbox{一般有}\quad\quad\quad  X(e^{j\omega}) = |X(e^{j\omega})|\cdot e^{j\varphi(\omega)}$$
$$ |X(e^{j\omega})| = \left|\frac{sin(\omega N/2)}{sin(\omega/2)}\right| \quad\quad\quad\quad$$
$$      \varphi(\omega) = -\frac{N-1}{2}\omega + \varphi_2(\omega) \quad\quad\quad\quad$$
\begin{equation*}
\mbox{其中:}\quad\quad\quad \varphi_2(\omega)
= \left\{\begin{array}
{r@{,\quad}l}
0    &  \frac{sin(\omega N/2)}{sin(\omega/2)} >0\\
\pi  &  \frac{sin(\omega N/2)}{sin(\omega/2)} <0
\end{array} \right.
\end{equation*}

%\begin{figure}[h]
%  % Requires \usepackage{graphicx}
%  \includegraphics[width=0.2\textwidth]{blankpic.jpg}\\
%  \caption{$R_{4}(n)$的幅度与相位曲线}
%  %\label{}
%\end{figure}

\end{frame}


\section{2.2 周期序列的离散傅里叶级数及傅里叶变换}
\subsection*{周期序列展开为离散傅里叶级数}

%%%%%%%%%%%%%%%%%%%%%%%%%%%%%%%%%%%%%%%%%%%%%%%%%%%%%%%%%%%%%%%%%%%%%%%%%%%%%%%%%%%%%%%%%%%%%%
\begin{frame}[shrink]\frametitle{周期序列展开为离散傅里叶级数}%[allowframebreaks][shrink]


\begin{enumerate}
\item [1] DFS正变换
\item [2] DFS反变换
\item [3] 几点说明
\end{enumerate}
%周期序列不满足绝对可和的条件,因此它的傅里叶变换是不存在的。但周期序列可以展开为傅里叶级
%数,引入奇异函数$\delta$,可以得到其傅里叶变换。


\end{frame}
%%%%%%%%%%%%%%%%%%%%%%%%%%%%%%%%%%%%%%%%%%%%%%%%%%%%%%%%%%%%%%%%%%%%%%%%%%%%%%%%%%%%%%%%%%%%%%%



%%%%%%%%%%%%%%%%%%%%%%%%%%%%%%%%%%%%%%%%%%%%%%%%%%%%%%%%%%%%%%%%%%%%%%%%%%%%%%%%%%%%%%%%%%%%%%
\begin{frame}[shrink]\frametitle{1、 DFS 正变换}%[allowframebreaks][shrink]
%\begin{enumerate}
%\item [1] DFS正变换
%\end{enumerate}
\begin{theorem}
设$\tilde{x}(n)$是以$N$为周期的离散序列,其必能展开为离散傅里叶级数,记做:
\begin{equation*}%\label{}
\tilde{x}(n) = \sum_{k=0}^{N-1}a_{k}\cdot e^{j\frac{2\pi}{N}kn}
\end{equation*}
式中$a_{k}$是傅里叶级数的系数。
\end{theorem}


%下面我们求系数$a_{k}$的表达式。

\par 问题在于得到系数$a_{k}$的表达式。

\end{frame}




\begin{frame}[allowframebreaks]\frametitle{系数$a_{k}$ 的求解}%[allowframebreaks][shrink]
\begin{enumerate}
\item [(1)]
等式两边同时乘以$e^{-j\frac{2\pi}{N}mn}$,并对$n$ 在一个周期内求和,
且$0\leqslant m\leqslant N-1$,这里$m,k\in Z$,则有:
\begin{equation*}
\begin{split}
\sum_{n=0}^{N-1}\tilde{x}(n)e^{-j\frac{2\pi}{N}mn}
&= \sum_{n=0}^{N-1}\left[\sum_{k=0}^{N-1}a_{k}e^{j\frac{2\pi}{N}kn}\right]e^{-j\frac{2\pi}{N}mn}\\
&=  \sum_{k=0}^{N-1}a_{k}\left[\sum_{n=0}^{N-1}e^{j\frac{2\pi}{N}(k-m)n}\right]
\end{split}
\end{equation*}
\end{enumerate}
%\end{frame}
%
%
%
%
%\begin{frame}[shrink]\frametitle{求解}%[allowframebreaks][shrink]


\newpage
\begin{enumerate}
\item [(2)] 式中
\begin{equation*}
\sum_{n=0}^{N-1}e^{j\frac{2\pi}{N}(k-m)n} = N\cdot \delta(k-m)
= \left\{\begin{array}
{r@{,\quad}l}
N&\mbox{$k=m$}\\0&\mbox{$k\neq m$}
\end{array} \right.
\end{equation*}

简单说明:
\begin{enumerate}\par
\item  [(a)]当$m-k=0$时,显然成立,代入即可。
\item  [(b)]当$m-k\neq 0$时,不妨令$k-m=p$,$p$为整数,有
\begin{equation*}
\begin{split}
\sum_{n=0}^{N-1}e^{j\frac{2\pi}{N}(k-m)n}
&= \sum_{n=0}^{N-1}e^{j\frac{2\pi}{N}pn} \\
&= \frac{1-\left[e^{j\frac{2\pi}{N}p}\right]^{N}}{1-e^{j\frac{2\pi}{N}p}}
= \frac{1-e^{j2\pi p}}{1-e^{j\frac{2\pi}{N}pn}}  =0
\end{split}
\end{equation*}
\end{enumerate}
\end{enumerate}
%\end{frame}
%
%
%
%\begin{frame}[shrink]\frametitle{求解}%[allowframebreaks][shrink]

\newpage
\begin{enumerate}
\item [(3)] 所以有

$$\sum_{n=0}^{N-1}\tilde{x}(n)e^{-j\frac{2\pi}{N}mn}=\sum_{k=0}^{N-1}a_{k}\cdot N\cdot \delta(k-m)
= a_{m}\cdot N $$
所以:
$$a_{m} = \frac{1}{N}\sum_{n=0}^{N-1}\tilde{x}(n)e^{-j\frac{2\pi}{N}mn}$$
最后,将$m$替换为$k$,即有:
$$a_{k} = \frac{1}{N}\sum_{n=0}^{N-1}\tilde{x}(n)e^{-j\frac{2\pi}{N}kn}\quad\quad 0\leqslant k\leqslant N-1$$
\end{enumerate}
\end{frame}


\begin{frame}[shrink]\frametitle{DFS正变换}%[allowframebreaks][shrink]
%\newpage
%\begin{enumerate}
%      \item  [(4)]
令$\tilde{X}(k)=N\cdot a_{k}$,则有

$$
\tilde{X}(k) = \sum_{n=0}^{N-1}\tilde{x}(n)e^{-j\frac{2\pi}{N}kn}\quad-\infty<k<\infty
$$
记做:
\begin{equation*}
\tilde{X}(k) = DFS[\tilde{x}(n)]
= \sum_{n=0}^{N-1}\tilde{x}(n)e^{-j\frac{2\pi}{N}kn}%\quad\quad0<k<N-1
\end{equation*}

$\tilde{X}(k)$也是以$N$为周期的周期序列,称为周期序列$\tilde{x}(n)$的离散傅里叶级数系数,用$DFS$ 表示。记为$DFS[\tilde{x}(n)]=\tilde{X}(k)$
%\end{enumerate}
\end{frame}
%故有
%        $$a_{k} = \frac{1}{N}\sum_{n=0}^{N-1}\tilde{x}(n)e^{-j\frac{2\pi}{N}kn} = \frac{1}{N}\sum_{n=0}^{N-1}\tilde{x}(n)e^{-j\frac{2\pi}{N}(k+lN)n}=a_{k+N}$$
%        显然$a_{k}$也是一个周期序列,满足$a_{k}=a_{k+lN}$\par
%        令$\tilde{X}(k)=Na_{k}$,则有,
%        \begin{equation}
%            \tilde{X}(k) = \sum_{n=0}^{N-1}\tilde{x}(n)e^{-j\frac{2\pi}{N}kn}\quad\quad0<k<N-1
%        \end{equation}
%        $\tilde{X}(k)$也是一个以N为周期的周期序列,称为周期序列$\tilde{x}(n)$的离散傅里叶级数。
%


\begin{frame}[shrink]\frametitle{2、$DFS$ 反变换}%[allowframebreaks][shrink]

用$a_k = \frac{1}{N}\tilde{X}(k)$代入上式,可得:

\begin{equation*}%\label{}
\begin{split}
\tilde{x}(n)
&= \sum_{k=0}^{N-1}a_{k}e^{j\frac{2\pi}{N}kn} \qquad \qquad\qquad\qquad\qquad\qquad\qquad\\
&= \frac{1}{N}\sum_{k=0}^{N-1}\tilde{X}(k)e^{j\frac{2\pi}{N}kn}  \\
&= IDFS[\tilde{X}(k)]
\end{split}
\end{equation*}


\end{frame}




\begin{frame}[shrink]\frametitle{结论:$DFS$变换的表达式}%[allowframebreaks][shrink]

上述两个公式可重写为:
\begin{equation*}
\left\{ \begin{aligned}
\tilde{X}(k) &= DFS[\tilde{x}(n)]\quad  = \sum_{n=0}^{N-1}\tilde{x}(n)e^{-j\frac{2\pi}{N}kn}  \qquad(-\infty<k<\infty)\\
\tilde{x}(n) &= IDFS[\tilde{X}(k)] =  \frac{1}{N}\sum_{k=0}^{N-1}\tilde{X}(k)e^{j\frac{2\pi}{N}kn}\quad(-\infty<n<\infty)
\end{aligned} \right.
\end{equation*}


\end{frame}







\begin{frame}[shrink]\frametitle{3、几点说明}%[allowframebreaks][shrink]

\begin{enumerate}
\item [(1)] 若将$n$视为时间变量,$k$视为频率变量,则$DFS$表示时域$\rightarrow$频域的变换,$IDFS$ 表示从
频域$\rightarrow$时域的变换。
\item [(2)]
$\tilde{x}(n)$与$\tilde{X}(k)$ 均是周期序列,且周期均为$N$\par
\begin{enumerate}
\item  [(a)] 周期序列虽然是无限长的,但只要知道一个周期数据,则整个序列的值可知。
\item  [(b)] 周期序列与有限长序列有着本质联系。实际处理的序列均为有限长序列,设长度为$N$,如以$N$ 为周期进行周期延拓,则可得到一个周期序列$\tilde{x}(n)$。
\end{enumerate}
\item [(3)]  物理意义:
$$\tilde{x}(n) = \frac{1}{N}\sum_{n=0}^{N-1}\tilde{X}(k)e^{j\frac{2\pi}{N}kn}$$
表明周期序列可分解为$N$次谐波,第$k$个谐波频率为$\omega_{k}=\frac{2\pi}{N}k$,幅度为$\frac{1}{N}\tilde{X}(k)$,
$k=0,1,2,\cdots,N-1$。
\end{enumerate}
\end{frame}
%%%%%%%%%%%%%%%%%%%%%%%%%%%%%%%%%%%%%%%%%%%%%%%%%%%%%%%%%%%%%%%%%%%%%%%%%%%%%%%%%%%%%%%%%%%%%%%


\subsection*{周期序列的傅里叶变换}
%%%%%%%%%%%%%%%%%%%%%%%%%%%%%%%%%%%%%%%%%%%%%%%%%%%%%%%%%%%%%%%%%%%%%%%%%%%%%%%%%%%%%%%%%%%%%%
\begin{frame}\frametitle{前言:周期序列的傅里叶变换}%[allowframebreaks][shrink]


严格说,周期序列是不存在傅里叶变换的,因为周期序列不满足绝对可和条件,但引入冲激函数后,
傅里叶变换条件可以放松,周期序列的傅里叶变换存在。
\end{frame}





\begin{frame}\frametitle{关键在于指数函数$e^{j\frac{2\pi}{N}kn}$的傅里叶变换}%[allowframebreaks][shrink]
我们再来看一下$\tilde{x}(n)$的DFS公式。
$$\tilde{x}(n) = \frac{1}{N}\sum_{n=0}^{N-1}\tilde{X}(k)e^{j\frac{2\pi}{N}kn}$$

$$FT\left[\tilde{x}(n)\right] = \frac{1}{N}\sum_{n=0}^{N-1}\tilde{X}(k)FT\left[e^{j\frac{2\pi}{N}kn}\right]$$
显然,想要得到$\tilde{x}(n)$的傅里叶变换,关键在于得到指数函数$e^{j\frac{2\pi}{N}kn}$ 的傅里叶变换。
\end{frame}
%%%%%%%%%%%%%%%%%%%%%%%%%%%%%%%%%%%%%%%%%%%%%%%%%%%%%%%%%%%%%%%%%%%%%%%%%%%%%%%%%%%%%%%%%%%%%%%


\subsubsection*{复指数序列$e^{j\omega_{0}n}$的傅里叶变换*}


%%%%%%%%%%%%%%%%%%%%%%%%%%%%%%%%%%%%%%%%%%%%%%%%%%%%%%%%%%%%%%%%%%%%%%%%%%%%%%%%%%%%%%%%%%%%%%
\begin{frame}[shrink]\frametitle{一、复指数序列$e^{j\omega_{0}n}$的傅里叶变换}%[allowframebreaks][shrink]
教材讲授思路: \newline


1  类比
\quad\newline\quad

\qquad 在模拟系统中,$x_{a}(t)= e^{j\Omega_{0} t}$的傅里叶变换是在$\Omega=\Omega_{0}$处的单位冲激函数,强度是$2\pi$,即
\begin{equation*}
X_{a}(j\Omega) = FT[x_{a}(t)] = \int_{-\infty}^{\infty}e^{j\Omega_{0}t}
e^{-j\Omega t}dt = 2\pi\delta(\Omega-\Omega_{0})
\end{equation*}
$$
\mbox{即: \quad\quad} e^{j\Omega_0 t} \longleftrightarrow 2\pi\delta(\Omega- \Omega_0)
$$
\end{frame}


\begin{frame}[shrink]\frametitle{}%[allowframebreaks][shrink]
2、 假设\par



\qquad 对于时域离散系统,暂时假定$x(n)=e^{j\omega_{0}n}$傅里叶变换形式与模拟系统中的类似,可写作:
$$
e^{j\omega_0 n} \longleftrightarrow 2\pi\delta(\omega- \omega_0)
$$

但$e^{j\omega_{0}n}$为周期函数,有:
$$e^{j\omega_{0}n} = e^{j(\omega_{0}+2\pi r)n}  \qquad (r\in Z,\quad -\infty< r <\infty)$$%\quad\quad\mbox{式中,$r$ 取整数}$$
%\newpage
%$$e^{j\omega_{0}n} = e^{j(\omega_{0}+2\pi r)n}$$%\quad\quad\mbox{式中,$r$取整数}$$
%因为公式中,$r$取整数,
那么,其傅里叶变换也应该是一个个冲激的叠加。即为:
\begin{equation*}
X(e^{j\omega_{0}})=FT[e^{j\omega_{0}n}]=\sum_{r=-\infty}^{\infty}2\pi
\delta(\omega-\omega_{0}-2\pi r)
\end{equation*}
即:复指数序列$e^{j\omega_{0}n}$的傅里叶变换为在$\omega_{0}$处强度为$2\pi$的冲激函数,
且以$2\pi$为周期延拓而成。
\end{frame}



\begin{frame}[shrink]\frametitle{}%[allowframebreaks][shrink]
3、验证\par
\qquad 上述推测给出,
复指数序列$e^{j\omega_{0}n}$的傅里叶变换为在$\omega_{0}$处强度为$2\pi$的冲激函数,
且以$2\pi$为周期延拓而成。

\qquad 如果这种假设如果成立,则要求其反变换必须成立且唯一,并等于$e^{j\omega_{0}n}$,即验证:
$IFT[X(e^{j\omega})] = e^{j\omega_0 n}$
即:
\begin{equation*}
\begin{split}
x(n) &= IFT[X(e^{j\omega})] \\
&= \frac{1}{2\pi}\int_{-\pi}^{\pi}X(e^{j\omega})e^{j\omega n}d\omega\\
&= \frac{1}{2\pi}\int_{-\pi}^{\pi}\sum_{r=-\infty}^{\infty}2\pi\delta(\omega-\omega_{0}-2\pi r)e^{j\omega n}d\omega\\
&= \frac{1}{2\pi}\int_{-\pi}^{\pi}2\pi\cdot \delta(\omega-\omega_{0})e^{j\omega n}d\omega \qquad(\mbox{在$(-\pi,\pi)$之间})\\
&= \int_{-\pi}^{\pi}\delta(\omega-\omega_{0})e^{j\omega n}d\omega = e^{j\omega_0 n}
\end{split}
\end{equation*}

\end{frame}
%%%%%%%%%%%%%%%%%%%%%%%%%%%%%%%%%%%%%%%%%%%%%%%%%%%%%%%%%%%%%%%%%%%%%%%%%%%%%%%%%%%%%%%%%%%%%%%



%%%%%%%%%%%%%%%%%%%%%%%%%%%%%%%%%%%%%%%%%%%%%%%%%%%%%%%%%%%%%%%%%%%%%%%%%%%%%%%%%%%%%%%%%%%%%%
\begin{frame}[shrink]\frametitle{2 推导$x(n)=e^{j\omega_0 n}$的傅里叶变换}%[allowframebreaks][shrink]

教材讲授思路较直观,这里将直接推导$x(n)=e^{j\omega_0 n}$的傅里叶变换,可与教材相互印证。思路如下
\begin{dablock}
\begin{enumerate}
\item [(1)] 将$x(n)=e^{j\omega_0 n}$代入傅里叶变换公式:
$$X(e^{j\omega})= \sum_{n=-\infty}^{\infty}e^{j\omega_{0}n}e^{-j\omega n}
= \sum_{n=-\infty}^{\infty}e^{j(\omega_{0}-\omega) n}
$$
\item [(2)] 讨论:%这是一个等比数列求和的问题。
可分两种情况讨论。
\begin{enumerate}
\item [(a)] 当$\omega_{0}-\omega\neq 2\pi r$时,$r$是整数。
\item [(b)] 当$\omega_{0}-\omega= 2\pi r$时,$r$是整数。
\begin{enumerate}
\item 考察$X(e^{j\omega})$ 的形式
\item 考察$X(e^{j\omega})$ 的强度。
\end{enumerate}
\end{enumerate}
\item [(3)] 结论。
\end{enumerate}
\end{dablock}
%下面将求解这个等比数列求和问题。
\end{frame}



\begin{frame}[shrink]\frametitle{讨论:求解等比数列求和问题}%[allowframebreaks][shrink]
可分两种情况讨论。
\begin{enumerate}

\item [(a)] 当$\omega_{0}-\omega\neq 2\pi r$ 时,$r$是整数。
\begin{equation*}
\begin{split}
X(e^{j\omega})
&= \sum_{n=-\infty}^{\infty}e^{j(\omega_{0}-\omega) n}\\
&= \sum_{n=-\infty}^{0}e^{j(\omega_{0}-\omega) n}+
\sum_{n=0}^{\infty}e^{j(\omega_{0}-\omega) n} -1\\
(\mbox{令左边}n=-n)\quad  &= \sum_{n=0}^{\infty}e^{-j(\omega_{0}-\omega) n}+
\sum_{n=0}^{\infty}e^{j(\omega_{0}-\omega) n} -1\\
&= \frac{1}{1-e^{-j(\omega_{0}-\omega) }}+ \frac{1}{1-e^{j(\omega_{0}-\omega) }}-1\\
&= \frac{e^{j(\omega_{0}-\omega) }}{e^{j(\omega_{0}-\omega) }-1}+
\frac{-1}{e^{j(\omega_{0}-\omega) }-1}-1\\
&= \frac{e^{j(\omega_{0}-\omega) }-1}{e^{j(\omega_{0}-\omega) }-1}-1 = 0
\end{split}
\end{equation*}
% \item [(b)] 当$\omega_{0}-\omega= 2\pi r$ 时,$r$是整数的情况。
\end{enumerate}
\end{frame}





\begin{frame}[shrink]\frametitle{讨论:求解等比数列求和问题}%[allowframebreaks][shrink]
\begin{enumerate}
\item [(b)] 当$\omega_{0}-\omega= 2\pi r$ 时,$r$是整数。

\begin{enumerate}
\item 考察$X(e^{j\omega})$ 的形式:
$$X(e^{j\omega})= \sum_{n=-\infty}^{\infty}e^{j(\omega_{0}-\omega) n}= \sum_{n=-\infty}^{\infty}1 = \infty$$
可见,$X(e^{j\omega})$是一个间隔等于$2\pi$ 的冲激序列,而冲激出现在频率
$\omega = \omega_{0}-2r\pi$ 上。
\item 考察$X(e^{j\omega})$ 每个冲激的强度。可得到结论:\\
$X(e^{j\omega})$ 在$\omega-\omega_{0}= 2\pi r$ 处出现的都是强度为$2\pi$的冲激。
\end{enumerate}
\end{enumerate}
\end{frame}


\begin{frame}[shrink]\frametitle{说明: $X(e^{j\omega})$在$\omega-\omega_{0}= 2\pi r$ 处冲激强度}%[allowframebreaks][shrink]
%\begin{enumerate}
%\item [(b)]考察每个冲激的强度。\par
对$X(e^{j\omega})$在$\omega_{0}$附近的一个周期$(-\pi,\pi)$中求积分,可得:
\begin{equation*}
\begin{split}
\int_{\omega_{0}-\pi}^{\omega_{0}+\pi}X(e^{j\omega})d\omega
&= \int_{\omega_{0}-\pi}^{\omega_{0}+\pi}\sum_{n=-\infty}^{\infty}e^{j(\omega_{0}-\omega) n}d\omega \\
&= \sum_{n=-\infty}^{\infty}\int_{\omega_{0}-\pi}^{\omega_{0}+\pi}e^{j(\omega_{0}-\omega) n}d\omega \\
&= \sum_{n=-\infty}^{\infty}\int_{-\pi}^{\pi} e^{j \omega' n}d\omega' \quad(\mbox{令}\omega'= \omega_{0}-\omega)
\end{split}
\end{equation*}

\begin{equation*}
\mbox{我们知道:}    \int_{-\pi}^{\pi} e^{j \omega' n}d\omega' = 2\pi\delta(n)
= \left\{\begin{array}
{r@{,\quad}l}
2\pi&\mbox{$n=0$}\\0&\mbox{$n\neq 0$}
\end{array} \right.
\end{equation*}
$$\therefore\quad\quad\int_{\omega_{0}-\pi}^{\omega_{0}+\pi}X(e^{j\omega})d\omega=
\sum_{n=-\infty}^{\infty}2\pi\delta(n) = 2\pi$$
这说明$X(e^{j\omega})$在$\omega-\omega_{0}= 2\pi r,-\infty<r<\infty$处出现的都是
强度为$2\pi$的冲激。
%\end{enumerate}
\end{frame}



\begin{frame}[shrink]\frametitle{推导$x(n)=e^{j\omega_0 n}$的傅里叶变换}%[allowframebreaks][shrink]
\begin{enumerate}
\item [(3)] 结论:\par
所以$X(e^{j\omega})$是一个周期性为$2\pi$,强度为$2\pi$的冲激序列,即:
\begin{equation*}
FT[e^{j\omega_{0}n}] = \sum_{r=-\infty}^{\infty}2\pi\cdot\delta(\omega-\omega_{0}-2\pi r)
\end{equation*}
\end{enumerate}
\end{frame}
%%%%%%%%%%%%%%%%%%%%%%%%%%%%%%%%%%%%%%%%%%%%%%%%%%%%%%%%%%%%%%%%%%%%%%%%%%%%%%%%%%%%%%%%%%%%%%%


\subsubsection*{一般周期序列的傅里叶变换}
%%%%%%%%%%%%%%%%%%%%%%%%%%%%%%%%%%%%%%%%%%%%%%%%%%%%%%%%%%%%%%%%%%%%%%%%%%%%%%%%%%%%%%%%%%%%%%
\begin{frame}[shrink]\frametitle{二、一般周期序列的傅里叶变换}%[allowframebreaks][shrink]


\begin{enumerate}
\item [(1)] 公式
\end{enumerate}
$$\tilde{x}(n)     = \frac{1}{N}\sum_{k=0}^{N-1}\tilde{X}(k)e^{j\frac{2\pi}{N}kn}$$%\vspace{-0.3cm}
$$FT[\tilde{x}(n)] = \frac{1}{N}\sum_{k=0}^{N-1}\tilde{X}(k) FT\left[e^{j\frac{2\pi}{N}kn}\right]$$%\vspace{-0.3cm}
$$\mbox{又因为:\qquad}FT\left[e^{j\omega_{0}n}\right] = \sum_{r=-\infty}^{\infty}2\pi\cdot\delta(\omega-\omega_{0}-2\pi r)
\qquad\qquad\qquad\qquad$$
令\quad$\omega_{0} = \frac{2\pi}{N}k$ \quad
$$
FT[\tilde{x}(n)] = \frac{2\pi}{N}\sum_{k=0}^{N-1}\tilde{X}(k)
\sum_{r=-\infty}^{\infty}\delta(\omega-\frac{2\pi}{N}k-2\pi r)
$$

\end{frame}

\begin{frame}[shrink]\frametitle{公式的化简}%[allowframebreaks][shrink]
显然,一般周期序列的傅里叶变换如下:
$$FT[\tilde{x}(n)]
= \frac{2\pi}{N}\sum_{k=0}^{N-1}\tilde{X}(k)\sum_{r=-\infty}^{\infty}\delta(\omega-\frac{2\pi}{N}k-2\pi r)
$$

公式还可以化简化简后,有:
\begin{dablock}
$$FT[\tilde{x}(n)] = \frac{2\pi}{N}\sum_{k=-\infty}^{\infty}\tilde{X}(k)\delta(\omega-\frac{2\pi}{N}k)$$
\end{dablock}
\end{frame}


\begin{frame}[shrink]\frametitle{公式化简过程}%[allowframebreaks][shrink]
\begin{enumerate}
\item 引理: 设序列$\tilde{x}(n)$ 周期为N,则有
\end{enumerate}
\begin{equation*}
\sum_{n=0}^{mN-1}\tilde{x}(n) =
\sum_{r=0}^{m-1}\sum_{n'=0}^{N-1}\tilde{x}(n'+rN) \quad\mbox{其中$0\leqslant n'\leqslant N-1$}
\end{equation*}
\textbf{理解:}很简单,周期序列$\tilde{x}(n)$ 的m个周期的和,等于每个周期的和,再乘以周期的个数($m$)。

\end{frame}



\begin{frame}[shrink]\frametitle{公式化简过程}%[allowframebreaks][shrink]
$$
\mbox{引理:}\qquad          \sum_{n=0}^{mN-1}\tilde{x}(n) =
\sum_{r=0}^{m-1}\sum_{n'=0}^{N-1}\tilde{x}(n'+rN) \quad\mbox{其中$0\leqslant n'\leqslant N-1$}
$$
%$$\frac{2\pi}{N}\sum_{k=-\infty}^{\infty}\tilde{X}(k)\delta(\omega-\frac{2\pi}{N}k)\qquad
%        \qquad\qquad\qquad\qquad\qquad\qquad\qquad\qquad$$

\begin{equation*}
\begin{split}
\frac{2\pi}{N}\sum_{k=-\infty}^{\infty}\tilde{X}(k)\delta(\omega-\frac{2\pi}{N}k)
&= \frac{2\pi}{N}\sum_{r=-\infty}^{\infty}\sum_{k'=0}^{N-1}\tilde{X}(k'+rN)\delta(\omega-\frac{2\pi}{N}(k'+rN))\\
&= \frac{2\pi}{N}\sum_{r=-\infty}^{\infty}\sum_{k'=0}^{N-1}\tilde{X}(k') \delta(\omega-\frac{2\pi}{N}k'-2\pi r)\\
&= \frac{2\pi}{N}\sum_{k'=0}^{N-1}\tilde{X}(k')\sum_{r=-\infty}^{\infty}\delta(\omega-\frac{2\pi}{N}k'-2\pi r)\\
(\mbox{将}  k'\rightarrow k) \qquad
&= \frac{2\pi}{N}\sum_{k=0}^{N-1}\tilde{X}(k)\sum_{r=-\infty}^{\infty}\delta(\omega-\frac{2\pi}{N}k-2\pi r)
\end{split}
\end{equation*}

\end{frame}



\begin{frame}[shrink]\frametitle{求$\tilde{x}(n)$傅里叶变换的过程}%[allowframebreaks][shrink]
求$\tilde{x}(n)$傅里叶变换的过程:

\begin{enumerate}
\item 求其离散傅里叶级数$\tilde{X}(k)$%= DFS[\tilde{x}(n)]$,
\item 套公式。
\begin{dablock}
$$FT\big[\tilde{x}(n)\big] = \frac{2\pi}{N}\sum_{k=-\infty}^{\infty}\tilde{X}(k)\delta(\omega-\frac{2\pi}{N}k)$$
\end{dablock}
\end{enumerate}
\end{frame}
%%%%%%%%%%%%%%%%%%%%%%%%%%%%%%%%%%%%%%%%%%%%%%%%%%%%%%%%%%%%%%%%%%%%%%%%%%%%%%%%%%%%%%%%%%%%%%%



%%%%%%%%%%%%%%%%%%%%%%%%%%%%%%%%%%%%%%%%%%%%%%%%%%%%%%%%%%%%%%%%%%%%%%%%%%%%%%%%%%%%%%%%%%%%%%%
%\begin{frame}[allowframebreaks]\frametitle{例题}%[allowframebreaks][shrink]
%\begin{example}
%设$\tilde{x}(n)$如下图所示,求$FT[\tilde{x}(n)]$.
%\newline\newline\newline\newline
%\end{example}
%\par\textbf{解}:$\tilde{x}(n)$为周期序列,且N=8,有:
%     $$\tilde{X}(e^{j\omega})= FT[\tilde{x}(n)]
%                             = \frac{2\pi}{N}\sum_{k=-\infty}^{\infty}\tilde{X}(k)\delta(\omega-\frac{2\pi}{N}k)$$
%     $$\mbox{其中:\quad\quad} \tilde{X}(k) = DFS[\tilde{x}(n)]  = \sum_{n=0}^{N-1}\tilde{x}(n)e^{-j\frac{2\pi}{N}kn}$$
%     \begin{equation*}
%        \begin{split}
%        \tilde{X}(k) &= \sum_{n=0}^{N-1}\tilde{x}(n)e^{-j\frac{2\pi}{N}kn} \quad\quad (N =8)  \\
%                     &= \sum_{n=0}^{3}e^{-j\frac{2\pi}{8}kn} = \frac{1-[e^{-j\frac{2\pi}{8}k}]^{4}}{1-e^{-j\frac{2\pi}{8}k}} \\
%                     &= \frac{e^{-j\frac{4\pi}{8}k}(e^{j\frac{4\pi}{8}k} - e^{-j\frac{4\pi}{8}k})}
%                             {e^{-j\frac{\pi}{8}k} (e^{j\frac{ \pi}{8}k} - e^{-j\frac{ \pi}{8}k})}\\
%                     &= \frac{\sin(\frac{\pi}{2}k)}{\sin(\frac{\pi}{8}k)}\cdot e^{-j\frac{3\pi}{8}k}
%        \end{split}
%     \end{equation*}
%     $$\tilde{X}(e^{j\omega})=  \frac{\pi}{4}\sum_{k=-\infty}^{\infty} e^{-j\frac{3\pi}{8}k}
%                             \frac{\sin(\frac{\pi}{2}k)}{\sin(\frac{\pi}{8}k)} \delta(\omega-\frac{\pi}{4}k)$$
%     \newpage
%     画出幅频特性图(考虑一个周期)
%     \begin{enumerate}
%       \item $k=0$ 时,对应$\delta(\omega)$,               强度为$\pi$
%       \item $k=1$ 时,对应$\delta(\omega-\frac{1}{4}\pi)$,强度为 $\frac{\pi/4}{sin(\pi/8)}$
%       \item $k=2$ 时,对应$\delta(\omega-\frac{1}{2}\pi)$,强度为 $0$
%       \item $k=3$ 时,对应$\delta(\omega-\frac{3}{4}\pi)$,强度为 $\frac{\pi/4}{sin(3\pi/8)}$
%       \item $k=4$ 时,对应$\delta(\omega-\pi)$,           强度为 $0$
%       \item $k=5$ 时,对应$\delta(\omega-\frac{5}{4}\pi)$,强度为 $\frac{\pi/4}{sin(3\pi/8)}$
%       \item $k=6$ 时,对应$\delta(\omega-\frac{3}{2}\pi)$,强度为 $0$
%       \item $k=7$ 时,对应$\delta(\omega-\frac{7}{4}\pi)$,强度为 $\frac{\pi/4}{sin(\pi/8)}$
%       \item $k=8$ 时,对应$\delta(\omega-2\pi)$,          强度为 $\pi$
%     \end{enumerate}
%     \newpage
%     幅频特性图如下:
%    \begin{figure}
%      \centering
%      % Requires \usepackage{graphicx}
%      \includegraphics[width=0.8\textwidth]{blankpic.jpg}
%      %\caption{}\label{}
%    \end{figure}
%
%
%\end{frame}
%%%%%%%%%%%%%%%%%%%%%%%%%%%%%%%%%%%%%%%%%%%%%%%%%%%%%%%%%%%%%%%%%%%%%%%%%%%%%%%%%%%%%%%%%%%%%%%%
%
%
%
%%%%%%%%%%%%%%%%%%%%%%%%%%%%%%%%%%%%%%%%%%%%%%%%%%%%%%%%%%%%%%%%%%%%%%%%%%%%%%%%%%%%%%%%%%%%%%%
%\begin{frame}\frametitle{例题}%[allowframebreaks][shrink]
%\begin{example}
%设$x(n) = a^{n}u(n)$如下图所示,求$FT[x(n)]$.
%\par\textbf{解}:
%\begin{equation}
%    \begin{split}
%    X(e^{j\omega}) &= \sum_{n=-\infty}^{\infty}a^{n}u(n) e^{-j\omega n} = \sum_{n=0}^{\infty}a^{n} e^{-j\omega n}
%                   = \sum_{n=0}^{\infty}(ae^{-j\omega})^{n} \\
%                   &= \frac{1}{1- ae^{-j\omega} }
%    \end{split}
%\end{equation}
%\end{example}
%\end{frame}
%%%%%%%%%%%%%%%%%%%%%%%%%%%%%%%%%%%%%%%%%%%%%%%%%%%%%%%%%%%%%%%%%%%%%%%%%%%%%%%%%%%%%%%%%%%%%%%%














%%%%%%%%%%%%%%%%%%%%%%%%%%%%%%%%%%%%%%%%%%%%%%%%%%%%%%%%%%%%%%%%%%%%%%%%%%%%%%%%%%%%%%%%%%%%%%%

\section{2.3 序列的傅里叶变换的性质}

\subsection*{FT的周期性}

%%%%%%%%%%%%%%%%%%%%%%%%%%%%%%%%%%%%%%%%%%%%%%%%%%%%%%%%%%%%%%%%%%%%%%%%%%%%%%%%%%%%%%%%%%%%%%
\begin{frame}[shrink]\frametitle{FT的周期性}%[allowframebreaks][shrink]
\begin{dablock}
在FT的定义式中,$n$取整数,显然有
\begin{equation*}
X(e^{j\omega}) = \sum_{n=-\infty}^{\infty}x(n)e^{-j\omega n} = \sum_{n=-\infty}^{\infty}x(n)e^{-j(\omega+2\pi M)n}
\end{equation*}
成立,式中$M$为整数,显然$X(e^{j\omega})$ 是$\omega$ 的周期函数,且周期为$2\pi$.

\end{dablock}

\end{frame}
%%%%%%%%%%%%%%%%%%%%%%%%%%%%%%%%%%%%%%%%%%%%%%%%%%%%%%%%%%%%%%%%%%%%%%%%%%%%%%%%%%%%%%%%%%%%%%%


\subsection*{线性}
%%%%%%%%%%%%%%%%%%%%%%%%%%%%%%%%%%%%%%%%%%%%%%%%%%%%%%%%%%%%%%%%%%%%%%%%%%%%%%%%%%%%%%%%%%%%%%
\begin{frame}\frametitle{线性}%[allowframebreaks][shrink]
\begin{dablock}
设$X_{1}(e^{j\omega})= FT[x_{1}(n)]$,$X_{2}(e^{j\omega})= FT[x_{2}(n)]$,那么
\begin{equation*}
FT[ax_{1}(n)+bx_{2}(n)] = aX_{1}(e^{j\omega}) + bX_{2}(e^{j\omega})
\end{equation*}
式中$a,b$为常数。
\end{dablock}
\end{frame}
%%%%%%%%%%%%%%%%%%%%%%%%%%%%%%%%%%%%%%%%%%%%%%%%%%%%%%%%%%%%%%%%%%%%%%%%%%%%%%%%%%%%%%%%%%%%%%%


%\subsection{线性}
%%%%%%%%%%%%%%%%%%%%%%%%%%%%%%%%%%%%%%%%%%%%%%%%%%%%%%%%%%%%%%%%%%%%%%%%%%%%%%%%%%%%%%%%%%%%%%%
%\begin{frame}\frametitle{title}%[allowframebreaks][shrink]
%
%设$X_{1}(e^{j\omega})= FT[x_{1}(n)]$,$X_{2}(e^{j\omega})= FT[x_{2}(n)]$,那么
%\begin{equation}
%    FT[ax_{1}(n)+bx_{2}(n)] = aX_{1}(e^{j\omega}) + bX_{2}(e^{j\omega})
%\end{equation}
%式中$a,b$为常数。
%\end{frame}
%%%%%%%%%%%%%%%%%%%%%%%%%%%%%%%%%%%%%%%%%%%%%%%%%%%%%%%%%%%%%%%%%%%%%%%%%%%%%%%%%%%%%%%%%%%%%%%%


\subsection*{时移与频移}
%%%%%%%%%%%%%%%%%%%%%%%%%%%%%%%%%%%%%%%%%%%%%%%%%%%%%%%%%%%%%%%%%%%%%%%%%%%%%%%%%%%%%%%%%%%%%%
\begin{frame}\frametitle{时移与频移性质}%[allowframebreaks][shrink]

设$X(e^{j\omega})= FT[x(n)]$,那么,%\vspace{-1cm}
\begin{equation*}
\begin{split}
FT[x(n-n_{0})]           &= e^{-j\omega n_{0}}X(e^{j\omega})    \\
FT[e^{j\omega_{0}n}x(n)] &= X(e^{j(\omega-\omega_{0})})\qquad\qquad\qquad\qquad
\end{split}
\end{equation*}
\begin{shuoming}
\begin{enumerate}
\item 信号在时域的延时和在频域中的移相相对应,对信号的幅度不产生影响。
\item 信号在时域乘以因子$e^{j\omega_{0}n}$, 等于在频域中将整个频谱向频率增加方向搬移了$\omega_{0}$。
\end{enumerate}
\end{shuoming}
\end{frame}
%%%%%%%%%%%%%%%%%%%%%%%%%%%%%%%%%%%%%%%%%%%%%%%%%%%%%%%%%%%%%%%%%%%%%%%%%%%%%%%%%%%%%%%%%%%%%%%



%%%%%%%%%%%%%%%%%%%%%%%%%%%%%%%%%%%%%%%%%%%%%%%%%%%%%%%%%%%%%%%%%%%%%%%%%%%%%%%%%%%%%%%%%%%%%%
\begin{frame}[shrink]\frametitle{时移定理的证明}%[allowframebreaks][shrink]
$$FT[x(n-n_0)] = \sum_{n=-\infty}^{\infty}x(n-n_0)\cdot e^{-j\omega n}$$
\quad\quad 令$n-n_0 = m$,则有
\begin{equation*}
\begin{split}
FT[x(n-n_0)] &= \sum_{m=-\infty}^{\infty}x(m)\cdot e^{-j\omega(n_0 +m)}\\
&= \sum_{m=-\infty}^{\infty}x(m)\cdot e^{-j\omega n_0}\cdot e^{-j\omega m}\\
&= e^{-j\omega n_0}\cdot\sum_{m=-\infty}^{\infty}x(m)\cdot e^{-j\omega m}\\
&= e^{-j\omega n_0}\cdot X(e^{j\omega})
\end{split}
\end{equation*}
\end{frame}


\begin{frame}[shrink]\frametitle{频移定理的证明}%[allowframebreaks][shrink]
从左到右,有
\begin{equation*}
\begin{split}
FT[e^{j\omega_{0}n}x(n)]
&= \sum_{n=-\infty}^{\infty}x(n)\cdot e^{j\omega_{0}n}\cdot e^{-j\omega n}\\
&= \sum_{n=-\infty}^{\infty}x(n)\cdot e^{-j(\omega-\omega_{0})n}\\
&= X(e^{j(\omega-\omega_{0})})
\end{split}
\end{equation*}

\end{frame}
%%%%%%%%%%%%%%%%%%%%%%%%%%%%%%%%%%%%%%%%%%%%%%%%%%%%%%%%%%%%%%%%%%%%%%%%%%%%%%%%%%%%%%%%%%%%%%%
\subsection*{FT的对称性}
%%%%%%%%%%%%%%%%%%%%%%%%%%%%%%%%%%%%%%%%%%%%%%%%%%%%%%%%%%%%%%%%%%%%%%%%%%%%%%%%%%%%%%%%%%%%%%
\begin{frame}\frametitle{共轭对称序列与共轭反对称序列}%[allowframebreaks][shrink]
\begin{definition}
\begin{enumerate}
\item [(1)] 共轭对称序列\par\qquad
如序列$x_{e}(n)$满足:
$$x_{e}(n) = x_{e}^{*}(-n)$$
\qquad 则称$x_{e}(n)$为共轭对称序列.
\item [(2)] 共轭反对称序列\par\qquad
如序列$x_{o}(n)$满足:
$$x_{o}(n) = -x_{o}^{*}(-n)$$
\qquad 则称$x_{o}(n)$为共轭反对称序列.
\end{enumerate}
\end{definition}
\end{frame}
%%%%%%%%%%%%%%%%%%%%%%%%%%%%%%%%%%%%%%%%%%%%%%%%%%%%%%%%%%%%%%%%%%%%%%%%%%%%%%%%%%%%%%%%%%%%%%%



%%%%%%%%%%%%%%%%%%%%%%%%%%%%%%%%%%%%%%%%%%%%%%%%%%%%%%%%%%%%%%%%%%%%%%%%%%%%%%%%%%%%%%%%%%%%%%
\begin{frame}\frametitle{共轭对称序列的性质}%[allowframebreaks][shrink]

\begin{enumerate}
\item [(1)] \textbf{共轭对称序列的实部是偶函数,而虚部是奇函数}。
\par 证明:
将序列$x_{e}(n)$用实部与虚部表示:%\vspace{-1cm}
$$x_{e}(n) = x_{er}(n) + jx_{ei}(n)$$
则有: \quad\quad\quad\quad
$x_{e}^{*}(-n) = x_{er}(-n) - jx_{ei}(-n)$
\par 根据共轭对称序列的定义式,有:
\begin{equation*}
\left\{ \begin{aligned}
x_{er}(n) &= x_{er}(-n)\\
x_{ei}(n) &= -x_{ei}(-n)
\end{aligned} \right.
\end{equation*}
\pause
\item [(2)]  \textbf{共轭反对称序列的实部是奇函数,而虚部是偶函数}。
\par 证明:
将序列$x_{o}(n)$用实部与虚部表示:
$$x_{o}(n) = x_{or}(n) + jx_{oi}(n)$$\vspace{-0.5cm}
同理可得:
\begin{equation*}
\left\{ \begin{aligned}
x_{or}(n) &= -x_{or}(-n)\\
x_{oi}(n) &= x_{oi}(-n)
\end{aligned} \right.
\end{equation*}
\end{enumerate}
\end{frame}
%%%%%%%%%%%%%%%%%%%%%%%%%%%%%%%%%%%%%%%%%%%%%%%%%%%%%%%%%%%%%%%%%%%%%%%%%%%%%%%%%%%%%%%%%%%%%%%



%%%%%%%%%%%%%%%%%%%%%%%%%%%%%%%%%%%%%%%%%%%%%%%%%%%%%%%%%%%%%%%%%%%%%%%%%%%%%%%%%%%%%%%%%%%%%%
\begin{frame}\frametitle{序列可分解为共轭对称序列与共轭反对称序列之和}%[allowframebreaks][shrink]

%  \begin{enumerate}
%    \item [(1)] 一般序列都可以分解为共轭对称序列与共轭反对称序列之和,即:
可假设
\begin{equation*}
x(n) = x_{e}(n) + x_{o}(n)%\vspace{-0.5cm}
\end{equation*}
%而式中的$x_{e}(n)$和$x_{o}(n)$ 都可以用原序列$x(n)$求出。
将上式中的$n$用$-n$代替,可得
\begin{equation*}
x^{*}(-n) = x^{*}_{e}(-n) + x^{*}_{o}(-n) = x_{e}(n) - x_{o}(n)
\end{equation*}
对上述两个方程联立求解可得:
\begin{equation*}
\left\{ \begin{aligned}
x_{e}(n) &= \frac{1}{2}[x(n) + x^{*}(-n)]\\
x_{o}(n) &= \frac{1}{2}[x(n) - x^{*}(-n)]
\end{aligned} \right.
\end{equation*}
\end{frame}



\begin{frame}\frametitle{序列可分解为实部序列和虚部序列之和}%[allowframebreaks][shrink]
%
%    \item [(2)]
%    很显然,每个序列$x(n)$总能分解为实部序列$x_{r}(n)$ 和虚部序列$x_{i}(n)$,即:
可设:
$$x(n) = x_{r}(n)+j x_{i}(n)$$
则有:
$$x^*(n) = x_{r}(n)-j x_{i}(n)$$

联立求解上述方程组,则实部序列$x_{r}(n)$ 和虚部序列$x_{i}(n)$总能用原序列表示为:
\begin{equation*}
\left\{ \begin{aligned}
x_{r}(n) &= \frac{1}{2}[x(n) + x^{*}(n)]\\
jx_{i}(n) &= \frac{1}{2}[x(n) - x^{*}(n)]
\end{aligned} \right.
\end{equation*}
% \end{enumerate}
\end{frame}
%%%%%%%%%%%%%%%%%%%%%%%%%%%%%%%%%%%%%%%%%%%%%%%%%%%%%%%%%%%%%%%%%%%%%%%%%%%%%%%%%%%%%%%%%%%%%%%

\begin{frame}[shrink]\frametitle{FT对称性的讨论}%[allowframebreaks][shrink]
\begin{enumerate}

\item [(1)] 时域离散序列$x(n)$可分解为实部序列与虚部序列之和,同时也能分解为共轭对称序列与共轭反对称序列之和。
\item [(2)] 序列$x(n)$的傅里叶变换$X(e^{j\omega)})$ 同样可分解为实部序列与虚部序列之和,也同时也能分解为共轭对称序列与共轭反对称序列之和。
\end{enumerate}

\begin{wenti}
那么,他们之间的关系如何呢?我们将从下面几个方面考虑。

\begin{enumerate}
\item  实序列与虚序列的傅里叶变换
%\par 设$\quad\quad x(n) = x_{r}(n)+jx_{i}(n), \quad X(e^{j\omega}) = FT[x(n)]$
%        \begin{enumerate}
%            \item 时域序列的实部部分的傅里叶变换,对应其频域序列的共轭对称部分。
%            \item 时域序列的虚部部分的傅里叶变换后,对应其频域序列的共轭反对称部分。
%        \end{enumerate}
\item 共轭对称序列与共轭反对称序列的傅里叶变换
%        \begin{enumerate}
%            \item 时域序列的共轭对称部分的傅里叶变换后,对应其频域序列的实部部分。
%            \item 时域序列的共轭反对称部分的傅里叶变换对应其频域序列的虚部部分。
%        \end{enumerate}
\end{enumerate}
\end{wenti}
\end{frame}


%%%%%%%%%%%%%%%%%%%%%%%%%%%%%%%%%%%%%%%%%%%%%%%%%%%%%%%%%%%%%%%%%%%%%%%%%%%%%%%%%%%%%%%%%%%%%%
\begin{frame}[shrink]\frametitle{时域序列的实数部分的傅里叶变换}%[allowframebreaks][shrink]
\begin{wenti}
前述时域序列可分解为实部序列与虚部序列之和,即:
$$x(n) = x_{r}(n)+j x_{i}(n) \qquad\qquad     \mbox{且有:} X(e^{j\omega}) = FT[x(n)] $$
% 时域序列的实部部分的傅里叶变换,对应其频域序列的共轭对称部分,即:
%    $$x_{r}(n)\quad \Longleftrightarrow \quad X_{e}(e^{j\omega}) $$
那么,实部序列$x_{r}(n)$的傅里叶变换,对应$X(e^{j\omega})$那个部分呢?
\end{wenti}
\par 前述有:\quad\quad $ x_{r}(n) = \frac{1}{2}[x(n) + x^{*}(n)]$
$$FT[x_{r}(n)] = \frac{1}{2}(FT[x(n)] + FT[x^{*}(n)])$$
而
\begin{equation*}
\begin{split}
FT[x^{*}(n)])
&= \sum_{n=-\infty}^{\infty}x^{*}(n)e^{-j\omega n}
= \left[\sum_{n=-\infty}^{\infty}x(n)e^{-j(-\omega) n}\right]^{*}
= X^{*}(e^{-j\omega})
\end{split}
\end{equation*}
%    那么
%    \begin{equation*}
%    \begin{split}
%    FT[x_{r}(n)] &= \frac{1}{2}(FT[x(n)] + FT[x^{*}(n)])
%                                    = \frac{1}{2}\left[X(e^{j\omega}) + X^{*}(e^{-j\omega})\right] \\
%                                   &= X_{e}(e^{j\omega})
%    \end{split}
%    \end{equation*}

\end{frame}



\begin{frame}[shrink]\frametitle{ 时域序列的实数部分的傅里叶变换}%[allowframebreaks][shrink]

那么
\begin{equation*}
\begin{split}
FT[x_{r}(n)] &= \frac{1}{2}(FT[x(n)] + FT[x^{*}(n)]) \qquad\qquad\qquad \\
&= \frac{1}{2}\left[X(e^{j\omega}) + X^{*}(e^{-j\omega})\right] \\
&= X_{e}(e^{j\omega})
\end{split}
\end{equation*}

\begin{jielun}
时域序列的实部部分的傅里叶变换,对应其频域序列的共轭对称部分,即:
$$x_{r}(n)\quad \Longleftrightarrow \quad X_{e}(e^{j\omega}) $$
\end{jielun}
\end{frame}




\begin{frame}[shrink]\frametitle{ 时域序列的虚部部分的傅里叶变换}%[allowframebreaks][shrink]
%时域序列的虚部部分的傅里叶变换后,对应其频域序列的共轭反对称部分,即:
%$$jx_i(n) \Longleftrightarrow X_{o}(e^{j\omega})$$
同理,有:
\begin{equation*}
\begin{split}
FT[jx_{i}(n)] &= \frac{1}{2}(FT[x(n)] - FT[x^{*}(n)]) \qquad\qquad\qquad\\
&= \frac{1}{2}\left[X(e^{j\omega}) - X^{*}(e^{-j\omega})\right] \\
&= X_{o}(e^{j\omega})
\end{split}
\end{equation*}

\begin{jielun}
时域序列的虚数部分的傅里叶变换,对应其频域序列的共轭反对称部分,即:
$$jx_i(n) \Longleftrightarrow X_{o}(e^{j\omega})$$
\end{jielun}



\end{frame}
%%%%%%%%%%%%%%%%%%%%%%%%%%%%%%%%%%%%%%%%%%%%%%%%%%%%%%%%%%%%%%%%%%%%%%%%%%%%%%%%%%%%%%%%%%%%%%%


%%%%%%%%%%%%%%%%%%%%%%%%%%%%%%%%%%%%%%%%%%%%%%%%%%%%%%%%%%%%%%%%%%%%%%%%%%%%%%%%%%%%%%%%%%%%%%
\begin{frame}[shrink]\frametitle{时域序列的共轭对称部分的傅里叶变换}%[allowframebreaks][shrink]

\begin{wenti}
序列$x(n)$总可分为共轭对称部分$x_{e}(n)$ 和共轭反对称部分$x_{o}(n)$之和,即:
$$x(n) = x_{e}(n)+x_{o}(n) \qquad\qquad     \mbox{且有:} X(e^{j\omega}) = FT[x(n)] $$
% 时域序列的实部部分的傅里叶变换,对应其频域序列的共轭对称部分,即:
%    $$x_{r}(n)\quad \Longleftrightarrow \quad X_{e}(e^{j\omega}) $$
那么,共轭对称序列$x_{e}(n)$和共轭反对称序列$x_{o}(n)$ 的傅里叶变换,对应$X(e^{j\omega})$那个部分呢?
\end{wenti}

%共轭对称序列与共轭反对成序列的傅里叶变换
%\par 任意序列$x(n)$总可分为共轭对称部分$x_{e}(n)$和共轭反对称部分$x_{o}(n)$之和,即,
%$$x(n) = x_{e}(n)+x_{o}(n)$$
%那么有:
%\begin{enumerate}
%\item 时域序列的共轭对称部分的傅里叶变换后,对应其频域序列的实部部分,即:
%                  $$x_{e}(n) \Longleftrightarrow X_{R}(e^{j\omega})$$
%证明:
\par 前述有:\quad $ x_{e}(n) = \frac{1}{2}[x(n) + x^{*}(-n)]$,
$$FT[x_{e}(n)] = \frac{1}{2}(FT[x(n)] + FT[x^{*}(-n)])$$
\begin{equation*}
\begin{split}
FT[x^{*}(-n)])&= \sum_{n=-\infty}^{\infty}x^{*}(-n)e^{-j\omega n}
= \sum_{m=-\infty}^{\infty}x^{*}(m)e^{j\omega m}\quad\mbox{(令$m=-n$})\\
&= \left[\sum_{m=-\infty}^{\infty}x(m)e^{-j\omega m}\right]^{*}
= X^{*}(e^{j\omega})
\end{split}
\end{equation*}
\end{frame}

\begin{frame}[shrink]\frametitle{时域序列的共轭对称部分的傅里叶变换}%[allowframebreaks][shrink]
那么
\begin{equation*}
\begin{split}
FT[x_{e}(n)] &= \frac{1}{2}(FT[x(n)] + FT[x^{*}(-n)])
= \frac{1}{2}\left[X(e^{j\omega}) + X^{*}(e^{j\omega})\right] \\
&= X_{R}(e^{j\omega}) %= RE[X(e^{j\omega})]
\end{split}
\end{equation*}


\begin{jielun}
时域序列的共轭对称部分的傅里叶变换,对应其频域序列的实部部分,即:
$$x_{e}(n) \Longleftrightarrow X_{R}(e^{j\omega})$$
\end{jielun}
\end{frame}





\begin{frame}[shrink]\frametitle{时域序列的共轭反对称部分的傅里叶变换}%[allowframebreaks][shrink]


%              \item 时域序列的共轭反对称部分的傅里叶变换对应其频域序列的虚部部分,即:
%              $$x_{o}(n) \Longleftrightarrow jX_{I}(e^{j\omega})$$
同理,有:
\begin{equation*}
\begin{split}
FT[x_{o}(n)] &= \frac{1}{2}(FT[x(n)] - FT[x^{*}(-n)])\qquad\qquad\qquad\qquad \\
&= \frac{1}{2}\left[X(e^{j\omega}) - X^{*}(e^{j\omega})\right] \\
&= jX_{I}(e^{j\omega}) %= IM[X(e^{j\omega})]
\end{split}
\end{equation*}
%          \end{enumerate}
\begin{jielun}
时域序列的共轭反对称部分的傅里叶变换对应其频域序列的虚部部分,即:
$$x_{o}(n) \Longleftrightarrow jX_{I}(e^{j\omega})$$
\end{jielun}
\end{frame}
%%%%%%%%%%%%%%%%%%%%%%%%%%%%%%%%%%%%%%%%%%%%%%%%%%%%%%%%%%%%%%%%%%%%%%%%%%%%%%%%%%%%%%%%%%%%%%%





%%%%%%%%%%%%%%%%%%%%%%%%%%%%%%%%%%%%%%%%%%%%%%%%%%%%%%%%%%%%%%%%%%%%%%%%%%%%%%%%%%%%%%%%%%%%%%
\begin{frame}\frametitle{结论}%[allowframebreaks][shrink]

\begin{enumerate}
\item [1]一个域的实信号,对应另一个域的共轭对称信号,反之亦然;
\newline
\item [2]一个域的虚信号,对应另一个域的共轭反对称信号,反之亦然。
\end{enumerate}
%\begin{figure}
%\centering
%\includegraphics[width=0.4\textwidth]{blankpic.jpg}
%\end{figure}


\end{frame}
%%%%%%%%%%%%%%%%%%%%%%%%%%%%%%%%%%%%%%%%%%%%%%%%%%%%%%%%%%%%%%%%%%%%%%%%%%%%%%%%%%%%%%%%%%%%%%%
%\begin{figure}
%\centering
%\includegraphics[width=0.8\textwidth]{blankpic.jpg}
%\end{figure}


%%%%%%%%%%%%%%%%%%%%%%%%%%%%%%%%%%%%%%%%%%%%%%%%%%%%%%%%%%%%%%%%%%%%%%%%%%%%%%%%%%%%%%%%%%%%%%
\begin{frame}[shrink]\frametitle{分析实序列$h(n)$的傅里叶变换$H(e^{j\omega})$的特性。}%[allowframebreaks][shrink]
%\textbf{分析实序列$h(n)$的傅里叶变换$H(e^{j\omega})$ 的特性。}
\begin{enumerate}
\item [(1)] $\because\quad  h(n)$ 为实序列.
\par $\therefore$\quad $H(e^{j\omega})$ 为共轭对称序列,即:$H(e^{j\omega})=H^{*}(e^{-j\omega})$
$$\mbox{设:\quad\quad\quad} H(e^{j\omega}) = H_{R}(e^{j\omega}) + jH_{I}(e^{j\omega})\quad\quad\quad\quad\quad\quad$$
则:$H_{R}(e^{j\omega})$为偶函数,而$H_{I}(e^{j\omega})$ 为奇函数。
$$\mbox{设:\quad\quad\quad} H(e^{j\omega}) = |H(e^{j\omega})|\cdot e^{j\varphi(\omega)}\quad\quad\quad\quad\quad\quad$$
$$\therefore \quad\quad |H(e^{j\omega})| = \sqrt{H_{R}^{2}(e^{j\omega}) +H_{I}^{2}(e^{j\omega})}$$
$$\quad\quad \varphi(\omega) = \arctan\left(\frac{H_{I}(e^{j\omega})}{H_{R}(e^{j\omega})}\right)$$
显然 $|H(e^{j\omega})|$ 是偶函数,而$\theta(\omega)$ 为奇函数。
\item [(2)]  \textbf{结论:}\par
实函数的幅频函数是偶函数,相频函数是奇函数,与模拟系统有相同的结论。
\end{enumerate}

%\begin{example}
%利用FT的对称性,分析实因果序列$h(n)$的对称性,并推导其偶函数$h_{e}(n)$和奇函数$h_{o}(n)$
%与$h(n)$之间的关系
%\par\textbf{解}:
%\end{example}
\end{frame}
%%%%%%%%%%%%%%%%%%%%%%%%%%%%%%%%%%%%%%%%%%%%%%%%%%%%%%%%%%%%%%%%%%%%%%%%%%%%%%%%%%%%%%%%%%%%%%%

\subsection*{卷积定理}
%%%%%%%%%%%%%%%%%%%%%%%%%%%%%%%%%%%%%%%%%%%%%%%%%%%%%%%%%%%%%%%%%%%%%%%%%%%%%%%%%%%%%%%%%%%%%%
\begin{frame}[shrink]\frametitle{时域卷积定理}%[allowframebreaks][shrink]

\begin{theorem}

时域两信号卷积,转换到频域服从相乘关系,即设$$y(n)=x(n)*h(n)$$
则有
\begin{equation*}
Y(e^{j\omega})=X(e^{j\omega})\cdot H(e^{j\omega})
\end{equation*}
\end{theorem}
%        \newpage
%        \textbf{证明:}
%          显然:$$y(n) = x(n)*h(n) = \sum_{m=-\infty}^{\infty}x(m)h(n-m)$$
%          $$Y(e^{j\omega})= FT[y(n)] = \sum_{n=-\infty}^{\infty}\left[\sum_{m=-\infty}^{\infty}x(m)h(n-m)\right]
%          e^{-j\omega n}$$
%          \quad\quad 交换求和序:
%          $$Y(e^{j\omega})= \sum_{m=-\infty}^{\infty}x(m)\left[\sum_{n=-\infty}^{\infty}h(n-m)e^{-j\omega n}\right]$$
%          $$\mbox{而}\quad \sum_{n=-\infty}^{\infty}h(n-m)e^{-j\omega n} = FT[h(n-m)]
%            = e^{-j\omega m}\cdot H(e^{j\omega})$$
%          $$Y(e^{j\omega})= \sum_{m=-\infty}^{\infty}x(m)e^{-j\omega m}\cdot H(e^{j\omega})
%                          = X(e^{j\omega})\cdot H(e^{j\omega})$$
\end{frame}
%%%%%%%%%%%%%%%%%%%%%%%%%%%%%%%%%%%%%%%%%%%%%%%%%%%%%%%%%%%%%%%%%%%%%%%%%%%%%%%%%%%%%%%%%%%%%%%



%%%%%%%%%%%%%%%%%%%%%%%%%%%%%%%%%%%%%%%%%%%%%%%%%%%%%%%%%%%%%%%%%%%%%%%%%%%%%%%%%%%%%%%%%%%%%%
\begin{frame}[shrink]\frametitle{}%[allowframebreaks][shrink]

\textbf{证明:}
显然:$$y(n) = x(n)*h(n) = \sum_{m=-\infty}^{\infty}x(m)h(n-m)$$
$$Y(e^{j\omega})= FT[y(n)] = \sum_{n=-\infty}^{\infty}\left[\sum_{m=-\infty}^{\infty}x(m)h(n-m)\right]
e^{-j\omega n}$$
\quad\quad 交换求和序:
$$Y(e^{j\omega})= \sum_{m=-\infty}^{\infty}x(m)\left[\sum_{n=-\infty}^{\infty}h(n-m)e^{-j\omega n}\right]$$
$$\mbox{而}\quad \sum_{n=-\infty}^{\infty}h(n-m)e^{-j\omega n} = FT[h(n-m)]
= e^{-j\omega m}\cdot H(e^{j\omega})$$
$$Y(e^{j\omega})= \sum_{m=-\infty}^{\infty}x(m)e^{-j\omega m}\cdot H(e^{j\omega})
= X(e^{j\omega})\cdot H(e^{j\omega})$$
\end{frame}
%%%%%%%%%%%%%%%%%%%%%%%%%%%%%%%%%%%%%%%%%%%%%%%%%%%%%%%%%%%%%%%%%%%%%%%%%%%%%%%%%%%%%%%%%%%%%%%



%%%%%%%%%%%%%%%%%%%%%%%%%%%%%%%%%%%%%%%%%%%%%%%%%%%%%%%%%%%%%%%%%%%%%%%%%%%%%%%%%%%%%%%%%%%%%%
\begin{frame}\frametitle{频域卷积定理}%[allowframebreaks][shrink]
\begin{theorem}
频域卷积定理:时域两信号相乘,转换到频域服从卷积关系,即,设
$$ y(n)=h(n)\cdot x(n)  \quad\quad\quad\quad\quad\quad\mbox{}$$
则:
$$
Y(e^{j\omega})= \frac{1}{2\pi}\int_{-\pi}^{\pi}H(e^{j\theta})X(e^{j(\omega-\theta)})d\theta=\frac{1}{2\pi}H(e^{j\omega})\ast X(e^{j\omega})
$$


\end{theorem}
\end{frame}
%%%%%%%%%%%%%%%%%%%%%%%%%%%%%%%%%%%%%%%%%%%%%%%%%%%%%%%%%%%%%%%%%%%%%%%%%%%%%%%%%%%%%%%%%%%%%%%



%%%%%%%%%%%%%%%%%%%%%%%%%%%%%%%%%%%%%%%%%%%%%%%%%%%%%%%%%%%%%%%%%%%%%%%%%%%%%%%%%%%%%%%%%%%%%%
\begin{frame}\frametitle{}%[allowframebreaks][shrink]
\textbf{证明:}
\begin{equation*}
\begin{split}
Y(e^{j\omega})  &= \sum_{n=-\infty}^{\infty}x(n)h(n)e^{-j\omega n} \\
&= \sum_{n=-\infty}^{\infty}x(n)\left[\frac{1}{2\pi}
\int_{-\pi}^{\pi}H(e^{j\theta})e^{j\theta n}d\theta\right]e^{-j\omega n}  \\
&=  \frac{1}{2\pi}\int_{-\pi}^{\pi}H(e^{j\theta})
\left[\sum_{n=-\infty}^{\infty}x(n)e^{-j(\omega-\theta) n}\right]d\theta\\
&= \frac{1}{2\pi}\int_{-\pi}^{\pi}H(e^{j\theta})X(e^{j(\omega-\theta)})d\theta \\
&= \frac{1}{2\pi}H(e^{j\theta})*X(e^{j\omega})
\end{split}
\end{equation*}
\end{frame}
%%%%%%%%%%%%%%%%%%%%%%%%%%%%%%%%%%%%%%%%%%%%%%%%%%%%%%%%%%%%%%%%%%%%%%%%%%%%%%%%%%%%%%%%%%%%%%%

\subsection*{帕斯维尔(Parseval)定理}

%%%%%%%%%%%%%%%%%%%%%%%%%%%%%%%%%%%%%%%%%%%%%%%%%%%%%%%%%%%%%%%%%%%%%%%%%%%%%%%%%%%%%%%%%%%%%%
\begin{frame}\frametitle{帕斯维尔(Parseval) 定理—非周期序列的表达方式}%[allowframebreaks][shrink]
%非周期序列的表达方式
\begin{enumerate}
\item 设$X(e^{j\omega})=FT[x(n)]$,则有
\begin{equation*}
\sum_{n=-\infty}^{\infty}|x(n)|^{2} = \frac{1}{2\pi}\int_{-\pi}^{\pi}|X(e^{j\omega})|^{2}d\omega
\end{equation*}
\item 物理意义:\textbf{信号在时域的总能量等于其在频域的总能量。}

\end{enumerate}
\end{frame}
%%%%%%%%%%%%%%%%%%%%%%%%%%%%%%%%%%%%%%%%%%%%%%%%%%%%%%%%%%%%%%%%%%%%%%%%%%%%%%%%%%%%%%%%%%%%%%%



%%%%%%%%%%%%%%%%%%%%%%%%%%%%%%%%%%%%%%%%%%%%%%%%%%%%%%%%%%%%%%%%%%%%%%%%%%%%%%%%%%%%%%%%%%%%%%
\begin{frame}\frametitle{}%[allowframebreaks][shrink]
\textbf{ 证明:}
\begin{equation*}
\begin{split}
\sum_{n=-\infty}^{\infty}|x(n)|^{2}
&= \sum_{n=-\infty}^{\infty}x^{*}(n)x(n)\\
&= \sum_{n=-\infty}^{\infty}x^{*}(n)\left[\frac{1}{2\pi}\int_{-\pi}^{\pi}X(e^{j\omega})e^{j\omega n}d\omega\right]\\
&= \frac{1}{2\pi}\int_{-\pi}^{\pi}X(e^{j\omega})\sum_{n=-\infty}^{\infty}x^{*}(n)e^{j\omega n}d\omega \\
&= \frac{1}{2\pi}\int_{-\pi}^{\pi}X(e^{j\omega})\left[\sum_{n=-\infty}^{\infty}x(n)e^{-j\omega n}\right]^{*}d\omega \\
&= \frac{1}{2\pi}\int_{-\pi}^{\pi}X(e^{j\omega})X^{*}(e^{j\omega})d\omega \\
&= \frac{1}{2\pi}\int_{-\pi}^{\pi}|X(e^{j\omega})|^{2}d\omega
\end{split}
\end{equation*}
\end{frame}
%%%%%%%%%%%%%%%%%%%%%%%%%%%%%%%%%%%%%%%%%%%%%%%%%%%%%%%%%%%%%%%%%%%%%%%%%%%%%%%%%%%%%%%%%%%%%%%



%%%%%%%%%%%%%%%%%%%%%%%%%%%%%%%%%%%%%%%%%%%%%%%%%%%%%%%%%%%%%%%%%%%%%%%%%%%%%%%%%%%%%%%%%%%%%%
\begin{frame}\frametitle{帕斯维尔(Parseval) 定理—周期序列的表达方式}%[allowframebreaks][shrink]

\begin{enumerate}
\item 对于周期序列$\tilde{x}(n)$而言,则有
\begin{equation*}
\sum_{n=0}^{N-1}|\tilde{x}(n)|^{2} = \frac{1}{N}\sum_{k=0}^{N-1}|\tilde{X}(k)|^{2}
\end{equation*}
\item 物理意义:\textbf{周期序列在时域中一个周期的总能量等于其频域中一个周期的总能量。}
\end{enumerate}
\end{frame}
%%%%%%%%%%%%%%%%%%%%%%%%%%%%%%%%%%%%%%%%%%%%%%%%%%%%%%%%%%%%%%%%%%%%%%%%%%%%%%%%%%%%%%%%%%%%%%%



%%%%%%%%%%%%%%%%%%%%%%%%%%%%%%%%%%%%%%%%%%%%%%%%%%%%%%%%%%%%%%%%%%%%%%%%%%%%%%%%%%%%%%%%%%%%%%
\begin{frame}\frametitle{}%[allowframebreaks][shrink]
\textbf{证明:}
%    首先,写出DFS公式
%    \begin{equation*}
%       \left\{ \begin{aligned}
%           \tilde{X}(k) &= \sum_{n=0}^{N-1}\tilde{x}(n)e^{-j\frac{2\pi}{N}kn}\\
%           \tilde{x}(n) &= \frac{1}{N}\sum_{k=0}^{N-1}\tilde{X}(k)e^{j\frac{2\pi}{N}kn}
%       \end{aligned} \right.
%     \end{equation*}
\begin{equation*}
\begin{split}
\sum_{n=0}^{N-1}|\tilde{x}(n)|^{2}
&= \sum_{n=0}^{N-1}\tilde{x}(n)\tilde{x}^{*}(n)\\
&= \sum_{n=0}^{N-1}\tilde{x}(n)
\left[\frac{1}{N}\sum_{k=0}^{N-1}\tilde{X}(k)e^{j\frac{2\pi}{N}kn}\right]^{*}\\
&= \sum_{n=0}^{N-1}\tilde{x}(n)
\left[\frac{1}{N}\sum_{k=0}^{N-1}\tilde{X}^{*}(k)e^{-j\frac{2\pi}{N}kn}\right]\\
&= \frac{1}{N}\sum_{k=0}^{N-1}\tilde{X}^{*}(k)
\left[\sum_{n=0}^{N-1}\tilde{x}(n)e^{-j\frac{2\pi}{N}kn}\right]\\
&= \frac{1}{N}\sum_{k=0}^{N-1}\tilde{X}^{*}(k)\tilde{X}(k)
= \frac{1}{N}\sum_{k=0}^{N-1}|\tilde{X}(k)|^{2}
\end{split}
\end{equation*}
\end{frame}
%%%%%%%%%%%%%%%%%%%%%%%%%%%%%%%%%%%%%%%%%%%%%%%%%%%%%%%%%%%%%%%%%%%%%%%%%%%%%%%%%%%%%%%%%%%%%%%






%%%%%%%%%%%%%%%%%%%%%%%%%%%%%%%%%%%%%%%%%%%%%%%%%%%%%%%%%%%%%%%%%%%%%%%%%%%%%%%%%%%%%%%%%%%%%%
\begin{frame}\frametitle{作业}%[allowframebreaks][shrink]
作业:
\par 1,2,3,5,6,7,10,11,12,13,14,16, 17,18,24,27,28,29
\end{frame}
%%%%%%%%%%%%%%%%%%%%%%%%%%%%%%%%%%%%%%%%%%%%%%%%%%%%%%%%%%%%%%%%%%%%%%%%%%%%%%%%%%%%%%%%%%%%%%%







\section{例题}
%
%
%
%
%
%
%
%
%
%%%%%%%%%%%%%%%%%%%%%%%%%%%%%%%%%%%%%%%%%%%%%%%%%%%%%%%%%%%%%%%%%%%%%%%%%%%%%%%%%%%%%%%%%%%%%%%
%\begin{frame}[shrink]\frametitle{例题}%[allowframebreaks][shrink]
%
%\begin{example}
%设$x(n)= R_{N}(n)$,求$x(n)$的FT。
%\end{example}
%\par\textbf{解}:
%    \begin{equation*}
%        \begin{split}
%        X(e^{j\omega})
%            &= \sum_{n=-\infty}^{\infty}R_{N}(n)e^{-j\omega n}  = \sum_{n=0}^{N-1}e^{-j\omega n}
%                       = \frac{1-e^{-j\omega N}}{1-e^{-j\omega}}\\
%                       &=\frac{e^{-j\omega N/2}(e^{j\omega N/2}-e^{-j\omega N/2})}{e^{-j\omega/2}(e^{j\omega/2}-e^{-j\omega/2})}\\
%                       &=e^{-j(N-1)\omega/2}\frac{sin(\omega N/2)}{sin(\omega/2)}
%        \end{split}
%    \end{equation*}
%    $$\mbox{一般有}\quad\quad\quad  X(e^{j\omega}) = |X(e^{j\omega})|\cdot e^{j\varphi(\omega)}$$
%    $$ |X(e^{j\omega})| = \left|\frac{sin(\omega N/2)}{sin(\omega/2)}\right| \quad\quad\quad\quad
%         \varphi(\omega) = -\frac{N-1}{2}\omega + \varphi_2(\omega)$$
%    \begin{equation}
%    \mbox{其中:}\quad\quad\quad \varphi_2(\omega)
%            = \left\{\begin{array}
%            {r@{,\quad}l}
%            0    &  \frac{sin(\omega N/2)}{sin(\omega/2)} >0\\
%            \pi  &  \frac{sin(\omega N/2)}{sin(\omega/2)} <0
%            \end{array} \right.
%    \end{equation}
%
%\end{frame}
%%%%%%%%%%%%%%%%%%%%%%%%%%%%%%%%%%%%%%%%%%%%%%%%%%%%%%%%%%%%%%%%%%%%%%%%%%%%%%%%%%%%%%%%%%%%%%%%
%
%
%
%%%%%%%%%%%%%%%%%%%%%%%%%%%%%%%%%%%%%%%%%%%%%%%%%%%%%%%%%%%%%%%%%%%%%%%%%%%%%%%%%%%%%%%%%%%%%%%
\begin{frame}[shrink]\frametitle{}%[allowframebreaks][shrink]
\begin{example}
求$u(n)$的傅里叶变化$U(e^{j\omega}) )$.
\end{example}
\par\textbf{解}:
$$\mbox{令: \quad\quad}  x(n) = u(n) -\frac{1}{2},\quad\quad\quad
\mbox{则: \quad\quad} x(n-1) = u(n-1) - \frac{1}{2}$$
$$x(n)-x(n-1) = u(n) - u(n-1) = \delta(n)\quad\quad\quad\quad (\mbox{注意}\delta(n) \leftrightarrow 1)$$
$$\mbox{(时移定律)\quad}   X(e^{j\omega}) - e^{-j\omega}X(e^{j\omega}) = 1 \quad  \Longrightarrow\quad   X(e^{j\omega}) = \frac{1}{1-e^{-j\omega}}$$
$$\therefore \quad\quad U(e^{j\omega}) = X(e^{j\omega}) + FT[\frac{1}{2}]$$
$$ FT[e^{j\omega_0 n}] = \sum_{k=-\infty}^{\infty}2\pi\cdot\delta(\omega-\omega_{0}-2\pi k) $$
$$\mbox{当$\omega_0 = 0$时\quad\quad}  FT\big[1 \big] = \sum_{k=-\infty}^{\infty}2\pi\cdot\delta(\omega-2\pi k) $$
$$\therefore \quad\quad U(e^{j\omega}) = X(e^{j\omega}) + \sum_{k=-\infty}^{\infty}\pi\cdot\delta(\omega-2\pi k)$$

\end{frame}
%%%%%%%%%%%%%%%%%%%%%%%%%%%%%%%%%%%%%%%%%%%%%%%%%%%%%%%%%%%%%%%%%%%%%%%%%%%%%%%%%%%%%%%%%%%%%%%%
%
%
%
%%%%%%%%%%%%%%%%%%%%%%%%%%%%%%%%%%%%%%%%%%%%%%%%%%%%%%%%%%%%%%%%%%%%%%%%%%%%%%%%%%%%%%%%%%%%%%%
\begin{frame}[shrink]\frametitle{例题}%[allowframebreaks][shrink]
\begin{example}
设$\tilde{x}(n)= cos(\omega_0 n);\omega_0 =\pi/2$,求$\tilde{x}(n)$的FT。
\end{example}
\par\textbf{解}:
$$\tilde{x}(n)=cos(\omega_0 n) = \frac{e^{j\omega_0 n} + e^{-j\omega_0 n}}{2}\quad\quad\quad\quad\quad\quad\quad\quad$$
\begin{equation*}
\begin{split}
FT[cos(\omega_0 n)] &= \frac{1}{2}FT[e^{j\omega_0 n}] + \frac{1}{2}FT[e^{-j\omega_0 n}]\\
&= \pi\cdot \sum_{r=-\infty}^{\infty}\delta(\omega-\omega_0 -2\pi r)
+ \pi\cdot \sum_{r=-\infty}^{\infty}\delta(\omega+\omega_0 -2\pi r)
\end{split}
\end{equation*}
%\newline\newline\newline\newline

\end{frame}
%%%%%%%%%%%%%%%%%%%%%%%%%%%%%%%%%%%%%%%%%%%%%%%%%%%%%%%%%%%%%%%%%%%%%%%%%%%%%%%%%%%%%%%%%%%%%%%%
%
%
%
%%%%%%%%%%%%%%%%%%%%%%%%%%%%%%%%%%%%%%%%%%%%%%%%%%%%%%%%%%%%%%%%%%%%%%%%%%%%%%%%%%%%%%%%%%%%%%%
\begin{frame}[shrink]\frametitle{例题}%[allowframebreaks][shrink]
\begin{example}
设$x(n) = a^{n}u(n),\quad  |a|<1$,求$FT[x(n)]$.
\par\textbf{解}:
\begin{equation}
\begin{split}
X(e^{j\omega}) &= \sum_{n=-\infty}^{\infty}a^{n}u(n) e^{-j\omega n} = \sum_{n=0}^{\infty}a^{n} e^{-j\omega n}
= \sum_{n=0}^{\infty}(ae^{-j\omega})^{n} \\
&= \frac{1}{1- ae^{-j\omega} }    \quad\quad  |a|<1
\end{split}
\end{equation}
\end{example}
\end{frame}
%%%%%%%%%%%%%%%%%%%%%%%%%%%%%%%%%%%%%%%%%%%%%%%%%%%%%%%%%%%%%%%%%%%%%%%%%%%%%%%%%%%%%%%%%%%%%%%%
%
%
%
%
%
%%%%%%%%%%%%%%%%%%%%%%%%%%%%%%%%%%%%%%%%%%%%%%%%%%%%%%%%%%%%%%%%%%%%%%%%%%%%%%%%%%%%%%%%%%%%%%%
\begin{frame}\frametitle{}%[allowframebreaks][shrink]
\begin{example}
$$\mbox{试证明\quad\quad} FT\big[nx(n)\big] =  j\frac{dX(e^{j\omega})}{d\omega}\qquad\qquad\qquad$$
\par\textbf{证}:
\begin{equation*}
\begin{split}
X(e^{j\omega}) &= \sum_{n=-\infty}^{\infty}x(n) e^{-j\omega n} \\
\frac{X(e^{j\omega})}{d\omega}
&= \sum_{n=-\infty}^{\infty}x(n)\cdot -j\cdot n e^{-j\omega n} \\
j\cdot \frac{X(e^{j\omega})}{d\omega}
&= \sum_{n=-\infty}^{\infty}n\cdot x(n)e^{-j\omega n}  = FT\big[n\cdot x(n)\big]\\
\mbox{即}:\quad\quad FT\big[nx(n)\big] &= j\cdot \frac{X(e^{j\omega})}{d\omega}
\end{split}
\end{equation*}
\end{example}
\end{frame}
%%%%%%%%%%%%%%%%%%%%%%%%%%%%%%%%%%%%%%%%%%%%%%%%%%%%%%%%%%%%%%%%%%%%%%%%%%%%%%%%%%%%%%%%%%%%%%%%
%%%%%%%%%%%%%%%%%%%%%%%%%%%%%%%%%%%%%%%%%%%%%%%%%%%%%%%%%%%%%%%%%%%%%%%%%%%%%%%%%%%%%%%%%%%%%%%%
%
%
%
%
%
%
%
%
%
%
%
%
%
%


\section{2.4 离散时域信号的傅里叶变换与模拟信号傅里叶变换的关系}
%%%%%%%%%%%%%%%%%%%%%%%%%%%%%%%%%%%%%%%%%%%%%%%%%%%%%%%%%%%%%%%%%%%%%%%%%%%%%%%%%%%%%%%%%%%%%%
\begin{frame}\frametitle{离散序列的傅里叶变换与模拟信号傅里叶变换的关系}%[allowframebreaks][shrink]
在滤波器设计的学习中将对这一节内容做详细的讲述。\newline

这里我们将首先给出相关结论。
\begin{equation*}%\label{}
X(e^{j\omega}) = \frac{1}{T}\sum_{k=-\infty}^{\infty}X_{a}(j\frac{\omega-2\pi k}{T})
\end{equation*}
\end{frame}

\begin{frame}\frametitle{离散序列的傅里叶变换与模拟信号傅里叶变换的关系}%[allowframebreaks][shrink]

一般说来,$x(n)$可被认为是从模拟信号$x_a(t)$采样得到,这个过程如下所示:
$$x_a(t) \quad\longrightarrow  \quad \hat{x}_a(t) \quad\longrightarrow  \quad x(n) $$
通俗的讲,前者是一个采样的过程,后者是一个抽象的过程。

\begin{wenti}
那么,$x(n)$的傅里叶变换$X(e^{j\omega})$ 与$x_a(t)$ 的傅里叶变换$X_{a}(j\Omega)$之间,有什么样的联系和区别,
这是一个值得研究的问题。

\end{wenti}



\end{frame}






\begin{frame}[shrink]\frametitle{推导:}%[allowframebreaks][shrink]

\begin{equation*}
\begin{split}
\hat{X}_a(j\Omega)
&=  FT\left[\hat{x}_a(t)\right] =\int_{-\infty}^{\infty}\hat{x}_a(t)e^{-j\Omega t}dt\\
&=  \int_{-\infty}^{\infty}\left[\sum_{n=-\infty}^{\infty}x_a(nT)\delta(t-nT)\right]e^{-j\Omega t}dt\\
&=  \sum_{n=-\infty}^{\infty}x_a(nT)\left[\int_{-\infty}^{\infty}\delta(t-nT)e^{-j\Omega t}dt\right]\\
&=  \sum_{n=-\infty}^{\infty}x_a(nT) \left[e^{-j\Omega nT} \right] \\
&=  \sum_{n=-\infty}^{\infty}x_a(nT)e^{-j(\Omega T) n}
%         =  X(e^{j(\Omega T)n}) \\
%         &= X(e^{j\omega})|_{\omega = \Omega T } \\
\end{split}
\end{equation*}
\end{frame}


\begin{frame}\frametitle{推导}%[allowframebreaks][shrink]
又因为:  $x(n) = x_a(nT)$,且令$\omega = \Omega T$,所以有:
\begin{equation*}
\begin{split}
\hat{X}_a(j\Omega)  %&=  FT\left[\hat{x}_a(t)\right] =\int_{-\infty}^{\infty}\hat{x}_a(t)e^{-j\Omega t}dt\\
&=  \sum_{n=-\infty}^{\infty}x_a(nT)e^{-j(\Omega T) n} \qquad\qquad\qquad\qquad \\
&=  \sum_{n=-\infty}^{\infty}x(n)e^{-j(\Omega T) n} \\
&= X(e^{j\omega})|_{\omega = \Omega T } \\
%&=  X(e^{j\cdot(\Omega T)}) \\
\end{split}
\end{equation*}

\begin{jielun}
$$\hat{X}_a(j\Omega)  =
%X(e^{j(\Omega T)n})  =
X(e^{j\omega})|_{\omega = \Omega T } $$
\end{jielun}

\end{frame}




\begin{frame}[shrink]\frametitle{离散序列的傅里叶变换与模拟信号傅里叶变换的关系}%[allowframebreaks][shrink]
又根据采样定理有:
$$\hat{X}_a(j\Omega)   =  \frac{1}{T}\sum_{k=-\infty}^{\infty}X_a(j\Omega -k \Omega_s)\qquad (\Omega_s=\frac{2\pi}{T})
\qquad\quad  $$
所以有:

\begin{jielun}
\begin{equation*}
\begin{split}
X(e^{j\omega})\big|_{\omega = \Omega T }  =\hat{X}_a(j\Omega)  %= X(e^{j\cdot\Omega T})
&= \frac{1}{T}\sum_{k=-\infty}^{\infty}X_a(j\Omega -k \Omega_s) \\
%&= \frac{1}{T}\sum_{k=-\infty}^{\infty}X_a\left(j\frac{T\Omega -k T\Omega_s}{T}\right) \\
%&= \frac{1}{T}\sum_{k=-\infty}^{\infty}X_a\left(j\frac{\omega - 2\pi k}{T}\right)\\
\end{split}
\end{equation*}

%$$X(e^{j\omega})\big|_{\omega = \Omega T } %= X(e^{j\cdot\Omega T})
%= \frac{1}{T}\sum_{k=-\infty}^{\infty}X_a(j\frac{T\Omega -k T\Omega_s}{T})
%= \frac{1}{T}\sum_{k=-\infty}^{\infty}X_a(j\Omega -k \Omega_s)$$
\end{jielun}



\end{frame}





%%%%%%%%%%%%%%%%%%%%%%%%%%%%%%%%%%%%%%%%%%%%%%%%%%%%%%%%%%%%%%%%%%%%%%%%%%%%%%%%%%%%%%%%%%%%%%%
\section{2.5 序列的Z变换}
%%%%%%%%%%%%%%%%%%%%%%%%%%%%%%%%%%%%%%%%%%%%%%%%%%%%%%%%%%%%%%%%%%%%%%%%%%%%%%%%%%%%%%%%%%%%%%
\begin{frame}\frametitle{序列的Z变换}%[allowframebreaks][shrink]

\begin{enumerate}
\item 在模拟信号和系统中,傅里叶变换用于频域分析,拉氏变换用于复频域分析,其实质为傅里叶变换的推广。
\item 类似,时域离散信号与系统中,Z变换为序列傅里叶变换的推广,用于对序列进行复频域分析。
\end{enumerate}



Z变换在数字信号处理中起着很重要的作用,本节讨论其定义、收敛域、逆$Z$变换,性质等
四个主要问题。

\end{frame}
%%%%%%%%%%%%%%%%%%%%%%%%%%%%%%%%%%%%%%%%%%%%%%%%%%%%%%%%%%%%%%%%%%%%%%%%%%%%%%%%%%%%%%%%%%%%%%%


\subsection*{Z变换的定义及基本概念}

%%%%%%%%%%%%%%%%%%%%%%%%%%%%%%%%%%%%%%%%%%%%%%%%%%%%%%%%%%%%%%%%%%%%%%%%%%%%%%%%%%%%%%%%%%%%%%
\begin{frame}\frametitle{Z变换的定义}%[allowframebreaks][shrink]
\begin{definition}
序列$x(n)$的$Z$变换定义为:
\begin{equation*} %\label{fol:sft}
X(z) = \sum_{n=-\infty}^{\infty}x(n)z^{-n} \quad\quad\mbox{双边Z变换}
\end{equation*}
\begin{equation*} %\label{fol:sft}
X(z) = \sum_{n=0}^{\infty}x(n)z^{-n}    \quad\quad\quad\mbox{单边Z变换}
\end{equation*}
式中$z$是一个复变量,它所在的平面称为$z$平面。对于因果序列,两者都一样。本书均使用双边$Z$变换定义。
\end{definition}
\end{frame}
%%%%%%%%%%%%%%%%%%%%%%%%%%%%%%%%%%%%%%%%%%%%%%%%%%%%%%%%%%%%%%%%%%%%%%%%%%%%%%%%%%%%%%%%%%%%%%%



%%%%%%%%%%%%%%%%%%%%%%%%%%%%%%%%%%%%%%%%%%%%%%%%%%%%%%%%%%%%%%%%%%%%%%%%%%%%%%%%%%%%%%%%%%%%%%
\begin{frame}\frametitle{$Z$变换存在的充要条件}%[allowframebreaks][shrink]
%$Z$变换收敛域的概念
\begin{enumerate}
\item [(1)]
$Z$变换实际上为一个罗朗级数,其存在条件为该级数绝对收敛,也就是满足:
$$\sum_{n=-\infty}^{\infty}|x(n)z^{-n}|  <\infty $$
\begin{itemize}
\item 满足罗朗级数收敛的$z$值取值范围称为$X(z)$的收敛域
\end{itemize}
%满足该条件的$z$值取值范围称为$X(z)$的收敛域。
\end{enumerate}
\end{frame}
%%%%%%%%%%%%%%%%%%%%%%%%%%%%%%%%%%%%%%%%%%%%%%%%%%%%%%%%%%%%%%%%%%%%%%%%%%%%%%%%%%%%%%%%%%%%%%%

%%%%%%%%%%%%%%%%%%%%%%%%%%%%%%%%%%%%%%%%%%%%%%%%%%%%%%%%%%%%%%%%%%%%%%%%%%%%%%%%%%%%%%%%%%%%%%
\begin{frame}[shrink]\frametitle{$Z$变换收敛域的概念}%[allowframebreaks][shrink]
%$Z$变换收敛域的概念
\begin{enumerate}
% \item [(1)] 满足罗朗级数收敛的$z$值取值范围称为$X(z)$的收敛域 \newline
\item [(2)]一般$X(z)$的收敛域为环状域,即:$R_{x-}<|z|<R_{x+}$,也就是说:
\begin{itemize}
\item 收敛域是以$R_{x-}$和$R_{x+}$为收敛半径的两个圆形成的环状域。 \newline
\item $R_{x-}$ 可以小到$0$,$R_{x+}$可以大到无穷大。
\end{itemize}

\end{enumerate}
%此处画出收敛域形状图:
%\newpage\newpage\newpage\newpage\newpage\newpage

\end{frame}
%%%%%%%%%%%%%%%%%%%%%%%%%%%%%%%%%%%%%%%%%%%%%%%%%%%%%%%%%%%%%%%%%%%%%%%%%%%%%%%%%%%%%%%%%%%%%%%

%%%%%%%%%%%%%%%%%%%%%%%%%%%%%%%%%%%%%%%%%%%%%%%%%%%%%%%%%%%%%%%%%%%%%%%%%%%%%%%%%%%%%%%%%%%%%%
\begin{frame}\frametitle{$Z$变换的零极点的概念}%[allowframebreaks][shrink]

常用的$Z$变换是一个有理函数,可用两个多项式之比表示。
$$X(z) = \frac{P(z)}{Q(z)}$$
$P(z)=0$的根,称为$X(z)$的零点。\par

$Q(z)=0$的根,称为$X(z)$的极点。\par
\begin{zhuyi}
$X(z)$的性质主要取决于极点,在极点处$Z$变换不存在。
\end{zhuyi}
\end{frame}
%%%%%%%%%%%%%%%%%%%%%%%%%%%%%%%%%%%%%%%%%%%%%%%%%%%%%%%%%%%%%%%%%%%%%%%%%%%%%%%%%%%%%%%%%%%%%%%



%%%%%%%%%%%%%%%%%%%%%%%%%%%%%%%%%%%%%%%%%%%%%%%%%%%%%%%%%%%%%%%%%%%%%%%%%%%%%%%%%%%%%%%%%%%%%%
\begin{frame}\frametitle{序列$Z$变换与傅里叶变换的关系}%[allowframebreaks][shrink]
对比Z变换和傅里叶变换的公式:
\begin{equation*}
\begin{split}
X(e^{j\omega}) &= \sum_{n=-\infty}^{\infty}x(n)e^{-j\omega n}\\
X(z) \:\:     &= \sum_{n=-\infty}^{\infty}x(n)z^{-n}
\end{split}
\end{equation*}
对比可得:$z = e^{j\omega}$,也就是说:
%$$\mbox{\quad 即:\quad}X(e^{j\omega}) = X(z)|_{z=e^{j\omega}}$$
$$X(e^{j\omega}) = X(z)|_{z=e^{j\omega}}$$
即,单位圆上的$Z$变换就是序列的傅里叶变换。\par
\begin{shuoming}
\begin{enumerate}
\item $Z$变换的收敛域必须包括$Z$平面的单位圆。
\item 傅里叶变换是$Z$变换的特例。
\end{enumerate}
\end{shuoming}
\quad\newline

\end{frame}
%%%%%%%%%%%%%%%%%%%%%%%%%%%%%%%%%%%%%%%%%%%%%%%%%%%%%%%%%%%%%%%%%%%%%%%%%%%%%%%%%%%%%%%%%%%%%%%



%%%%%%%%%%%%%%%%%%%%%%%%%%%%%%%%%%%%%%%%%%%%%%%%%%%%%%%%%%%%%%%%%%%%%%%%%%%%%%%%%%%%%%%%%%%%%%
\begin{frame}\frametitle{}%[allowframebreaks][shrink]
\begin{example}
求$x(n)=u(n)$的$Z$变换。\par
\end{example}
\begin{answer}
%s\textbf{解:}
$$X(z) = ZT[x(n)] = \sum_{n=-\infty}^{\infty}x(n)z^{- n} = \sum_{n=0}^{\infty}(z^{-1})^{ n} = \frac{1}{1-z^{-1}}$$
\begin{enumerate}
\item [1] 显然,$X(z)$的极点为$z=1$
\item [2]$X(z)$存在的条件是: \quad $|z^{-1}|<1$\quad 即$|z|>1$
\item [3]$Z$变换收敛域不包含单位圆,其傅里叶变换也不存在
\item [4]引入冲激函数后,可得到其傅里叶变换。
\end{enumerate}

\end{answer}

\end{frame}
%%%%%%%%%%%%%%%%%%%%%%%%%%%%%%%%%%%%%%%%%%%%%%%%%%%%%%%%%%%%%%%%%%%%%%%%%%%%%%%%%%%%%%%%%%%%%%%

%
\subsection*{Z变换收敛域的讨论}
%%%%%%%%%%%%%%%%%%%%%%%%%%%%%%%%%%%%%%%%%%%%%%%%%%%%%%%%%%%%%%%%%%%%%%%%%%%%%%%%%%%%%%%%%%%%%%
\begin{frame}\frametitle{Z变换收敛域的讨论}%[allowframebreaks][shrink]
%在$x(n)\leftrightarrow X(z)$时,
使得$Z$变换存在,也就是使得洛朗级数绝对可和的$z$变换取值范围,
为$Z$变换的收敛域。
$$X(z)= \sum_{n=-\infty}^{\infty}x(n)z^{-n} = \cdots + x(-1)z+x(0)+x(1)z^{-1}+\cdots $$
%注意: \textbf{收敛域取决于$x(n)$的性质。}\par
%$x(n)$可分为以下四种类型:
\begin{enumerate}
\item 有限长序列
\item 右序列
\item 左序列
\item 双边序列
\end{enumerate}
\end{frame}
%%%%%%%%%%%%%%%%%%%%%%%%%%%%%%%%%%%%%%%%%%%%%%%%%%%%%%%%%%%%%%%%%%%%%%%%%%%%%%%%%%%%%%%%%%%%%%%



%%%%%%%%%%%%%%%%%%%%%%%%%%%%%%%%%%%%%%%%%%%%%%%%%%%%%%%%%%%%%%%%%%%%%%%%%%%%%%%%%%%%%%%%%%%%%%
\begin{frame}[shrink]\frametitle{有限长序列}%[allowframebreaks][shrink]
\begin{equation*}
x(n) = \left\{
\begin{array}
{r@{,\quad}l}
x(n)    & \ n_1 \leqslant n \leqslant n_2    \mbox{(不全为0)}\\
0\quad  & \mbox{其他 \quad\quad (全为0)}
\end{array} \right.
\end{equation*}
\textbf{有限项级数求和必定收敛}。仅需考虑0 点,及$\infty$两点情况。\par 可分四种情况讨论
\begin{enumerate}
\item[$1^0$] $n_2 > n_1\geqslant 0$,\quad\quad 仅存在负幂级数,$0<|z|\leqslant \infty$ (因果序列)
\item[$2^0$] $n_1 < n_2\leqslant 0$,\quad\quad 仅存在正幂级数,$0\leqslant|z|< \infty$
\item[$3^0$] $n_1<0,n_2 >0$,        \quad 存在正负幂级数,$0<|z|<\infty$
\item[$4^0$] $n_1 = 0, n_2 =0$,     \quad 整个$Z$ 平面。
\end{enumerate}
\end{frame}
%%%%%%%%%%%%%%%%%%%%%%%%%%%%%%%%%%%%%%%%%%%%%%%%%%%%%%%%%%%%%%%%%%%%%%%%%%%%%%%%%%%%%%%%%%%%%%




%%%%%%%%%%%%%%%%%%%%%%%%%%%%%%%%%%%%%%%%%%%%%%%%%%%%%%%%%%%%%%%%%%%%%%%%%%%%%%%%%%%%%%%%%%%%%%
\begin{frame}[shrink]\frametitle{有限长序列}%[allowframebreaks][shrink]
\begin{example}
求$x(n)=R_{N}(n)$的$Z$变换.\end{example}
\begin{answer}
%\textbf{解:}
$$X(z)= \sum_{n=-\infty}^{\infty}R_{N}(n)z^{-n} = \sum_{n=0}^{N-1}(z^{-1})^{n} = \frac{1-z^{-N}}{1-z^{-1}}$$
$X(z)$为有限长级数求和,必定收敛,只需考虑$z=0,z=\infty$两点。\par
显然有: \quad\quad\quad $0<|z|\leqslant \infty$。
\end{answer}
\end{frame}
%%%%%%%%%%%%%%%%%%%%%%%%%%%%%%%%%%%%%%%%%%%%%%%%%%%%%%%%%%%%%%%%%%%%%%%%%%%%%%%%%%%%%%%%%%%%%%


%%%%%%%%%%%%%%%%%%%%%%%%%%%%%%%%%%%%%%%%%%%%%%%%%%%%%%%%%%%%%%%%%%%%%%%%%%%%%%%%%%%%%%%%%%%%%%
\begin{frame}[shrink]\frametitle{右序列}%[shrink]

\begin{equation*}
x(n) = \left\{
\begin{array}
{r@{,\quad}l}
x(n)    & n \geqslant n_1 \\
0\quad  & n \leqslant n_1
\end{array} \right.
\end{equation*}
%$$X(z)= \sum_{n=n_1}^{\infty}x(n)z^{-n} = \sum_{n=n_1}^{-1}x(n)z^{-n} + \sum_{n=0}^{\infty}x(n)z^{-n}$$
%可以看出,左半为有限长序列,右半为因果序列。
右边序列收敛域为Z平面上以原点为圆心的某个园外。\par
可分两种情况讨论:
\begin{enumerate}
\item $n_1<0$时,存在正幂级数,$|z|$不能为$\infty$,\par 此时有: $R_{x-} <|z| <\infty$
\item $n_1\geqslant0$时,为因果序列。有: \quad $R_{x-} <|z| \leqslant\infty$
\end{enumerate}
\begin{zhuyi}
显然,当$x(n)$为因果序列时,其收敛域包含无穷远点。
\end{zhuyi}
%newline\newline\newline
\end{frame}
%%%%%%%%%%%%%%%%%%%%%%%%%%%%%%%%%%%%%%%%%%%%%%%%%%%%%%%%%%%%%%%%%%%%%%%%%%%%%%%%%%%%%%%%%%%%%%


%%%%%%%%%%%%%%%%%%%%%%%%%%%%%%%%%%%%%%%%%%%%%%%%%%%%%%%%%%%%%%%%%%%%%%%%%%%%%%%%%%%%%%%%%%%%%%
\begin{frame}[shrink]\frametitle{左序列}%[allowframebreaks][shrink]
\begin{equation*}
x(n) = \left\{
\begin{array}
{r@{,\quad}l}
x(n)    & n \leqslant n_2 \\
0\quad  & n > n_2
\end{array} \right.
\end{equation*}
%$$X(z)= \sum_{n=-\infty}^{n_2}x(n)z^{-n} $$
左边序列收敛域为Z平面上以原点为圆心的某个园内。\par
可分两种情况讨论:
\begin{enumerate}
\item $n_2>0$时,包含负项级数,收敛区不包括$0$ 点,此时有: \quad $0 <|z| < R_{x+}$
\item $n_2\leqslant0$时,仅存在负项级数,此时有: \quad $0 \leqslant|z| <R_{x+}$
\end{enumerate}
%如下图: \newline\newline\newline\newline\newline\newline
\end{frame}


\begin{frame}[shrink]\frametitle{双边序列}%[allowframebreaks][shrink]

$$X(z)= \sum_{n=-\infty}^{\infty}x(n)z^{-n} = \sum_{n=-\infty}^{-1}x(n)z^{-n} + \sum_{n=0}^{\infty}x(n)z^{-n}$$
\begin{enumerate}
\item 可见双边序列由因果序列和反因果序列组成,两者的收敛域分别为:$R_{x-} <|z|$ 和$|z|<R_{x+}$。
\item 双边序列取两者的交集,收敛域为: \quad$R_{x-} <|z|<R_{x+}\quad $,其为一环状区域。
\item 如收敛域无交集,则ZT不存在。
\end{enumerate}

\par
\end{frame}
%%%%%%%%%%%%%%%%%%%%%%%%%%%%%%%%%%%%%%%%%%%%%%%%%%%%%%%%%%%%%%%%%%%%%%%%%%%%%%%%%%%%%%%%%%%%%%%
%
%
%
%%%%%%%%%%%%%%%%%%%%%%%%%%%%%%%%%%%%%%%%%%%%%%%%%%%%%%%%%%%%%%%%%%%%%%%%%%%%%%%%%%%%%%%%%%%%%%%
\begin{frame}[shrink]\frametitle{}%[allowframebreaks][shrink]
\begin{example}      \textbf{求$x(n)=a^{n}u(n)$的$Z$ 变换及收敛域.(因果序列)}\end{example}
\begin{answer}
\begin{equation*}
\begin{split}
X(z) &= \sum_{n=-\infty}^{\infty}x(n)z^{-n} = \sum_{n=0}^{\infty}\left(\frac{a}{z}\right)^{n}
= \frac{1}{1-\frac{a}{z}} = \frac{z}{z-a}  %\mbox{\quad\quad 收敛域为:}  |z|>|a|
\end{split}
\end{equation*}
$\mbox{\quad\quad 收敛域为:} \quad\quad |z|>|a|$
\end{answer}
\end{frame}
%%%%%%%%%%%%%%%%%%%%%%%%%%%%%%%%%%%%%%%%%%%%%%%%%%%%%%%%%%%%%%%%%%%%%%%%%%%%%%%%%%%%%%%%%%%%%%%


%%%%%%%%%%%%%%%%%%%%%%%%%%%%%%%%%%%%%%%%%%%%%%%%%%%%%%%%%%%%%%%%%%%%%%%%%%%%%%%%%%%%%%%%%%%%%%
\begin{frame}[shrink]\frametitle{}%[allowframebreaks][shrink]
\begin{example}
\textbf{求$x(n)=-a^{n}u(-n-1)$的$Z$ 变换及收敛域.(反因果序列)}\end{example}
\begin{answer}
\begin{equation*}
\begin{split}
X(z)         &= \sum_{n=-\infty}^{\infty}x(n)z^{-n} = -\sum_{n=-\infty}^{-1}a^{n}z^{-n} \\
\mbox{注意}:&\quad  \quad\quad\sum_{n=-\infty}^{-1}x(n)z^{-n} = \sum_{n=1}^{\infty}x(-n)z^{n}\\
X(z)         &= -\sum_{n=1}^{\infty}a^{-n}z^{n} = -\frac{\frac{z}{a}}{1-\frac{z}{a}}= \frac{z}{z-a}
\end{split}
\end{equation*}
$\mbox{\quad\quad 收敛域为:}  |z|<|a|$
\end{answer}
\end{frame}
%%%%%%%%%%%%%%%%%%%%%%%%%%%%%%%%%%%%%%%%%%%%%%%%%%%%%%%%%%%%%%%%%%%%%%%%%%%%%%%%%%%%%%%%%%%%%%%


%%%%%%%%%%%%%%%%%%%%%%%%%%%%%%%%%%%%%%%%%%%%%%%%%%%%%%%%%%%%%%%%%%%%%%%%%%%%%%%%%%%%%%%%%%%%%%%
\begin{frame}\frametitle{对于指数序列ZT的小结}%[allowframebreaks][shrink]
\begin{summary}
\begin{equation*}
\begin{split}
a^{n}u(n)     &\leftrightarrow \frac{z}{z-a} \quad\quad |z|>|a| \mbox{\quad\quad 因果序列}\\
-a^{n}u(-n-1) &\leftrightarrow \frac{z}{z-a} \quad\quad |z|<|a| \mbox{\quad\quad 反因果序列}
\end{split}
\end{equation*}
\end{summary}
\begin{explain}
%  \textbf{说明:} \par
不同的$x(n)$,其$Z$变换表达式可能相同,所以$Z$变换$X(z)$与收敛域联系在一起才有意义。即:
$$X(z)\quad + \quad \mbox{收敛域}\quad\Longleftrightarrow x(n)$$
\end{explain}
\end{frame}
%%%%%%%%%%%%%%%%%%%%%%%%%%%%%%%%%%%%%%%%%%%%%%%%%%%%%%%%%%%%%%%%%%%%%%%%%%%%%%%%%%%%%%%%%%%%%%%%



%%%%%%%%%%%%%%%%%%%%%%%%%%%%%%%%%%%%%%%%%%%%%%%%%%%%%%%%%%%%%%%%%%%%%%%%%%%%%%%%%%%%%%%%%%%%%%
\begin{frame}[shrink]\frametitle{}%[allowframebreaks][shrink]
\begin{example}     \textbf{设$x(n)=a^{|n|}$,$a\in R$,求$x(n)$ 的$Z$变换并给出收敛域.}\end{example}
\begin{answer}
\begin{equation*}
\begin{split}
X(z) &= \sum_{n=-\infty}^{\infty}x(n)z^{-n} = \sum_{n=-\infty}^{\infty}a^{|n|}z^{-n} \\
&= \sum_{n=-\infty}^{-1}a^{-n}z^{-n} + \sum_{n=0}^{\infty}a^{n}z^{-n}
= \sum_{n=1}^{\infty}a^{n}z^{n} + \sum_{n=0}^{\infty}a^{n}z^{-n}\\
&= \frac{az}{1-az} + \frac{1}{1-a z^{-1}} = \frac{1-a^{2}}{(1-az)(1-az^{-1})}
\end{split}
\end{equation*}
收敛域需同时满足$|az|<1, |az^{-1}|<1,\mbox{即}:|a|<|z|<|\frac{1}{a}|$,这意味着必须有$|a|<1$,
如果$|a|>1$,则无公共收敛域,$X(z)$ 不存在。
\end{answer}
\end{frame}
%%%%%%%%%%%%%%%%%%%%%%%%%%%%%%%%%%%%%%%%%%%%%%%%%%%%%%%%%%%%%%%%%%%%%%%%%%%%%%%%%%%%%%%%%%%%%%
\subsection*{逆Z变换}


%%%%%%%%%%%%%%%%%%%%%%%%%%%%%%%%%%%%%%%%%%%%%%%%%%%%%%%%%%%%%%%%%%%%%%%%%%%%%%%%%%%%%%%%%%%%%%
\begin{frame}\frametitle{逆Z变换}%[allowframebreaks][shrink]

已知$X(z)$及收敛域,求$x(n)$。

\end{frame}
%%%%%%%%%%%%%%%%%%%%%%%%%%%%%%%%%%%%%%%%%%%%%%%%%%%%%%%%%%%%%%%%%%%%%%%%%%%%%%%%%%%%%%%%%%%%%%%

\subsubsection*{部分分式法}


%%%%%%%%%%%%%%%%%%%%%%%%%%%%%%%%%%%%%%%%%%%%%%%%%%%%%%%%%%%%%%%%%%%%%%%%%%%%%%%%%%%%%%%%%%%%%%
\begin{frame}[shrink]\frametitle{部分分式法}%[allowframebreaks][shrink]
步骤:
\begin{enumerate}
\item  分解因式
$$\frac{X(z)}{z}=\sum_{m=1}^{N}\frac{A_m}{z-z_m} \mbox{\quad\quad 部分分式和}\quad\quad\quad\quad\quad\quad\quad\quad$$
$$\mbox{\quad\quad 其中系数为:} A_m =\left[\frac{X(z)}{z}(z-z_m)\right]_{z=z_m}$$

\item 套公式。
\begin{equation*}
\begin{split}
a^{n}u(n)     &\leftrightarrow \frac{z}{z-a} \quad\quad |z|>|a| \mbox{\quad\quad 因果序列}\\
-a^{n}u(-n-1) &\leftrightarrow \frac{z}{z-a} \quad\quad |z|<|a| \mbox{\quad\quad 反因果序列}
\end{split}
\end{equation*}
\end{enumerate}
\end{frame}
%%%%%%%%%%%%%%%%%%%%%%%%%%%%%%%%%%%%%%%%%%%%%%%%%%%%%%%%%%%%%%%%%%%%%%%%%%%%%%%%%%%%%%%%%%%%%%%



%%%%%%%%%%%%%%%%%%%%%%%%%%%%%%%%%%%%%%%%%%%%%%%%%%%%%%%%%%%%%%%%%%%%%%%%%%%%%%%%%%%%%%%%%%%%%%
\begin{frame}[allowframebreaks]\frametitle{}%[allowframebreaks][shrink]
\begin{example}
设$$X(z)=\frac{5z^{-1}}{1+z^{-1} -6z^{-2}},\quad\quad 2<|z|<3$$
\end{example}
\textbf{解:}\par
收敛域为圆环,显然$x(n)$为双边序列。
\begin{equation*}
\begin{split}
\frac{X(z)}{z} &= \frac{5z^{-2}}{1+z^{-1} -6z^{-2}} =\frac{5}{z^2 + z -6}= \frac{5}{(z+3)(z-2)}\\
&= \frac{A_1}{z-2} + \frac{A_2}{z+3}
\end{split}
\end{equation*}
\begin{equation*}
\begin{split}
A_1   &= \left[\frac{X(z)}{z}(z-2)\right]_{z =2 } =\frac{5}{z+3}|_{z=2} =1\\
A_2   &= \left[\frac{X(z)}{z}(z+3)\right]_{z =-3 } =\frac{5}{z-2}|_{z=-3} =-1
\end{split}
\end{equation*}
$$\frac{X(z)}{z} = \frac{1}{z-2} - \frac{1}{z+3} \quad\quad \Longrightarrow\quad\quad X(z) = \frac{z}{z-2} - \frac{z}{z+3}$$
我们知道:
\begin{equation*}
\left\{ \begin{aligned}
a^{n}u(n)     &\leftrightarrow \frac{z}{z-a} \quad\quad |z|>|a| \mbox{\quad\quad 因果序列}\\
-a^{n}u(-n-1)   &\leftrightarrow \frac{z}{z-a} \quad\quad |z|<|a| \mbox{\quad\quad 反因果序列}
\end{aligned} \right.
\end{equation*}
套公式:
\begin{equation*}
\begin{split}
x(n)    &=  2^{n}u(n) - (-(-3)^{n}u(-n-1))\\
&=  2^{n}u(n) +(-3)^{n}u(-n-1)
\end{split}
\end{equation*}



\end{frame}
%%%%%%%%%%%%%%%%%%%%%%%%%%%%%%%%%%%%%%%%%%%%%%%%%%%%%%%%%%%%%%%%%%%%%%%%%%%%%%%%%%%%%%%%%%%%%%%


\subsubsection*{留数法(围线积分法)}
%%%%%%%%%%%%%%%%%%%%%%%%%%%%%%%%%%%%%%%%%%%%%%%%%%%%%%%%%%%%%%%%%%%%%%%%%%%%%%%%%%%%%%%%%%%%%%
\begin{frame}[allowframebreaks]\frametitle{留数法—逆Z 变换}%[allowframebreaks][shrink]

$$X(z) = \sum_{n=-\infty}^{\infty}x(n)z^{-n}\quad\quad R_{x-} <|z|<R_{x+}$$
则有:
$$x(n) = \frac{1}{2\pi j}\oint_{c} X(z) z^{n-1}dz \quad\quad\quad\quad\quad\quad$$
\par   \textbf{证明:}此处引入柯西积分定理(引入,不做证明)
\begin{equation*}
\frac{1}{2\pi j}\oint_{c} z^{m-1}dz = \left\{
\begin{array}
{r@{,\quad}l}
1    & m   =  0 \\
0    & m \neq 0
\end{array} \right.
\end{equation*}
这里$C$为一个逆时针封闭曲线,那么:
\begin{equation*}
\begin{split}
\frac{1}{2\pi j}\oint_{c} X(z) z^{k-1}dz
&= \frac{1}{2\pi j}\oint_{c} \left[ \sum_{n=-\infty}^{\infty}x(n)z^{-n} z^{k-1}\right]dz \\
&= \frac{1}{2\pi j}\oint_{c} \left[  \sum_{n=-\infty}^{\infty}x(n)z^{-n+k-1}\right]dz \\
&= \sum_{n=-\infty}^{\infty}x(n)\left[\frac{1}{2\pi j}\oint_{c}  z^{-n+k-1}dz\right]   %        = x(k)
\end{split}
\end{equation*}
显然,仅当$k=n$时,有:
$$x(k) = \frac{1}{2\pi j}\oint_{c} X(z) z^{k-1}dz $$

\newpage
$$x(k) = \frac{1}{2\pi j}\oint_{c} X(z) z^{k-1}dz $$

交换符号 $k \rightarrow n$,有:\vspace{-0.3cm}
$$x(n) = \frac{1}{2\pi j}\oint_{c} X(z) z^{n-1}dz $$

显然,给出反变换的公式后,关键在于求解这个围线积分。围线积分的直接求解非常麻烦。\newline


\par 直接计算围线积分非常麻烦,可利用留数定理得到。

\end{frame}
%%%%%%%%%%%%%%%%%%%%%%%%%%%%%%%%%%%%%%%%%%%%%%%%%%%%%%%%%%%%%%%%%%%%%%%%%%%%%%%%%%%%%%%%%%%%%%%



%%%%%%%%%%%%%%%%%%%%%%%%%%%%%%%%%%%%%%%%%%%%%%%%%%%%%%%%%%%%%%%%%%%%%%%%%%%%%%%%%%%%%%%%%%%%%%
\begin{frame}[shrink]\frametitle{留数法——留数定理}%[allowframebreaks][shrink]
%
%
\begin{theorem}
%\item [a] 留数定理

$$\frac{1}{2\pi j}\oint_{c} F(z)dz %= \sum_{k}Res\left[F(z),z_k\right]
=\sum Res\left[F(z)\right]_{\mbox{C 内诸极点}} $$
\end{theorem}

%$$x(n) = \frac{1}{2\pi j}\oint_{c} X(z) z^{n-1}dz $$
%设 $F(z) = X(z)\cdot z^{n-1}$, 则有
%               $$x(n) = \frac{1}{2\pi j}\oint_{c} F(z)dz $$
%               如$F(z)$在围线$C$内的极点为$z_k$,则根据留数定理,有:
%
%               $$ x(n) = \sum_{k}Res\left[F(z),z_k\right]  \quad\quad\quad\quad\quad\quad$$
%               $\mbox{即:\quad\quad}x(n)$是围线$C$ 内所有极点的留数之和。
\end{frame}



\begin{frame}[shrink]\frametitle{利用留数定理求逆Z变换}%[allowframebreaks][shrink]
已知逆Z变换公式为:
$$x(n) = \frac{1}{2\pi j}\oint_{c} X(z) z^{n-1}dz $$
令$F(z) = X(z)\cdot z^{n-1}$, 则有
$$x(n) = \frac{1}{2\pi j}\oint_{c} F(z)dz $$
如$F(z)$在围线$C$内的极点为$z_k$,则根据留数定理,有:

$$ x(n)= \frac{1}{2\pi j}\oint_{c} F(z)dz= \sum_{k}Res\left[F(z),z_k\right]  \quad$$
$\mbox{即:\quad\quad}x(n)$是围线$C$ 内所有极点的留数之和。
\end{frame}



\begin{frame}[shrink]\frametitle{留数法——留数的求法}%[allowframebreaks][shrink]
%\begin{enumerate}
%\item [b] 留数的求法:\par
%\end{enumerate}
\begin{enumerate}
\item 对于单阶极点:$z=z_0$,有
$$Res\left[F(z),z_0\right] = F(z)(z-z_0)\big|_{z=z_0}$$
\item 对于m阶高阶极点:$z=z_0$,有
$$Res\left[X(z)\cdot z^{n-1},z_0\right] =\qquad\qquad\qquad\qquad\qquad\qquad\qquad\qquad\qquad\qquad$$
$$    \frac{1}{(m-1)!}\frac{d^{m-1}}{dz^{m-1}}\left[F(z)(z-z_0)^{m}\right]_{z=z_0}$$
\end{enumerate}

\end{frame}

\begin{frame}[shrink]\frametitle{留数法——留数辅助定理}%[allowframebreaks][shrink]

\textbf{对于高阶极点,一般难于求解,往往利用留数辅助定理。}\par
\begin{theorem}
设被积函数$F(z)=X(z)\cdot z^{n-1}$ 是有理函数,分母多项式的最高阶次大于等于分子多项式最高阶次$2$次以上,则有:
$$\oint_{C\rightarrow \infty}F(z)dz = 0$$
$$\therefore\quad\quad\quad \frac{1}{2\pi j}\oint_{C\rightarrow \infty}F(z)dz = 0$$
即在整个$Z$平面内,该围线积分积分结果为$0$。
\end{theorem}
\end{frame}

\begin{frame}[shrink]\frametitle{留数法——留数辅助定理}%[allowframebreaks][shrink]
如果在$X(z)$收敛域内选取一个围线$C$,显然有:
$$\sum Res\left[X(z)z^{n-1}\right]_{\mbox{C内极点}} +
\sum Res\left[X(z)z^{n-1}\right]_{\mbox{C外极点}} =0$$
在用留数法求$x(n)$时,如果围线$C$内存在高阶极点,可通过求围线外的一阶极点的留数,再取反即可。\par
\end{frame}

\begin{frame}[shrink]\frametitle{留数法——留数辅助定理成立条件}%[allowframebreaks][shrink]
\textbf{成立条件:}
\begin{itemize}
\item 留数辅助定理成立的条件是:$F(z) = X(z)\cdot z^{n-1}$ 的分母多项式的阶次大于等于分子阶次$2$ 次以上。
$$\mbox{设:\quad} X(z) = \frac{P(z)}{Q(z)}\frac{\rightarrow M\mbox{次}}
{\rightarrow N\mbox{次}}  \Longrightarrow  N-(M+n-1)\geqslant 2  $$
$$\mbox{即:\quad} n< N-M \quad\quad \mbox{注意:此即为判断依据}$$
\end{itemize}
\end{frame}
%%%%%%%%%%%%%%%%%%%%%%%%%%%%%%%%%%%%%%%%%%%%%%%%%%%%%%%%%%%%%%%%%%%%%%%%%%%%%%%%%%%%%%%%%%%%%%%
%%%%%%%%%%%%%%%%%%%%%%%%%%%%%%%%%%%%%%%%%%%%%%%%%%%%%%%%%%%%%%%%%%%%%%%%%%%%%%%%%%%%%%%%%%%%%%%
\begin{frame}[allowframebreaks]\frametitle{}%[allowframebreaks][shrink]
\begin{example}
设$$X(z)=\frac{5z^{-1}}{1+z^{-1} -6z^{-2}},\quad 2<|z|<3,\quad\mbox{求}x(n)$$
\end{example}
\textbf{解:}\par
\begin{enumerate}
\item [(1)] $X(z) = \frac{5z}{z^2 + z -6} \quad\quad 2<|z|<3$,\newline
\par 收敛域为圆环,显然$x(n)$为双边序列。
$$F(z) = X(z)z^{n-1} = \frac{5z^n}{z^2 + z -6} = \frac{5z^n}{(z-2)(z+3)}$$
\item [(2)] 确定收敛区域,确定围线$C$;
\newpage
\item [(3)] 对不同的$n$进行讨论
\begin{enumerate}
\item [(a)] $n\geqslant0$,显然极点为 $z =2$ ,$z=-3$,\\
但围线C 内只有极点$z=2$,依留数定理,有
\begin{equation*}
\begin{split}
x(n) &= \sum_{k}Res\left[X(z)z^{n-1},z_0 = 2\right]\\
&= \left[X(z)z^{n-1}(z-z_0)\right]_{z_0 = 2} \\
&= \left[\frac{5\cdot z^{n}}{(z+3)}\right]_{z = 2} = 2^n
\end{split}
\end{equation*}
\item [(b)] 当$n<0$时,$C$内有极点$z=0$,$z=2$,$C$外有极点$z=-3$,且$z=0$处的极点为高阶极点,这时有:
$$ X(z) = \frac{5z}{z^2+z-6}=\frac{P(z)}{Q(z)}\frac{\rightarrow M=1\mbox{次}}
{\rightarrow N=2\mbox{次}}$$
$$\mbox{有:\quad\quad}  n<N-M = 1\quad\mbox{适用于留数辅助定理}$$
\begin{equation*}
\begin{split}
x(n) &=\sum Res\left[X(z)z^{n-1}\right]_{\mbox{C 内诸极点}}    \\
&= -\sum Res\left[X(z)z^{n-1}\right]_{\mbox{C 外诸极点}}  \\
&=\sum Res\left[X(z)z^{n-1}\right]_{z=0,z=2} \\
&=-\sum Res\left[X(z)z^{n-1}\right]_{z=-3} \\
&= -\left[\frac{5\cdot z^{n}}{(z-2)}\right]_{z = -3} = (-3)^n
\end{split}
\end{equation*}
\end{enumerate}
\begin{equation*}
\therefore  x(n) = \left\{
\begin{array}
{r@{,\quad}l}
2^n\quad  & n \geqslant  0 \\
(-3)^{n} & n  < 0
\end{array} \right.
\end{equation*}
$$\therefore\quad\quad x(n) = 2^{n}u(n) +(-3)^{n}u(-n-1)$$
\end{enumerate}

\end{frame}
%%%%%%%%%%%%%%%%%%%%%%%%%%%%%%%%%%%%%%%%%%%%%%%%%%%%%%%%%%%%%%%%%%%%%%%%%%%%%%%%%%%%%%%%%%%%%%%



%%%%%%%%%%%%%%%%%%%%%%%%%%%%%%%%%%%%%%%%%%%%%%%%%%%%%%%%%%%%%%%%%%%%%%%%%%%%%%%%%%%%%%%%%%%%%%
\begin{frame}[allowframebreaks]\frametitle{}%[allowframebreaks][shrink]
\begin{example}已知
$$ X(z)=\frac{1}{1-az^{-1}},\quad |z|>|a|,\mbox{求}x(n)$$\end{example}
\textbf{解:}\par
\begin{enumerate}
\item [(1)] 因:
$$X(z) = \frac{z}{z -a}, \quad\quad|z|>|a|$$
收敛域为圆外,显然$x(n)$为因果序列。
$$F(z) = X(z)z^{n-1} = \frac{z^n}{ z -a}$$ %= \frac{5z^n}{(z-2)(z+3)}
\item [(2)] 确定收敛区域,围线$C$;
\newpage
\item [(3)] 显然有$n\geqslant0$,仅有极点$z =a$。
\begin{equation*}
\begin{split}
x(n) &= \sum_{k}Res\left[\frac{z^n}{z-a},z=a\right]
= \left[\frac{z^n}{z-a}(z-a)\right]_{z= a} \quad
= a^n
\end{split}
\end{equation*}
$\therefore\quad\quad x(n) = a^{n}u(n) $
\end{enumerate}

\end{frame}
%%%%%%%%%%%%%%%%%%%%%%%%%%%%%%%%%%%%%%%%%%%%%%%%%%%%%%%%%%%%%%%%%%%%%%%%%%%%%%%%%%%%%%%%%%%%%%%



%%%%%%%%%%%%%%%%%%%%%%%%%%%%%%%%%%%%%%%%%%%%%%%%%%%%%%%%%%%%%%%%%%%%%%%%%%%%%%%%%%%%%%%%%%%%%%
\begin{frame}[allowframebreaks]\frametitle{}%[allowframebreaks][shrink]
\begin{example}
设$$X(z)=\frac{1-a^2}{(1-az)(1-\frac{a}{z})},\quad |a|<|z|<\left|\frac{1}{a}\right|,\quad\mbox{求}x(n)$$
\end{example}
\textbf{解:}\par
\begin{enumerate}
\item [(1)] 收敛域为圆环,$|a|<|z|<\left|\frac{1}{a}\right|$, 显然$x(n)$为双边序列。令
$$F(z) = \frac{(1-a^2)\cdot z^n}{-a(z-\frac{1}{a})(z-a)}$$
\item [(2)] 确定收敛区域,围线$C$
\newpage
\item [(3)] 对不同的$n$进行讨论
%\begin{figure}
%\centering
%\includegraphics[width=0.3\textwidth]{blankpic.jpg}
%\end{figure}
\begin{enumerate}
\item [(a)] $n\geqslant0$,显然C内有极点 $z =a$,有
\begin{equation*}
\begin{split}
x(n) &= \sum_{k} Res\left[F(z),z = a\right] \\
&= \frac{(1-a^2)\cdot z^n}{-a(z-\frac{1}{a})(z-a)}(z-a)|_{z=a}  =a^n
\end{split}
\end{equation*}
\item [(b)] 当$n<0$时,$C$内有极点$z=0$,$z=a$,$C$外有极点$z=\frac{1}{a}$,且$z=0$ 处的极点为高阶极点,这时可利用留数辅助定理。\par
首先看是否满足条件:
$$\mbox{有:\quad} X(z) = \frac{(1-a^2)z}{(1-az)(z-a)}\frac{\rightarrow M=1\mbox{次}}
{\rightarrow N=2\mbox{次}}$$
$$\mbox{则:\quad\quad}  n<N-M = 1\mbox{适用于留数辅助定理}$$
\end{enumerate}
\end{enumerate}
\newpage
应用留数辅助定理可得:
\begin{equation*}
\begin{split}
x(n) &= -\sum Res\left[X(z)z^{n-1}\right]_{\mbox{C 外诸极点}}  \\
&= -\frac{(1-a^2)\cdot z^n}{-a(z-\frac{1}{a})(z-a)}(z-\frac{1}{a})\Big|_{z=\frac{1}{a}} \\
&= a^{-n}
\end{split}
\end{equation*}
综合上述分析可得:
\begin{equation*}
x(n) = \left\{
\begin{array}
{r@{,\quad}l}
a^n\quad  & n \geqslant  0 \\
a^{-n}    & n  < 0
\end{array} \right.
\end{equation*}
$$\therefore\quad\quad x(n) = a^{|n|}$$


\end{frame}

%%%%%%%%%%%%%%%%%%%%%%%%%%%%%%%%%%%%%%%%%%%%%%%%%%%%%%%%%%%%%%%%%%%%%%%%%%%%%%%%%%%%%%%%%%%%%%%
%\begin{frame}[allowframebreaks]\frametitle{}%[allowframebreaks][shrink]
%\begin{example}
%    设$X(z)=\frac{5z^{-1}}{1+z^{-1} -6z^{-2}},\quad\quad 2<|z|<3$\par
%\end{example}
%    \textbf{解:}\par
%    收敛域为圆环,显然$x(n)$为双边序列。
%    \begin{equation*}
%     \begin{split}
%            \frac{X(z)}{z} &= \frac{5z^{-2}}{1+z^{-1} -6z^{-2}} =\frac{5z}{z^2 + z -6}= \frac{5z}{(z+3)(z-2)}\\
%                           &= \frac{A_1}{z-2} + \frac{A_2}{z+3}
%      \end{split}
%    \end{equation*}
%    \begin{equation*}
%        \begin{split}
%            A_1   &= \left[\frac{X(z)}{z}(z-2)\right]_{z =2 } =\frac{5}{z+3}|_{z=2} =1\\
%            A_2   &= \left[\frac{X(z)}{z}(z+3)\right]_{z =-3 } =\frac{5}{z-2}|_{z=-} =1
%        \end{split}
%    \end{equation*}
%    $$\frac{X(z)}{z} = \frac{1}{z-2} - \frac{1}{z+3} \quad\quad \Longrightarrow\quad\quad X(z) = \frac{z}{z-2} - \frac{z}{z+3}$$
%    我们知道:
%    \begin{equation}
%        \left\{ \begin{aligned}
%            a^{n}u(n)     &\leftrightarrow \frac{z}{z-a} \quad\quad |z|>|a| \mbox{\quad\quad 因果序列}\\
%          -a^{n}u(-n-1)   &\leftrightarrow \frac{z}{z-a} \quad\quad |z|<|a| \mbox{\quad\quad 反因果序列}
%        \end{aligned} \right.
%    \end{equation}
%    套公式:
%    \begin{equation*}
%       \begin{split}
%          x(n)    &=  2^{n}u(n) - (-(-3)^{n}u(-n-1)]\\
%                  &=  2^{n}u(n) +(-3)^{n}u(-n-1)
%       \end{split}
%    \end{equation*}
%
%
%\end{frame}
%%%%%%%%%%%%%%%%%%%%%%%%%%%%%%%%%%%%%%%%%%%%%%%%%%%%%%%%%%%%%%%%%%%%%%%%%%%%%%%%%%%%%%%%%%%%%%%%
%
%
%
%%%%%%%%%%%%%%%%%%%%%%%%%%%%%%%%%%%%%%%%%%%%%%%%%%%%%%%%%%%%%%%%%%%%%%%%%%%%%%%%%%%%%%%%%%%%%%%
%\begin{frame}[allowframebreaks]\frametitle{}%[allowframebreaks][shrink]
%\begin{example}
%    设$$X(z)=\frac{5z^{-1}}{1+z^{-1} -6z^{-2}},\quad 2<|z|<3,\quad\mbox{求}x(n)$$
%\end{example}
%    \textbf{解:}\par
%    \begin{enumerate}
%      \item $X(z) = \frac{5z}{z^2 + z -6} \quad\quad 2<|z|<3$,收敛域为圆环,显然$x(n)$为双边序列。
%            $$F(z) = X(z)z^{n-1} = \frac{5z^n}{z^2 + z -6} = \frac{5z^n}{(z-2)(z+3)}$$
%      \item 确定收敛区域,围线$C$;
%      \newpage
%      \item 对不同的$n$进行讨论
%      \begin{enumerate}
%        \item $n\geqslant0$,显然极点为 $z =2$ ,$z=-3$,但围线内只有几点$z=2$,依留数定理,有
%              \begin{equation*}
%              \begin{split}
%                   x(n) &= \sum_{k}Res\left[X(z)z^{n-1},z_0 = 2\right]\\
%                        &= \left[X(z)z^{n-1}(z-z_0)\right]_{z_0 = 2} \\
%                        &= \left[\frac{5\cdot z^{n}}{(z+3)}\right]_{z = 2} = 2^n
%              \end{split}
%              \end{equation*}
%        \item 当$n<0$时,$C$内有极点$z=0$,$z=2$,$C$外有极点$z=-3$,且$z=0$处的极点为高阶极点,这时有:
%              $$\mbox{设:\quad} X(z) = \frac{P(z)}{Q(z)}\frac{\rightarrow M=1\mbox{次}}
%                            {\rightarrow N=2\mbox{次}}$$
%              $$\mbox{有:\quad\quad}  n<N-M = 1\mbox{适用于留数辅助定理}$$
%              \begin{equation*}
%              \begin{split}
%                   x(n) &=\sum Res\left[X(z)z^{n-1}\right]_{\mbox{C 内诸极点}}    \\
%                        &= -\sum Res\left[X(z)z^{n-1}\right]_{\mbox{C 外诸极点}}  \\
%                        &=\sum Res\left[X(z)z^{n-1}\right]_{z=0,z=2} \\
%                        &=-\sum Res\left[X(z)z^{n-1}\right]_{z=-3} \\
%                        &= \left[\frac{5\cdot z^{n}}{(z-2)}\right]_{z = -3} = (-3)^n
%              \end{split}
%              \end{equation*}
%      \end{enumerate}
%      \begin{equation}
%        \therefore  x(n) = \left\{
%        \begin{array}
%                {r@{,\quad}l}
%                2^n\quad  & n \geqslant  0 \\
%                (-3)^{n} & n  < 0
%            \end{array} \right.
%      \end{equation}
%      $$\therefore\quad\quad x(n) = 2^{n}u(n) +(-3)^{n}u(-n-1)$$
%    \end{enumerate}
%
%\end{frame}
%%%%%%%%%%%%%%%%%%%%%%%%%%%%%%%%%%%%%%%%%%%%%%%%%%%%%%%%%%%%%%%%%%%%%%%%%%%%%%%%%%%%%%%%%%%%%%%%
%
%
%
%%%%%%%%%%%%%%%%%%%%%%%%%%%%%%%%%%%%%%%%%%%%%%%%%%%%%%%%%%%%%%%%%%%%%%%%%%%%%%%%%%%%%%%%%%%%%%%
%\begin{frame}\frametitle{}%[allowframebreaks][shrink]
%\begin{example}已知
%    $$ X(z)=\frac{1}{1-az^{-1}},\quad |z|>|a|,\mbox{求}x(n)$$\end{example}
%    \textbf{解:}\par
%    \begin{enumerate}
%      \item 因:
%            $$X(z) = \frac{z}{z -a}, \quad\quad|z|>|a|$$,收敛域为圆外,显然$x(n)$为因果序列。
%            $$F(z) = X(z)z^{n-1} = \frac{z^n}{ z -a}$$ %= \frac{5z^n}{(z-2)(z+3)}
%      \item 确定收敛区域,围线$C$;
%
%      \item 显然有$n\geqslant0$,仅有极点$z =a$。
%              \begin{equation*}
%              \begin{split}
%                   x(n) &= \sum_{k}Res\left[\frac{z^n}{z-a},z=a\right]
%                         = \left[\frac{z^n}{z-a}(z-a)\right]_{z= a} \\
%                        &=  a^n
%              \end{split}
%              \end{equation*}
%      $\therefore\quad\quad x(n) = a^{n}u(n) $
%    \end{enumerate}
%
%\end{frame}
%%%%%%%%%%%%%%%%%%%%%%%%%%%%%%%%%%%%%%%%%%%%%%%%%%%%%%%%%%%%%%%%%%%%%%%%%%%%%%%%%%%%%%%%%%%%%%%%
%
%
%
%%%%%%%%%%%%%%%%%%%%%%%%%%%%%%%%%%%%%%%%%%%%%%%%%%%%%%%%%%%%%%%%%%%%%%%%%%%%%%%%%%%%%%%%%%%%%%%
%\begin{frame}[allowframebreaks]\frametitle{}%[allowframebreaks][shrink]
%\begin{example}
%设$$X(z)=\frac{1-a^2}{(1-az)(1-\frac{a}{z})},\quad |a|<|z|<\left|\frac{1}{a}\right|,\quad\mbox{求}x(n)$$
%\end{example}
%    \textbf{解:}\par
%    \begin{enumerate}
%      \item $$F(z) = \frac{(1-a^2)\cdot z^n}{-a(z-\frac{1}{a})(z-a)}$$,
%            收敛域为圆环,$|a|<|z|<\left|\frac{1}{a}\right|$, 显然$x(n)$为双边序列。
%            %$$F(z) = X(z)z^{n-1} = \frac{5z^n}{z^2 + z -6} = \frac{5z^n}{(z-2)(z+3)}$$
%      \item 确定收敛区域,围线$C$,对不同的$n$进行讨论
%\begin{figure}
%\centering
%\includegraphics[width=0.3\textwidth]{blankpic.jpg}
%\end{figure}
%      \begin{enumerate}
%        \item $n\geqslant0$,显然C内有极点 $z =a$,有
%              \begin{equation*}
%              \begin{split}
%                   x(n) &= \sum_{k} Res\left[F(z),z = a\right] \\
%                        &= \frac{(1-a^2)\cdot z^n}{-a(z-\frac{1}{a})(z-a)}(z-a)|_{z=a}  =a^n
%              \end{split}
%              \end{equation*}
%        \item 当$n<0$时,$C$内有极点$z=0$,$z=a$,$C$外有极点$z=\frac{1}{a}$,且$z=0$ 处的极点为高阶极点,这时有:
%              $$\mbox{设:\quad} X(z) = X(z)=\frac{(1-a^2)z}{(1-az)(z-a)}\frac{\rightarrow M=1\mbox{次}}
%                            {\rightarrow N=2\mbox{次}}$$
%              $$\mbox{有:\quad\quad}  n<N-M = 1\mbox{适用于留数辅助定理}$$
%              \begin{equation*}
%              \begin{split}
%                   x(n) &= -\sum Res\left[X(z)z^{n-1}\right]_{\mbox{C 外诸极点}}  \\
%                        &= -\frac{(1-a^2)\cdot z^n}{-a(z-\frac{1}{a})(z-a)}(z-\frac{1}{a})|_{z=\frac{1}{a}} \\
%                        &= a^{-n}
%              \end{split}
%              \end{equation*}
%      \end{enumerate}
%      \begin{equation}
%        \therefore  x(n) = \left\{
%        \begin{array}
%                {r@{,\quad}l}
%                a^n\quad  & n \geqslant  0 \\
%                a^{-n}    & n  < 0
%            \end{array} \right.
%      \end{equation}
%      $$\therefore\quad\quad x(n) = a^{|n|}$$
%    \end{enumerate}
%
%\end{frame}

\subsection*{Z变换的性质和定理}

%%%%%%%%%%%%%%%%%%%%%%%%%%%%%%%%%%%%%%%%%%%%%%%%%%%%%%%%%%%%%%%%%%%%%%%%%%%%%%%%%%%%%%%%%%%%%%
\begin{frame}[shrink]\frametitle{移位性质}%[allowframebreaks][shrink]

\begin{dablock}
$$\mbox{设:\quad} x(n)\leftrightarrow X(z),\quad\quad \mbox{则:}x(n-n_{0})\leftrightarrow z^{-n_0}X(z)$$
\end{dablock}
\begin{daproof}
\begin{equation*}
\begin{split}
ZT[x(n)]    &= \sum_{n=-\infty}^{\infty}x(n)z^{-n}\\
ZT[x(n-n_0)] &= \sum_{n=-\infty}^{\infty}x(n-n_0)z^{-n}\mbox{\quad\quad 令:}m= n-n_0\\
&= \sum_{m=-\infty}^{\infty}x(m)z^{-(m+n_0)}  = z^{-n_0}\sum_{m=-\infty}^{\infty}x(m)z^{-m}\\
&= z^{-n_0}X(z)
\end{split}
\end{equation*}
\end{daproof}
\end{frame}
%%%%%%%%%%%%%%%%%%%%%%%%%%%%%%%%%%%%%%%%%%%%%%%%%%%%%%%%%%%%%%%%%%%%%%%%%%%%%%%%%%%%%%%%%%%%%%%


%%%%%%%%%%%%%%%%%%%%%%%%%%%%%%%%%%%%%%%%%%%%%%%%%%%%%%%%%%%%%%%%%%%%%%%%%%%%%%%%%%%%%%%%%%%%%%
\begin{frame}[shrink]\frametitle{序列乘以指数序列的性质}%[allowframebreaks][shrink]
%序列乘以$n$的$ZT$
\begin{dablock}
$$\mbox{设:} x(n)\leftrightarrow X(z),R_{x^-}<|a^{-1}z|<R_{x^+} \quad y(n) = a^n x(n)\quad a\in R$$
$$\mbox{则:}a^nx(n)\leftrightarrow X(a^{-1}z)\quad\quad |a|R_{x^-}<|z|<|a|R_{x^+} $$
\end{dablock}
\begin{daproof}
\begin{equation*}
\begin{split}
X(z)            &= \sum_{n=-\infty}^{\infty}x(n)z^{-n}\\
Y(z)            &= \sum_{n=-\infty}^{\infty}a^nx(n)z^{-n} = \sum_{n=-\infty}^{\infty}x(n)(a^{-1}z)^{-n}\\
&= X(a^{-1}z)
\end{split}
\end{equation*}
$$\mbox{又}\because R_{x^-}<|a^{-1}z|<R_{x^+}\quad\therefore |a|R_{x^-}<|z|<|a|R_{x^+}$$
\end{daproof}
\end{frame}
%%%%%%%%%%%%%%%%%%%%%%%%%%%%%%%%%%%%%%%%%%%%%%%%%%%%%%%%%%%%%%%%%%%%%%%%%%%%%%%%%%%%%%%%%%%%%%%


%%%%%%%%%%%%%%%%%%%%%%%%%%%%%%%%%%%%%%%%%%%%%%%%%%%%%%%%%%%%%%%%%%%%%%%%%%%%%%%%%%%%%%%%%%%%%%
\begin{frame}[shrink]\frametitle{序列乘以$n$的$ZT$}%[allowframebreaks][shrink]
%序列乘以$n$的$ZT$
\begin{dablock}
$$\mbox{设:\quad} x(n)\leftrightarrow X(z),\quad\quad \mbox{则:}nx(n)\leftrightarrow -z\frac{dX(z)}{dz}$$

\end{dablock}
\begin{daproof}
\begin{equation*}
\begin{split}
X(z)            &= \sum_{n=-\infty}^{\infty}x(n)z^{-n}\\
\frac{dX(z)}{dz} &= -\sum_{n=-\infty}^{\infty}x(n) n z^{-n-1}
= -z^{-1}\sum_{n=-\infty}^{\infty}nx(n) z^{-n} \\
&= -z^{-1}ZT[nx(n)]
\end{split}
\end{equation*}
$$\therefore \quad ZT[nx(n)] = -z\frac{dX(z)}{dz}$$
\end{daproof}
\end{frame}
%%%%%%%%%%%%%%%%%%%%%%%%%%%%%%%%%%%%%%%%%%%%%%%%%%%%%%%%%%%%%%%%%%%%%%%%%%%%%%%%%%%%%%%%%%%%%%%

%%%%%%%%%%%%%%%%%%%%%%%%%%%%%%%%%%%%%%%%%%%%%%%%%%%%%%%%%%%%%%%%%%%%%%%%%%%%%%%%%%%%%%%%%%%%%%
\begin{frame}[shrink]\frametitle{复共轭序列的ZT}%[allowframebreaks][shrink]
%序列乘以$n$的$ZT$
\begin{dablock}
$$\mbox{设:\quad} x(n)\longleftrightarrow X(z),\quad\quad \mbox{则:}x^*(n)\longleftrightarrow X^*(z^*)$$

\end{dablock}
\begin{daproof}
\begin{equation*}
\begin{split}
ZT[x(n)]    &= \sum_{n=-\infty}^{\infty}x(n)z^{-n}\\
ZT[x^*(n)]  &= \sum_{n=-\infty}^{\infty}x^*(n)z^{-n}
= \sum_{n=-\infty}^{\infty}\left[x(n)(z^*)^{-n}\right]^* \\
&= X^*(z^*)
\end{split}
\end{equation*}
%$$\therefore \quad ZT[nx(n)] = -z\frac{dX(z)}{dz}$$
\end{daproof}
\end{frame}
%%%%%%%%%%%%%%%%%%%%%%%%%%%%%%%%%%%%%%%%%%%%%%%%%%%%%%%%%%%%%%%%%%%%%%%%%%%%%%%%%%%%%%%%%%%%%%%



%%%%%%%%%%%%%%%%%%%%%%%%%%%%%%%%%%%%%%%%%%%%%%%%%%%%%%%%%%%%%%%%%%%%%%%%%%%%%%%%%%%%%%%%%%%%%%
\begin{frame}[shrink]\frametitle{初值定理}%[allowframebreaks][shrink]
\begin{dablock}
设$x(n)$是因果序列,$x(n)\leftrightarrow X(z)$,则有 $$x(0) = \lim_{z\rightarrow \infty} X(z)$$
\end{dablock}
\begin{daproof}
\begin{equation*}
\begin{split}
X(z)         &= \sum_{n=0}^{\infty}x(n)z^{-n}= x(0) + x(1)z^{-1}+x(2)z^{-2}+\cdots\\
\therefore\  x(0) &= \lim_{z\rightarrow \infty} X(z)
\end{split}
\end{equation*}
\end{daproof}
\end{frame}
%%%%%%%%%%%%%%%%%%%%%%%%%%%%%%%%%%%%%%%%%%%%%%%%%%%%%%%%%%%%%%%%%%%%%%%%%%%%%%%%%%%%%%%%%%%%%%%



%%%%%%%%%%%%%%%%%%%%%%%%%%%%%%%%%%%%%%%%%%%%%%%%%%%%%%%%%%%%%%%%%%%%%%%%%%%%%%%%%%%%%%%%%%%%%%
\begin{frame}[shrink]\frametitle{终值定理}%[allowframebreaks][shrink]
\begin{dablock}
\quad 设$x(n)$是因果序列,其$Z$变换的极点,除可以有一个一阶极点在$z=1$ 上,其他极点都在单位园内,则:
$$\lim_{n\rightarrow \infty}x(n) = \lim_{z\rightarrow 1}(z-1)X(z)$$
\end{dablock}
%\begin{daproof}
\textbf{证明:}
\begin{equation*}
\begin{split}
(z-1)X(z) &= zX(z) -X(z) \\
&= ZT[x(n+1)] - ZT[x(n)]\quad\quad\mbox{根据移位性质}\\
&= \sum_{n=-\infty}^{\infty}[x(n+1)-x(n)]z^{-n}  \\
&  \because  x(n)\quad\mbox{因果} \therefore x(n)=0,n<0  \\
(z-1)X(z) &= \lim_{n\rightarrow \infty}\left[\sum_{m=-1}^{n}x(m+1)z^{-m}- \sum_{m=0}^{n}x(m)z^{-m}\right]\\
%                      &  \because (z-1)X(z)\mbox{在单位圆上无极点,上式两端对$z=1$取极限}\\
%            \lim_{z\rightarrow 1}(z-1)X(z)
%                      &= \lim_{n\rightarrow \infty}\left[\sum_{m=-1}^{n}x(m+1)- \sum_{m=0}^{n}x(m)\right]\\
%                      &= \lim_{n\rightarrow \infty} \left[(x(0)+x(1)+\cdots +x(n+1))-(x(0)+x(1) + \cdots +x(n))\right] \\
%                      &= \lim_{n\rightarrow \infty}x(n+1)\\
%                      &= \lim_{n\rightarrow \infty}x(n)
\end{split}
\end{equation*}
%        如$X(z)$在单位圆上没有极点,则$x(\infty)=0$
%\end{daproof}
\end{frame}
%%%%%%%%%%%%%%%%%%%%%%%%%%%%%%%%%%%%%%%%%%%%%%%%%%%%%%%%%%%%%%%%%%%%%%%%%%%%%%%%%%%%%%%%%%%%%%%

%%%%%%%%%%%%%%%%%%%%%%%%%%%%%%%%%%%%%%%%%%%%%%%%%%%%%%%%%%%%%%%%%%%%%%%%%%%%%%%%%%%%%%%%%%%%%%
\begin{frame}[shrink]\frametitle{终值定理}%[allowframebreaks][shrink]
%
%        \quad 设$x(n)$是因果序列,其$Z$ 变换的极点,除可以有一个一阶极点在$z=1$ 上,其他极点都在单位园内,则:
%        $$\lim_{n\rightarrow \infty}x(n) = \lim_{z\rightarrow 1}(z-1)X(z)$$
%        证明:
\vspace{-0.5cm}
\begin{equation*}
\begin{split}
%            (z-1)X(z) &= zX(z) -X(z) \\
%%                      &= ZT[x(n+1)] - ZT[x(n)]\quad\quad\mbox{根据移位性质}\\
%                      &= \sum_{n=-\infty}^{\infty}[x(n+1)-x(n)]z^{-n}  \\
%                      &  \because  x(n)\quad\mbox{因果} \therefore x(n)=0,n<0  \\
(z-1)X(z) &= \lim_{n\rightarrow \infty}\left[\sum_{m=-1}^{n}x(m+1)z^{-m}- \sum_{m=0}^{n}x(m)z^{-m}\right]\\
\because (z-1)X(z)       &\mbox{在单位圆上无极点,上式两端对$z=1$取极限}\\
\lim_{z\rightarrow 1}(z-1)X(z)
&= \lim_{n\rightarrow \infty}\left[\sum_{m=-1}^{n}x(m+1)- \sum_{m=0}^{n}x(m)\right]\\
&= \lim_{n\rightarrow \infty} \big[(x(0)+x(1)+\cdots +x(n+1))\\
&\quad  \qquad\qquad-(x(0)+x(1) + \cdots +x(n))\big] \\
&= \lim_{n\rightarrow \infty}x(n+1)\\
&= \lim_{n\rightarrow \infty}x(n)
\end{split}
\end{equation*}
如$X(z)$在单位圆上没有极点,则$x(\infty)=0$

\end{frame}
%%%%%%%%%%%%%%%%%%%%%%%%%%%%%%%%%%%%%%%%%%%%%%%%%%%%%%%%%%%%%%%%%%%%%%%%%%%%%%%%%%%%%%%%%%%%%%%


%%%%%%%%%%%%%%%%%%%%%%%%%%%%%%%%%%%%%%%%%%%%%%%%%%%%%%%%%%%%%%%%%%%%%%%%%%%%%%%%%%%%%%%%%%%%%%
\begin{frame}\frametitle{时域卷积定理}%[allowframebreaks][shrink]
\begin{dablock}
设
$$X(n)\leftrightarrow X(z)$$
$$h(n)\leftrightarrow H(z)$$
$$y(n)\leftrightarrow Y(z)$$
如$$y(n) = x(n)*h(n)$$
则有:
$$Y(z) = X(z)\cdot H(z)$$
\end{dablock}
%证明:
%$$ y(n)      = x(n)*h(n) = \sum_{k=-\infty}^{\infty}x(k)h(n-k) $$
%\begin{equation*}
%\begin{split}
%    %y(n)      &= x(n)*h(n) = \sum_{k=-\infty}^{\infty}x(k)h(n-k) \\
%    Y(z)      &= \sum_{n=-\infty}^{\infty}y(n)z^{-n}
%               =\sum_{n=-\infty}^{\infty}\left[\sum_{k=-\infty}^{\infty}x(k)h(n-k)\right]z^{-n} \\
%              &= \sum_{k=-\infty}^{\infty}x(k)\sum_{n=-\infty}^{\infty}h(n-k)z^{-n}
%              = \sum_{k=-\infty}^{\infty}x(k)\cdot z^{-k}\cdot H(z) \\
%              &= H(z) \sum_{k=-\infty}^{\infty}x(k)\cdot z^{-k}  = X(z)\cdot H(z)
%\end{split}
%\end{equation*}

\end{frame}
%%%%%%%%%%%%%%%%%%%%%%%%%%%%%%%%%%%%%%%%%%%%%%%%%%%%%%%%%%%%%%%%%%%%%%%%%%%%%%%%%%%%%%%%%%%%%%%



%%%%%%%%%%%%%%%%%%%%%%%%%%%%%%%%%%%%%%%%%%%%%%%%%%%%%%%%%%%%%%%%%%%%%%%%%%%%%%%%%%%%%%%%%%%%%%
\begin{frame}[shrink]\frametitle{时域卷积定理}%[allowframebreaks][shrink]

%设\quad $X(n)\leftrightarrow X(z), \quad\quad h(n)\leftrightarrow H(z),\quad\quad y(n)\leftrightarrow Y(z)$\par
%且\quad $y(n) = x(n)*h(n)$,则有:\quad\quad $Y(z) = X(z)\cdot H(z)$

\begin{daproof}
$$ y(n)      = x(n)*h(n) = \sum_{k=-\infty}^{\infty}x(k)h(n-k) $$
\begin{equation*}
\begin{split}
%y(n)      &= x(n)*h(n) = \sum_{k=-\infty}^{\infty}x(k)h(n-k) \\
Y(z)      &= \sum_{n=-\infty}^{\infty}y(n)z^{-n}
=\sum_{n=-\infty}^{\infty}\left[\sum_{k=-\infty}^{\infty}x(k)h(n-k)\right]z^{-n} \\
&= \sum_{k=-\infty}^{\infty}x(k)\sum_{n=-\infty}^{\infty}h(n-k)z^{-n}\\
&= \sum_{k=-\infty}^{\infty}x(k)\cdot z^{-k}\cdot H(z) \\
&= H(z) \sum_{k=-\infty}^{\infty}x(k)\cdot z^{-k}\quad\quad  = X(z)\cdot H(z)
\end{split}
\end{equation*}
\end{daproof}

\end{frame}
%%%%%%%%%%%%%%%%%%%%%%%%%%%%%%%%%%%%%%%%%%%%%%%%%%%%%%%%%%%%%%%%%%%%%%%%%%%%%%%%%%%%%%%%%%%%%%%

%\section{利用Z变换分析信号和系统的频域特性}
%\subsection{传输函数与系统函数}
%%%%%%%%%%%%%%%%%%%%%%%%%%%%%%%%%%%%%%%%%%%%%%%%%%%%%%%%%%%%%%%%%%%%%%%%%%%%%%%%%%%%%%%%%%%%%%%
%\begin{frame}\frametitle{传输函数与系统函数}%[allowframebreaks][shrink]
%\begin{enumerate}
%  \item 传输函数
%
%      \par $h(n)$为系统的单位脉冲响应,即为系统对单位脉冲序列$\delta(n)$的零状态响应:
%      \begin{equation*}
%      \begin{split}
%      \mbox{令:}\quad H(e^{j\omega}) &= FT[h(n)] = \sum_{n=-\infty}^{\infty}h(n)e^{-j\omega n}\\
%                         &= |H(e^{j\omega})|\cdot e^{j\varphi(\omega)}
%      \end{split}
%      \end{equation*}
%
%      这里$|H(e^{j\omega})|$称为幅频特性函数,$\varphi(\omega)$称为相频特性函数。
%
%  \item 系统函数
%        $$h(n) \leftrightarrow H(z)\quad\quad : H(z)\mbox{称为系统函数}$$
%        小结:
%        \newline \newline \newline
%\end{enumerate}
%\end{frame}
%
%
%\begin{frame}[shrink]\frametitle{系统函数$H(z)$与差分方程}%[allowframebreaks][shrink]
%
%        $$\mbox{已知}\sum_{k=0}^{N}a_k y(n-k) = \sum_{k=0}^{M}b_k x(n-k),\mbox{求其对应的}H(z).$$
%        对方程两边进行Z变换, 设:$x(n)\leftrightarrow X(z)$,$y(n)\leftrightarrow Y(z)$,可得:
%        \begin{equation*}
%        \begin{split}
%          ZT\left[\sum_{k=0}^{N}a_k y(n-k)\right] &= ZT\left[\sum_{k=0}^{M}b_k x(n-k)\right]\\
%             \sum_{k=0}^{N}a_k z^{-k} Y(z)        &=  \sum_{k=0}^{M} b_k z^{-k} X(z)\\
%             H(z) = \frac{Y(z)}{X(z)} &= \frac{\sum_{k=0}^{M} b_k z^{-k}}{\sum_{k=0}^{N}a_k z^{-k}}
%          \end{split}
%        \end{equation*}
%
%\end{frame}
%%%%%%%%%%%%%%%%%%%%%%%%%%%%%%%%%%%%%%%%%%%%%%%%%%%%%%%%%%%%%%%%%%%%%%%%%%%%%%%%%%%%%%%%%%%%%%%%
%
%
%
%%%%%%%%%%%%%%%%%%%%%%%%%%%%%%%%%%%%%%%%%%%%%%%%%%%%%%%%%%%%%%%%%%%%%%%%%%%%%%%%%%%%%%%%%%%%%%%
%\begin{frame}[shrink]\frametitle{$H(e^{j\omega})$ 的物理意义-对于单频复指数信号$e^{j\omega n}$的响应}%[allowframebreaks][shrink]
%
% %        \textbf{(a)} 对于单频复指数信号$e^{j\omega n}$的响应\par
%         $H(e^{j\omega})$表示系统对特征序列$e^{j\omega n}$的响应特性,这也是$H(e^{j\omega})$的物理意义。
%
%         如系统输入信号为$x(n)=e^{j\omega n}$ ,则输出信号为:
%         $$y(n) = x(n)*h(n) = \sum_{m=-\infty}^{\infty}h(m)x(n-m)=\sum_{m=-\infty}^{\infty}h(m)e^{j\omega (n-m)}$$
%         $$ = e^{j\omega n}\sum_{m=-\infty}^{\infty}h(m)e^{-j\omega m} =H(e^{j\omega})e^{j\omega n}\quad
%         \quad\quad\quad\quad\quad $$
%         $$\mbox{即:\quad} y(n) = H(e^{j\omega})e^{j\omega n} = |H(e^{j\omega})|e^{j(\omega n + \varphi(\omega))}  \quad
%         \quad\quad\quad\quad\quad $$
%         可见,系统的输入为单频复指数函数$e^{j\omega n}$时,输出仍为单频复指数函数,只不过幅度放大$|H(e^{j\omega})|$ 倍,相移$\varphi(\omega)$。
%\end{frame}
%\begin{frame}[shrink]\frametitle{$H(e^{j\omega})$ 的物理意义}%[allowframebreaks][shrink]
%         \textbf{(b)}对正弦信号$x(n)=cos(\omega n)$ 的响应 %\par
%         \begin{equation*}
%          \begin{split}
%            x(n)      &= cos(\omega n) = \frac{1}{2}(e^{j\omega n}+e^{-j\omega n}) \\
%            y(n)      &= x(n)*h(n)\\
%                      &= \frac{1}{2}\left(H(e^{j\omega})e^{j\omega n}+ H(e^{j(-\omega)})e^{-j\omega n}\right)\\
%                      &= \frac{1}{2}\left(|H(e^{j\omega})|e^{j\varphi(\omega)}e^{j\omega n}+|H(e^{-j\omega})|e^{j\varphi(-\omega)}e^{-j\omega n}\right) \\
%                      &\quad  \quad\quad\quad\quad\mbox{设}  h(n)\in R\quad H^{*}(e^{j\omega})=H(e^{-j\omega}) \\
%                      &\quad  \quad\quad\therefore |H(e^{j\omega})| = |H(e^{-j\omega})|\quad \varphi(-\omega) = -\varphi(\omega) \\
%                      &= \frac{1}{2}|H(e^{j\omega})|\left(e^{j(\varphi(\omega)+\omega n)}+e^{-j(\varphi(\omega)+\omega n)}\right)\\
%                      &= |H(e^{j\omega})|cos(\omega n +\varphi(\omega))
%          \end{split}
%         \end{equation*}
%         可见,线性时不变系统对单频正弦信号$cos(\omega n)$的响应为同频正弦信号,其幅度放大$|H(e^{j\omega n})|$ 倍,相移增加$\varphi(\omega)$
%\end{frame}
%\begin{frame}[shrink]\frametitle{说明}%[allowframebreaks][shrink]
%         \textbf{(c) 说明:} \par
%         对于一般序列x(n),可通过傅立叶变换分解为一系列正弦函数的加权和。此时可通过$H(e^{j\omega n})$对不同的频率成分进行加权处理。
%\end{frame}
%%%%%%%%%%%%%%%%%%%%%%%%%%%%%%%%%%%%%%%%%%%%%%%%%%%%%%%%%%%%%%%%%%%%%%%%%%%%%%%%%%%%%%%%%%%%%%%%
%
%
%\subsection{系统的因果性和稳定性}
%%%%%%%%%%%%%%%%%%%%%%%%%%%%%%%%%%%%%%%%%%%%%%%%%%%%%%%%%%%%%%%%%%%%%%%%%%%%%%%%%%%%%%%%%%%%%%%
%\begin{frame}[allowframebreaks]\frametitle{}%[allowframebreaks][shrink]
%\begin{enumerate}
%\item 因果系统的收敛域特点
%     \begin{enumerate}
%       \item
%           $$\mbox{系统因果}\Longleftrightarrow h(n)=0,n<0\quad\mbox{(第一章讨论过)}$$
%       \item 从Z变换的角度看,则有
%           \begin{equation}
%           \begin{split}
%           \mbox{系统因果} &\longleftrightarrow   \mbox{$H(z)$的收敛域包含$\infty$点,或$|z|>R_{x^{-}}$}\\
%                           &\quad\quad\quad       \mbox{即极点在某个圆内,收敛域在圆外}
%           \end{split}
%           \end{equation}
%           \textbf{证明:} \par
%           1 充分性 \quad($\Longleftarrow$),设$H(z)$ 在$z=\infty$ 处收敛,往证系统因果。
%           \begin{equation}
%           \begin{split}
%               H(z) &=  \sum_{n=-\infty}^{\infty}h(n)z^{-n}\\
%                    &=  \underbrace{\sum_{n=-\infty}^{-1}h(n)z^{-n}}_{\mbox{正项级数}} \quad
%                        + \quad\underbrace{\sum_{n=0}^{\infty}h(n)z^{-n}}_{\mbox{负项级数}}
%           \end{split}
%           \end{equation}
%           如$z=\infty$处收敛,则必有$n\leqslant -1$ 时,$h(n)=0$  \par
%           即: $n<0$时,$h(n)=0$,所以系统因果。
%
%           2 必要性 \quad($\Longrightarrow$),设系统因果,往证$H(z)$在$z=\infty$ 处收敛。
%           \par 因系统因果,则有$h(n)=0,n<0$,
%           $$H(z)=\sum_{n=0}^{\infty}h(n)z^{-n}= h(0)+h(1)z^{-1}+h(2)z^{-2}+h(3)z^{-3}+\cdots$$
%           显然,在$z=\infty$处,$H(z)$ 收敛。
%     \end{enumerate}
%
%\item 系统稳定的收敛域的特点
%
%    \begin{enumerate}
%      \item $$\mbox{系统稳定}\longleftrightarrow  \sum_{n=-\infty}^{\infty}|h(n)|<\infty
%              \quad\mbox{也是$H(e^{j\omega})$ 存在的条件}$$
%      \item 从Z变换的角度看,则有
%            $$\mbox{系统稳定}\Longleftrightarrow  H(z)\mbox{的收敛域包含单位圆}$$
%            \textbf{证明:} \par
%           1 充分性 \quad($\Longleftarrow$),已知$H(z)$收敛域包含单位圆,往证系统因果。
%           \begin{equation*}
%           \begin{split}
%               H(z) &=  \sum_{n=-\infty}^{\infty}h(n)z^{-n}\\
%                    &   \mbox{令$z=1$,则有} \\
%               H(1) &=  \sum_{n=-\infty}^{\infty}h(n)  < \infty\\
%               \therefore
%               y(n) &= h(n)*x(n) = \sum_{k=-\infty}^{\infty}h(k)x(n-k)\\
%                    & \because x(n)\mbox{为有界输入},\quad\therefore |x(n)|<B \\
%               \therefore
%               y(n) &= h(n)*x(n) = \sum_{k=-\infty}^{\infty}h(k)x(n-k)\leqslant B\sum_{k=-\infty}^{\infty}h(k)<\infty\\
%                    &  \therefore \mbox{系统稳定}
%           \end{split}
%           \end{equation*}
%
%           2 必要性 \quad($\Longrightarrow$),已知系统稳定,往证$H(z)$收敛域包含单位圆。
%           \begin{equation*}
%           \begin{split}
%           \sum_{n=-\infty}^{\infty}|h(n)|
%                  &= \sum_{n=-\infty}^{\infty}|h(n)|\cdot|e^{-j\omega n}|\quad\quad(\mbox{显然$|e^{-j\omega n}|=1$})\\
%                  &\geqslant \left|\sum_{n=-\infty}^{\infty}h(n)e^{-j\omega n}\right| =
%                                        \left|\left[\sum_{n=-\infty}^{\infty}h(n)z^{-n}\right]\right|_{z=e^{j\omega}}\\
%                  &= \left|H(z)\right|_{z=e^{j\omega}}
%           \end{split}
%           \end{equation*}
%           $$\because \quad\sum_{n=-\infty}^{\infty}|h(n)| <\infty
%              \quad\quad\quad\therefore \left|H(z)|_{z=e^{j\omega}}\right|<\infty$$
%           $H(z)$在单位圆上收敛,即$|z|=1$ 是收敛域的一部分。
%    \end{enumerate}
%\item 因果稳定系统的收敛域、极点分布特点
%      \begin{equation*}
%           \begin{split}
%           \mbox{系统因果} &\Rightarrow  r<|z| \quad \mbox{收敛域在某个圆外(或收敛域包含无穷远点)}\\
%           \mbox{系统稳定} &\Rightarrow    0 <r<1,     \quad \mbox{收敛域包含单位圆}
%      \end{split}
%      \end{equation*}
%      系统因果稳定  :1 $r<|z|\mbox{且} 0 <r<1 $ 或 2. X(z)的极点全部在单位圆内部。
%\end{enumerate}
%\end{frame}
%%%%%%%%%%%%%%%%%%%%%%%%%%%%%%%%%%%%%%%%%%%%%%%%%%%%%%%%%%%%%%%%%%%%%%%%%%%%%%%%%%%%%%%%%%%%%%%%
%
%
%
%%%%%%%%%%%%%%%%%%%%%%%%%%%%%%%%%%%%%%%%%%%%%%%%%%%%%%%%%%%%%%%%%%%%%%%%%%%%%%%%%%%%%%%%%%%%%%%
%\begin{frame}[allowframebreaks]\frametitle{}%[allowframebreaks][shrink]
%\begin{example}
%    设$X(z)=\frac{5z^{-1}}{1+z^{-1} -6z^{-2}},\quad\quad 2<|z|<3$\par
%\end{example}
%    \textbf{解:}\par
%    收敛域为圆环,显然$x(n)$为双边序列。
%    \begin{equation*}
%     \begin{split}
%            \frac{X(z)}{z} &= \frac{5z^{-2}}{1+z^{-1} -6z^{-2}} =\frac{5z}{z^2 + z -6}= \frac{5z}{(z+3)(z-2)}\\
%                           &= \frac{A_1}{z-2} + \frac{A_2}{z+3}
%      \end{split}
%    \end{equation*}
%    \begin{equation*}
%        \begin{split}
%            A_1   &= \left[\frac{X(z)}{z}(z-2)\right]_{z =2 } =\frac{5}{z+3}|_{z=2} =1\\
%            A_2   &= \left[\frac{X(z)}{z}(z+3)\right]_{z =-3 } =\frac{5}{z-2}|_{z=-} =1
%        \end{split}
%    \end{equation*}
%    $$\frac{X(z)}{z} = \frac{1}{z-2} - \frac{1}{z+3} \quad\quad \Longrightarrow\quad\quad X(z) = \frac{z}{z-2} - \frac{z}{z+3}$$
%    我们知道:
%    \begin{equation}
%        \left\{ \begin{aligned}
%            a^{n}u(n)     &\leftrightarrow \frac{z}{z-a} \quad\quad |z|>|a| \mbox{\quad\quad 因果序列}\\
%          -a^{n}u(-n-1)   &\leftrightarrow \frac{z}{z-a} \quad\quad |z|<|a| \mbox{\quad\quad 反因果序列}
%        \end{aligned} \right.
%    \end{equation}
%    套公式:
%    \begin{equation*}
%       \begin{split}
%          x(n)    &=  2^{n}u(n) - (-(-3)^{n}u(-n-1)]\\
%                  &=  2^{n}u(n) +(-3)^{n}u(-n-1)
%       \end{split}
%    \end{equation*}
%
%
%\end{frame}
%%%%%%%%%%%%%%%%%%%%%%%%%%%%%%%%%%%%%%%%%%%%%%%%%%%%%%%%%%%%%%%%%%%%%%%%%%%%%%%%%%%%%%%%%%%%%%%%
%
%
%
%%%%%%%%%%%%%%%%%%%%%%%%%%%%%%%%%%%%%%%%%%%%%%%%%%%%%%%%%%%%%%%%%%%%%%%%%%%%%%%%%%%%%%%%%%%%%%%
%\begin{frame}[allowframebreaks]\frametitle{}%[allowframebreaks][shrink]
%\begin{example}
%    设$$X(z)=\frac{5z^{-1}}{1+z^{-1} -6z^{-2}},\quad 2<|z|<3,\quad\mbox{求}x(n)$$
%\end{example}
%    \textbf{解:}\par
%    \begin{enumerate}
%      \item $X(z) = =\frac{5z}{z^2 + z -6} \quad\quad 2<|z|<3$,收敛域为圆环,显然$x(n)$ 为双边序列。
%            $$F(z) = X(z)z^{n-1} = \frac{5z^n}{z^2 + z -6} = \frac{5z^n}{(z-2)(z+3)}$$
%      \item 确定收敛区域,围线$C$;
%      \newpage
%      \item 对不同的$n$进行讨论
%      \begin{enumerate}
%        \item $n\geqslant0$,显然极点为 $z =2$ ,$z=-3$,但围线内只有几点$z=2$,依留数定理,有
%              \begin{equation*}
%              \begin{split}
%                   x(n) &= \sum_{k}Res\left[X(z)z^{n-1},z_0 = 2\right]\\
%                        &= \left[X(z)z^{n-1}(z-z_0)\right]_{z_0 = 2} \\
%                        &= \left[\frac{5\cdot z^{n}}{(z+3)}\right]_{z = 2} = 2^n
%              \end{split}
%              \end{equation*}
%        \item 当$n<0$时,$C$内有极点$z=0$,$z=2$,$C$外有极点$z=-3$,且$z=0$ 处的极点为高阶极点,这时有:
%              $$\mbox{设:\quad} X(z) = \frac{P(z)}{Q(z)}\frac{\rightarrow M=1\mbox{次}}
%                            {\rightarrow N=2\mbox{次}}$$
%              $$\mbox{有:\quad\quad}  n<N-M = 1\mbox{适用于留数辅助定理}$$
%              \begin{equation*}
%              \begin{split}
%                   x(n) &=\sum Res\left[X(z)z^{n-1}\right]_{\mbox{C内诸极点}}    \\
%                        &= -\sum Res\left[X(z)z^{n-1}\right]_{\mbox{C外诸极点}}  \\
%                        &=\sum Res\left[X(z)z^{n-1}\right]_{z=0,z=2} \\
%                        &=-\sum Res\left[X(z)z^{n-1}\right]_{z=-3} \\
%                        &= \left[\frac{5\cdot z^{n}}{(z-2)}\right]_{z = -3} = (-3)^n
%              \end{split}
%              \end{equation*}
%      \end{enumerate}
%      \begin{equation}
%        \therefore  x(n) = \left\{
%        \begin{array}
%                {r@{,\quad}l}
%                2^n\quad  & n \geqslant  0 \\
%                (-3)^{n} & n  < 0
%            \end{array} \right.
%      \end{equation}
%      $$\therefore\quad\quad x(n) = 2^{n}u(n) +(-3)^{n}u(-n-1)$$
%    \end{enumerate}
%
%\end{frame}
%%%%%%%%%%%%%%%%%%%%%%%%%%%%%%%%%%%%%%%%%%%%%%%%%%%%%%%%%%%%%%%%%%%%%%%%%%%%%%%%%%%%%%%%%%%%%%%%
%
%
%
%%%%%%%%%%%%%%%%%%%%%%%%%%%%%%%%%%%%%%%%%%%%%%%%%%%%%%%%%%%%%%%%%%%%%%%%%%%%%%%%%%%%%%%%%%%%%%%
%\begin{frame}\frametitle{}%[allowframebreaks][shrink]
%\begin{example}已知
%    $$ X(z)=\frac{1}{1-az^{-1}},\quad |z|>|a|,\mbox{求}x(n)$$\end{example}
%    \textbf{解:}\par
%    \begin{enumerate}
%      \item 因:
%            $$X(z) = \frac{z}{z -a}, \quad\quad|z|>|a|$$,收敛域为圆外,显然$x(n)$ 为因果序列。
%            $$F(z) = X(z)z^{n-1} = \frac{z^n}{ z -a}$$ %= \frac{5z^n}{(z-2)(z+3)}
%      \item 确定收敛区域,围线$C$;
%
%      \item 显然有$n\geqslant0$,仅有极点$z =a$。
%              \begin{equation*}
%              \begin{split}
%                   x(n) &= \sum_{k}Res\left[\frac{z^n}{z-a},z=a\right]
%                         = \left[\frac{z^n}{z-a}(z-a)\right]_{z= a} \\
%                        &=  a^n
%              \end{split}
%              \end{equation*}
%      $\therefore\quad\quad x(n) = a^{n}u(n) $
%    \end{enumerate}
%
%\end{frame}
%%%%%%%%%%%%%%%%%%%%%%%%%%%%%%%%%%%%%%%%%%%%%%%%%%%%%%%%%%%%%%%%%%%%%%%%%%%%%%%%%%%%%%%%%%%%%%%%
%
%
%
%%%%%%%%%%%%%%%%%%%%%%%%%%%%%%%%%%%%%%%%%%%%%%%%%%%%%%%%%%%%%%%%%%%%%%%%%%%%%%%%%%%%%%%%%%%%%%%
%\begin{frame}[allowframebreaks]\frametitle{}%[allowframebreaks][shrink]
%\begin{example}
%设$$X(z)=\frac{1-a^2}{(1-az)(1-\frac{a}{z})},\quad |a|<|z|<\left|\frac{1}{a}\right|,\quad\mbox{求}x(n)$$
%\end{example}
%    \textbf{解:}\par
%    \begin{enumerate}
%      \item $$F(z) = \frac{(1-a^2)\cdot z^n}{-a(z-\frac{1}{a})(z-a)}$$,
%            收敛域为圆环,$|a|<|z|<\left|\frac{1}{a}\right|$,显然$x(n)$ 为双边序列。
%            %$$F(z) = X(z)z^{n-1} = \frac{5z^n}{z^2 + z -6} = \frac{5z^n}{(z-2)(z+3)}$$
%      \item 确定收敛区域,围线$C$,对不同的$n$ 进行讨论
%\begin{figure}
%\centering
%\includegraphics[width=0.3\textwidth]{blankpic.jpg}
%\end{figure}
%      \begin{enumerate}
%        \item $n\geqslant0$,显然C内有极点 $z =a$,有
%              \begin{equation*}
%              \begin{split}
%                   x(n) &= \sum_{k} Res\left[F(z),z = a\right] \\
%                        &= \frac{(1-a^2)\cdot z^n}{-a(z-\frac{1}{a})(z-a)}(z-a)|_{z=a}  =a^n
%              \end{split}
%              \end{equation*}
%        \item 当$n<0$时,$C$内有极点$z=0$,$z=a$,$C$外有极点$z=\frac{1}{a}$,且$z=0$处的极点为高阶极点,这时有:
%              $$\mbox{设:\quad} X(z) = X(z)=\frac{(1-a^2)z}{(1-az)(z-a)}\frac{\rightarrow M=1\mbox{次}}
%                            {\rightarrow N=2\mbox{次}}$$
%              $$\mbox{有:\quad\quad}  n<N-M = 1\mbox{适用于留数辅助定理}$$
%              \begin{equation*}
%              \begin{split}
%                   x(n) &= -\sum Res\left[X(z)z^{n-1}\right]_{\mbox{C外诸极点}}  \\
%                        &= -\frac{(1-a^2)\cdot z^n}{-a(z-\frac{1}{a})(z-a)}(z-\frac{1}{a})|_{z=\frac{1}{a}} \\
%                        &= a^{-n}
%              \end{split}
%              \end{equation*}
%      \end{enumerate}
%      \begin{equation}
%        \therefore  x(n) = \left\{
%        \begin{array}
%                {r@{,\quad}l}
%                a^n\quad  & n \geqslant  0 \\
%                a^{-n}    & n  < 0
%            \end{array} \right.
%      \end{equation}
%      $$\therefore\quad\quad x(n) = a^{|n|}$$
%    \end{enumerate}
%
%\end{frame}
%%%%%%%%%%%%%%%%%%%%%%%%%%%%%%%%%%%%%%%%%%%%%%%%%%%%%%%%%%%%%%%%%%%%%%%%%%%%%%%%%%%%%%%%%%%%%%%%







%\section{序列的Z变换}
%%%%%%%%%%%%%%%%%%%%%%%%%%%%%%%%%%%%%%%%%%%%%%%%%%%%%%%%%%%%%%%%%%%%%%%%%%%%%%%%%%%%%%%%%%%%%%%
%\begin{frame}\frametitle{前言}%[allowframebreaks][shrink]
%
%在模拟信号和系统中,傅里叶变换用于频域分析,拉氏变换用于复频域分析,其实质为傅里叶
%变换的推广。类似,时域离散信号与系统中,Z 变换为序列傅里叶变换的推广,用于对序
%列进行复频域分析。
%
%Z变换在数字信号处理中起着很重要的作用,本节讨论其定义、收敛域、逆$Z$变换,性质等
%四个主要问题。
%
%\end{frame}
%%%%%%%%%%%%%%%%%%%%%%%%%%%%%%%%%%%%%%%%%%%%%%%%%%%%%%%%%%%%%%%%%%%%%%%%%%%%%%%%%%%%%%%%%%%%%%%%
%
%
%\subsection*{Z变换的定义及基本概念}
%
%%%%%%%%%%%%%%%%%%%%%%%%%%%%%%%%%%%%%%%%%%%%%%%%%%%%%%%%%%%%%%%%%%%%%%%%%%%%%%%%%%%%%%%%%%%%%%%
%\begin{frame}\frametitle{$Z$变换的定义}%[allowframebreaks][shrink]
%\begin{definition}
%序列$x(n)$的$Z$变换定义为:
%    \begin{equation} %\label{fol:sft}
%        X(z) = \sum_{n=-\infty}^{\infty}x(n)z^{-n} \quad\quad\mbox{双边Z变换}
%    \end{equation}
%    \begin{equation} %\label{fol:sft}
%         X(z) = \sum_{n=0}^{\infty}x(n)z^{-n}    \quad\quad\mbox{单边Z变换}
%    \end{equation}
%    式中$z$是一个复变量,它所在的平面称为$z$ 平面。对于因果序列,两者都一样。本书均使用双边$Z$变换定义。
%\end{definition}
%\end{frame}
%%%%%%%%%%%%%%%%%%%%%%%%%%%%%%%%%%%%%%%%%%%%%%%%%%%%%%%%%%%%%%%%%%%%%%%%%%%%%%%%%%%%%%%%%%%%%%%%
%
%
%
%%%%%%%%%%%%%%%%%%%%%%%%%%%%%%%%%%%%%%%%%%%%%%%%%%%%%%%%%%%%%%%%%%%%%%%%%%%%%%%%%%%%%%%%%%%%%%%
%\begin{frame}\frametitle{$Z$变换收敛域的概念}%[allowframebreaks][shrink]
%$Z$变换收敛域的概念
%      \begin{enumerate}
%        \item $Z$变换实际上为一个罗朗级数,其存在条件为该级数绝对收敛,也就是满足:
%              $$\sum_{n=-\infty}^{\infty}|x(n)z^{-n}|  <\infty $$
%              满足该条件的$z$值取值范围称为$X(z)$的收敛域。
%        \item 一般的收敛域为环状域,即:$R_{x-}<|z|<R_{x+}$,也就是说,收敛域是以$R_{x-}$和$R_{x+}$
%              为收敛半径的两个圆形成的环状域。$R_{x-}$ 可以小到$0$,$R_{x+}$可以达到无穷大。
%              \par 此处画出收敛域形状图:
%              \newpage\newpage\newpage\newpage\newpage\newpage
%      \end{enumerate}
%\end{frame}
%%%%%%%%%%%%%%%%%%%%%%%%%%%%%%%%%%%%%%%%%%%%%%%%%%%%%%%%%%%%%%%%%%%%%%%%%%%%%%%%%%%%%%%%%%%%%%%%
%
%
%
%%%%%%%%%%%%%%%%%%%%%%%%%%%%%%%%%%%%%%%%%%%%%%%%%%%%%%%%%%%%%%%%%%%%%%%%%%%%%%%%%%%%%%%%%%%%%%%
%\begin{frame}\frametitle{title}%[allowframebreaks][shrink]
%$Z$变换的零极点的概念\par
%        常用的$Z$变换是一个有理函数,可用两个多项式之比表示。
%        $$X(z) = \frac{P(z)}{Q(z)}$$
%        $P(z)=0$的根,称为$X(z)$的零点。\par
%        $Q(z)=0$的根,称为$X(z)$的极点。\par
%        $X(z)$的性质主要取决于极点,在极点处$ZT$不存在。
%\end{frame}
%%%%%%%%%%%%%%%%%%%%%%%%%%%%%%%%%%%%%%%%%%%%%%%%%%%%%%%%%%%%%%%%%%%%%%%%%%%%%%%%%%%%%%%%%%%%%%%%
%
%
%
%%%%%%%%%%%%%%%%%%%%%%%%%%%%%%%%%%%%%%%%%%%%%%%%%%%%%%%%%%%%%%%%%%%%%%%%%%%%%%%%%%%%%%%%%%%%%%%
%\begin{frame}\frametitle{title}%[allowframebreaks][shrink]
%序列$Z$变换与傅里叶变换的关系
%        \begin{equation*}
%          \begin{split}
%          X(e^{j\omega}) &= \sum_{n=-\infty}^{\infty}x(n)e^{-j\omega n}\\
%          X(z)           &= \sum_{n=-\infty}^{\infty}x(n)z^{-n}
%          \end{split}
%        \end{equation*}
%        对比可得:
%        $$z = e^{j\omega}\mbox{,\quad 即:\quad}X(e^{j\omega}) = X(z)|_{z=e^{j\omega}}$$
%        即,单位圆上的$Z$变换就是序列的傅里叶变换。
%        \begin{enumerate}[说明:]\par
%          \item $Z$平面的单位圆一定要在$Z$ 变换的收敛域内。
%          \item 傅里叶变换是$Z$变换的特例。
%        \end{enumerate}
%\end{frame}
%%%%%%%%%%%%%%%%%%%%%%%%%%%%%%%%%%%%%%%%%%%%%%%%%%%%%%%%%%%%%%%%%%%%%%%%%%%%%%%%%%%%%%%%%%%%%%%%
%
%
%
%%%%%%%%%%%%%%%%%%%%%%%%%%%%%%%%%%%%%%%%%%%%%%%%%%%%%%%%%%%%%%%%%%%%%%%%%%%%%%%%%%%%%%%%%%%%%%%
%\begin{frame}\frametitle{title}%[allowframebreaks][shrink]
%\begin{example}
%        求$x(n)=u(n)$的$Z$变换。\par
%        \textbf{解:}
%            $$X(z) = ZT[x(n)] = \sum_{n=-\infty}^{\infty}x(n)z^{- n} = \sum_{n=0}^{\infty}(z^{-1})^{ n} = \frac{1}{1-z^{-1}}$$
%
%            $X(z)$存在的条件是: \quad\quad $|z^{-1}|<1$\quad\quad 即$|z|>1$\par
%            极点为$z=1$,单位圆上$Z$变换不存在,其傅里叶变换也不存在。\par
%            但引入冲激函数后,可得到其傅里叶变换。
%
%        \end{example}
%\end{frame}
%%%%%%%%%%%%%%%%%%%%%%%%%%%%%%%%%%%%%%%%%%%%%%%%%%%%%%%%%%%%%%%%%%%%%%%%%%%%%%%%%%%%%%%%%%%%%%%%
%
%
%\subsection*{Z变换收敛域的讨论}
%%%%%%%%%%%%%%%%%%%%%%%%%%%%%%%%%%%%%%%%%%%%%%%%%%%%%%%%%%%%%%%%%%%%%%%%%%%%%%%%%%%%%%%%%%%%%%%
%\begin{frame}\frametitle{title}%[allowframebreaks][shrink]
%在$x(n)\leftrightarrow X(z)$时,使得$Z$ 变换存在,也就是使得洛朗级数绝对可和的$z$ 变换取值范围,
%为$Z$变换的收敛域。
%
%注意: 收敛域取决于$x(n)$的性质。
%$$X(z)= \sum_{n=-\infty}^{\infty}x(n)z^{-n} = \cdots + x(-1)z+x(0)+x(1)z^{-1}+\cdots $$
%\end{frame}
%%%%%%%%%%%%%%%%%%%%%%%%%%%%%%%%%%%%%%%%%%%%%%%%%%%%%%%%%%%%%%%%%%%%%%%%%%%%%%%%%%%%%%%%%%%%%%%%
%
%
%
%%%%%%%%%%%%%%%%%%%%%%%%%%%%%%%%%%%%%%%%%%%%%%%%%%%%%%%%%%%%%%%%%%%%%%%%%%%%%%%%%%%%%%%%%%%%%%%
%\begin{frame}[allowframebreaks]\frametitle{}%[allowframebreaks][shrink]
%\begin{enumerate}
%  \item \textbf{有限长序列}\par
%      \begin{equation}
%            x(n) = \left\{
%                \begin{array}
%                    {r@{,\quad}l}
%                    x(n)    & \ n_1 \leqslant n \leqslant n_2    \mbox{(不全为0)}\\
%                    0\quad  & \mbox{其他 \quad\quad (全为0)}
%                \end{array} \right.
%      \end{equation}
%      有限项级数求和必定收敛,仅需考虑0 点,及$\infty$ 两点情况。
%      \begin{enumerate}[可分四种情况讨论:]\par
%        \item[$1^0$] $n_2 > n_1\geqslant 0$,\quad\quad 仅存在负幂级数,$0<|z|\leqslant \infty$ (因果序列)
%        \item[$2^0$] $n_1 < n_2\leqslant 0$,\quad\quad 仅存在正幂级数,$0\leqslant|z|< \infty$
%        \item[$3^0$] $n_1<0,n_2 >0$,        \quad 存在正负幂级数,$0<|z|<\infty$
%        \item[$4^0$] $n_1 = 0, n_2 =0$,     \quad 整个$Z$平面。
%      \end{enumerate}
%      \newpage
%      \begin{example}
%      求$x(n)=R_{N}(n)$的$Z$变换.\par
%      \textbf{解:}
%      $$X(z)= \sum_{n=-\infty}^{\infty}R_{N}(n)z^{-n} = \sum_{n=0}^{N-1}(z^{-1})^{n} = \frac{1-z^{-N}}{1-z^{-1}}$$
%      $X(z)$为有限长级数求和,必定收敛,只需考虑$z=0,z=\infty$两点。\par
%      显然有: \quad\quad\quad $0<|z|\leqslant \infty$。
%      \end{example}
%      \newpage
%  \item \textbf{右序列}
%      \begin{equation}
%        x(n) = \left\{
%            \begin{array}
%                {r@{,\quad}l}
%                x(n)    & n \geqslant n_1 \\
%                0\quad  & n \leqslant n_1
%            \end{array} \right.
%      \end{equation}
%      %$$X(z)= \sum_{n=n_1}^{\infty}x(n)z^{-n} = \sum_{n=n_1}^{-1}x(n)z^{-n} + \sum_{n=0}^{\infty}x(n)z^{-n}$$
%      %可以看出,左半为有限长序列,右半为因果序列。
%      右边序列收敛域为Z平面上以原点为圆心的某个园外。
%      \begin{enumerate}[可分两种情况讨论:]\par
%        \item $n_1<0$时,存在正幂级数,$|z|$ 不能为$\infty$,此时有: $R_{x-} <|z| <\infty$
%        \item $n_1\geqslant0$时,为因果序列。有: \quad $R_{x-} <|z| \leqslant\infty$
%      \end{enumerate}
%      %如下图: \newline\newline\newline
%      \newpage
%  \item \textbf{左序列}
%      \begin{equation}
%        x(n) = \left\{
%            \begin{array}
%                {r@{,\quad}l}
%                x(n)    & n \leqslant n_2 \\
%                0\quad  & n > n_2
%            \end{array} \right.
%      \end{equation}
%      %$$X(z)= \sum_{n=-\infty}^{n_2}x(n)z^{-n} $$
%      左边序列收敛域为Z平面上以原点为圆心的某个园内。
%      \begin{enumerate}[可分两种情况讨论:]\par
%        \item $n_2>0$时,收敛区不包括$0$ 点,此时有: \quad $0 <|z| < R_{x+}$
%        \item $n_2\leqslant0$时,有: \quad $0 \leqslant|z| <R_{x+}$
%      \end{enumerate}
%      %如下图: \newline\newline\newline\newline\newline\newline
%     \newpage
%  \item \textbf{双边序列}
%      $$X(z)= \sum_{n=-\infty}^{\infty}x(n)z^{-n} = \sum_{n=-\infty}^{-1}x(n)z^{-n} + \sum_{n=0}^{\infty}x(n)z^{-n}$$
%      可见双边序列由因果序列和反因果序列组成,两者的收敛域分别为:$R_{x-} <|z|$和$|z|<R_{x+}$\par
%      双边序列取两者的交集,收敛域为: \quad$R_{x-} <|z|<R_{x+}\quad $,为一环状区域。\par
%      如无交集,ZT不存在。
%\end{enumerate}
%\end{frame}
%%%%%%%%%%%%%%%%%%%%%%%%%%%%%%%%%%%%%%%%%%%%%%%%%%%%%%%%%%%%%%%%%%%%%%%%%%%%%%%%%%%%%%%%%%%%%%%%
%
%
%
%%%%%%%%%%%%%%%%%%%%%%%%%%%%%%%%%%%%%%%%%%%%%%%%%%%%%%%%%%%%%%%%%%%%%%%%%%%%%%%%%%%%%%%%%%%%%%%
%\begin{frame}[shrink]\frametitle{}%[allowframebreaks][shrink]
%\begin{example}
%      \textbf{求$x(n)=a^{n}u(n)$的$Z$变换及收敛域.(因果序列)}\end{example}
%      \textbf{解:}
%      \begin{equation*}
%        \begin{split}
%            X(z) &= \sum_{n=-\infty}^{\infty}x(n)z^{-n} = \sum_{n=0}^{\infty}\left(\frac{a}{z}\right)^{n}
%                 = \frac{1}{1-\frac{a}{z}} = \frac{z}{z-a}  %\mbox{\quad\quad 收敛域为:}  |z|>|a|
%        \end{split}
%      \end{equation*}
%      $\mbox{\quad\quad 收敛域为:} \quad\quad |z|>|a|$
%  %\end{example}
%  \begin{example}
%      \textbf{求$x(n)=-a^{n}u(-n-1)$的$Z$ 变换及收敛域.(反因果序列)}\end{example}
%
%      \textbf{解:}
%      \begin{equation*}
%      \begin{split}
%          X(z)         &= \sum_{n=-\infty}^{\infty}x(n)z^{-n} = -\sum_{n=-\infty}^{-1}a^{n}z^{-n} \\
%          \mbox{注意}:&\quad  \quad\quad\sum_{n=-\infty}^{-1}x(n)z^{-n} = \sum_{n=1}^{\infty}x(-n)z^{n}\\
%          X(z)         &= -\sum_{n=1}^{\infty}a^{-n}z^{n} = -\frac{\frac{z}{a}}{1-\frac{z}{a}}= \frac{z}{z-a}
%      \end{split}
%      \end{equation*}
%      $\mbox{\quad\quad 收敛域为:}  |z|<|a|$
%  %\end{example}
%\end{frame}
%%%%%%%%%%%%%%%%%%%%%%%%%%%%%%%%%%%%%%%%%%%%%%%%%%%%%%%%%%%%%%%%%%%%%%%%%%%%%%%%%%%%%%%%%%%%%%%%
%
%
%
%%%%%%%%%%%%%%%%%%%%%%%%%%%%%%%%%%%%%%%%%%%%%%%%%%%%%%%%%%%%%%%%%%%%%%%%%%%%%%%%%%%%%%%%%%%%%%%
%\begin{frame}\frametitle{title}%[allowframebreaks][shrink]
%\begin{example}
%      \textbf{求$x(n)=-a^{n}u(-n-1)$的$Z$ 变换及收敛域.(反因果序列)}\newline
%      \textbf{解:}
%      \begin{equation*}
%      \begin{split}
%          X(z)         &= \sum_{n=-\infty}^{\infty}x(n)z^{-n} = -\sum_{n=-\infty}^{-1}a^{n}z^{-n} \\
%          \mbox{注意}:&\quad  \quad\quad\sum_{n=-\infty}^{-1}x(n)z^{-n} = \sum_{n=1}^{\infty}x(-n)z^{n}\\
%          X(z)         &= -\sum_{n=1}^{\infty}a^{-n}z^{n} = -\frac{\frac{z}{a}}{1-\frac{z}{a}}= \frac{z}{z-a}
%      \end{split}
%      \end{equation*}
%      $\mbox{\quad\quad 收敛域为:}  |z|<|a|$
%  \end{example}
%\end{frame}
%%%%%%%%%%%%%%%%%%%%%%%%%%%%%%%%%%%%%%%%%%%%%%%%%%%%%%%%%%%%%%%%%%%%%%%%%%%%%%%%%%%%%%%%%%%%%%%%
%
%
%
%%%%%%%%%%%%%%%%%%%%%%%%%%%%%%%%%%%%%%%%%%%%%%%%%%%%%%%%%%%%%%%%%%%%%%%%%%%%%%%%%%%%%%%%%%%%%%%
%\begin{frame}\frametitle{title}%[allowframebreaks][shrink]
%\textbf{小结:}
%  \begin{equation*}
%     \begin{split}
%          a^{n}u(n)     &\leftrightarrow \frac{z}{z-a} \quad\quad |z|>|a| \mbox{\quad\quad 因果序列}\\
%          -a^{n}u(-n-1) &\leftrightarrow \frac{z}{z-a} \quad\quad |z|<|a| \mbox{\quad\quad 反因果序列}
%     \end{split}
%  \end{equation*}
%  \textbf{说明:} 不同的$x(n)$,其$Z$变换表达式可能相同,所以$Z$变换$X(z)$与收敛域联系在一起才有意义。即:
%  $$X(z)\quad + \quad \mbox{收敛域}\quad\Longleftrightarrow x(n)$$
%\end{frame}
%%%%%%%%%%%%%%%%%%%%%%%%%%%%%%%%%%%%%%%%%%%%%%%%%%%%%%%%%%%%%%%%%%%%%%%%%%%%%%%%%%%%%%%%%%%%%%%%
%
%
%
%%%%%%%%%%%%%%%%%%%%%%%%%%%%%%%%%%%%%%%%%%%%%%%%%%%%%%%%%%%%%%%%%%%%%%%%%%%%%%%%%%%%%%%%%%%%%%%
%\begin{frame}[shrink]\frametitle{}%[allowframebreaks][shrink]
%\begin{example}
%    \textbf{求$x(n)=a^{|n|}$,$a$为实数,求$x(n)$的$Z$ 变换及收敛域.}\par
%      \textbf{解:}
%      \begin{equation*}
%        \begin{split}
%            X(z) &= \sum_{n=-\infty}^{\infty}x(n)z^{-n} = \sum_{n=-\infty}^{\infty}a^{|n|}z^{-n} \\
%                 &= \sum_{n=-\infty}^{-1}a^{-n}z^{-n} + \sum_{n=0}^{\infty}a^{n}z^{-n}
%                 = \sum_{n=1}^{\infty}a^{n}z^{n} + \sum_{n=0}^{\infty}a^{n}z^{-n}\\
%                 &= \frac{az}{1-az} + \frac{1}{1-a z^{-1}} = \frac{1-a^{2}}{(1-az)(1-az^{-1})}
%        \end{split}
%      \end{equation*}
%      $\mbox{\quad\quad 且收敛域需同时满足}  |az|<1, |az^{-1}|<1,\mbox{即}:|a|<|z|<|\frac{1}{a}| (|a|<1)$
%
%      $\mbox{如: \quad\quad} |a|>1,\mbox{\quad 无公共收敛域,则$X(z)$ 不存在。} $
%\end{example}
%\end{frame}
%%%%%%%%%%%%%%%%%%%%%%%%%%%%%%%%%%%%%%%%%%%%%%%%%%%%%%%%%%%%%%%%%%%%%%%%%%%%%%%%%%%%%%%%%%%%%%%%
%\subsection*{逆Z变换}
%
%
%%%%%%%%%%%%%%%%%%%%%%%%%%%%%%%%%%%%%%%%%%%%%%%%%%%%%%%%%%%%%%%%%%%%%%%%%%%%%%%%%%%%%%%%%%%%%%%
%\begin{frame}\frametitle{title}%[allowframebreaks][shrink]
%
%    已知$X(z)$及收敛域,求$x(n)$。
%
%\end{frame}
%%%%%%%%%%%%%%%%%%%%%%%%%%%%%%%%%%%%%%%%%%%%%%%%%%%%%%%%%%%%%%%%%%%%%%%%%%%%%%%%%%%%%%%%%%%%%%%%
%
%\subsubsection*{部分分式法}
%
%
%%%%%%%%%%%%%%%%%%%%%%%%%%%%%%%%%%%%%%%%%%%%%%%%%%%%%%%%%%%%%%%%%%%%%%%%%%%%%%%%%%%%%%%%%%%%%%%
%\begin{frame}\frametitle{部分分式法}%[allowframebreaks][shrink]
%[步骤:]
%\begin{enumerate}
%  \item  分解因式
%        $$\frac{X(z)}{z}=\sum_{m=1}^{N}\frac{A_m}{z-z_m} \mbox{\quad\quad 部分分式和}\quad\quad\quad\quad\quad\quad\quad\quad$$
%        $$\mbox{\quad\quad 其中系数为:} A_m =\left[\frac{X(z)}{z}(z-z_m)\right]_{z=z_m}$$
%
%  \item 套公式。
%\end{enumerate}
%\end{frame}
%%%%%%%%%%%%%%%%%%%%%%%%%%%%%%%%%%%%%%%%%%%%%%%%%%%%%%%%%%%%%%%%%%%%%%%%%%%%%%%%%%%%%%%%%%%%%%%%
%
%
%
%%%%%%%%%%%%%%%%%%%%%%%%%%%%%%%%%%%%%%%%%%%%%%%%%%%%%%%%%%%%%%%%%%%%%%%%%%%%%%%%%%%%%%%%%%%%%%%
%\begin{frame}[allowframebreaks]\frametitle{}%[allowframebreaks][shrink]
%\begin{example}
%    设$$X(z)=\frac{5z^{-1}}{1+z^{-1} -6z^{-2}},\quad\quad 2<|z|<3$$
%\end{example}
%    \textbf{解:}\par
%    收敛域为圆环,显然$x(n)$为双边序列。
%    \begin{equation*}
%     \begin{split}
%            \frac{X(z)}{z} &= \frac{5z^{-2}}{1+z^{-1} -6z^{-2}} =\frac{5}{z^2 + z -6}= \frac{5}{(z+3)(z-2)}\\
%                           &= \frac{A_1}{z-2} + \frac{A_2}{z+3}
%      \end{split}
%    \end{equation*}
%    \begin{equation*}
%        \begin{split}
%            A_1   &= \left[\frac{X(z)}{z}(z-2)\right]_{z =2 } =\frac{5}{z+3}|_{z=2} =1\\
%            A_2   &= \left[\frac{X(z)}{z}(z+3)\right]_{z =-3 } =\frac{5}{z-2}|_{z=-3} =-1
%        \end{split}
%    \end{equation*}
%    $$\frac{X(z)}{z} = \frac{1}{z-2} - \frac{1}{z+3} \quad\quad \Longrightarrow\quad\quad X(z) = \frac{z}{z-2} - \frac{z}{z+3}$$
%    我们知道:
%    \begin{equation}
%        \left\{ \begin{aligned}
%            a^{n}u(n)     &\leftrightarrow \frac{z}{z-a} \quad\quad |z|>|a| \mbox{\quad\quad 因果序列}\\
%          -a^{n}u(-n-1)   &\leftrightarrow \frac{z}{z-a} \quad\quad |z|<|a| \mbox{\quad\quad 反因果序列}
%        \end{aligned} \right.
%    \end{equation}
%    套公式:
%    \begin{equation*}
%       \begin{split}
%          x(n)    &=  2^{n}u(n) - (-(-3)^{n}u(-n-1))\\
%                  &=  2^{n}u(n) +(-3)^{n}u(-n-1)
%       \end{split}
%    \end{equation*}
%
%
%
%\end{frame}
%%%%%%%%%%%%%%%%%%%%%%%%%%%%%%%%%%%%%%%%%%%%%%%%%%%%%%%%%%%%%%%%%%%%%%%%%%%%%%%%%%%%%%%%%%%%%%%%
%
%
%\subsubsection*{留数法(围线积分法)}
%%%%%%%%%%%%%%%%%%%%%%%%%%%%%%%%%%%%%%%%%%%%%%%%%%%%%%%%%%%%%%%%%%%%%%%%%%%%%%%%%%%%%%%%%%%%%%%
%\begin{frame}[allowframebreaks]\frametitle{}%[allowframebreaks][shrink]
%\begin{enumerate}
%  \item [1] 逆Z变换:
%          $$X(z) = \sum_{n=-\infty}^{\infty}x(n)z^{-n}\quad\quad R_{x-} <|z|<R_{x+}$$
%          则有:
%          $$x(n) = \frac{1}{2\pi j}\oint_{c} X(z) z^{n-1}dz \quad\quad\quad\quad\quad\quad$$
%  \par   \textbf{证明:}此处引入柯西积分定理(引入,不做证明)
%        \begin{equation}
%            \frac{1}{2\pi j}\oint_{c} z^{m-1}dz = \left\{
%            \begin{array}
%                {r@{,\quad}l}
%                1    & m   =  0 \\
%                0    & m \neq 0
%            \end{array} \right.
%        \end{equation}
%        这里$C$为一个逆时针封闭曲线,那么:
%        \begin{equation*}
%        \begin{split}
%            \frac{1}{2\pi j}\oint_{c} X(z) z^{k-1}dz
%                 &= \frac{1}{2\pi j}\oint_{c} \left[ \sum_{n=-\infty}^{\infty}x(n)z^{-n} z^{k-1}\right]dz \\
%                 &= \frac{1}{2\pi j}\oint_{c} \left[  \sum_{n=-\infty}^{\infty}x(n)z^{-n+k-1}\right]dz \\
%                 &= \sum_{n=-\infty}^{\infty}x(n)\left[\frac{1}{2\pi j}\oint_{c}  z^{-n+k-1}dz\right]   %        = x(k)
%        \end{split}
%    \end{equation*}
%    显然,仅当$k=n$时,有:\vspace{-0.3cm}
%    $$x(k) = \frac{1}{2\pi j}\oint_{c} X(z) z^{k-1}dz $$
%    交换符号 $k \rightarrow n$,有:\vspace{-0.3cm}
%    $$x(n) = \frac{1}{2\pi j}\oint_{c} X(z) z^{n-1}dz $$
%
%  \end{enumerate}
%
%\end{frame}
%%%%%%%%%%%%%%%%%%%%%%%%%%%%%%%%%%%%%%%%%%%%%%%%%%%%%%%%%%%%%%%%%%%%%%%%%%%%%%%%%%%%%%%%%%%%%%%%
%
%
%
%%%%%%%%%%%%%%%%%%%%%%%%%%%%%%%%%%%%%%%%%%%%%%%%%%%%%%%%%%%%%%%%%%%%%%%%%%%%%%%%%%%%%%%%%%%%%%%
%\begin{frame}\frametitle{}%[allowframebreaks][shrink]
%\begin{enumerate}
%\item [2]  留数定理
%\end{enumerate}
%       \par 直接计算围线积分非常麻烦,可利用留数定理得到。
%       \begin{enumerate}
%         \item [a] 设 $F(z) = X(z)\cdot z^{n-1}$,则有:
%               $$x(n) = \frac{1}{2\pi j}\oint_{c} F(z)dz $$
%               如$F(z)$在围线内的极点为$z_k$,则根据留数定理,有:
%               $$\frac{1}{2\pi j}\oint_{c} F(z)dz = \sum_{k}Res\left[F(z),z_k\right] =
%                \sum Res\left[F(z)\right]_{\mbox{C 内诸极点}} $$\vspace{-0.3cm}
%               $$ x(n) = \sum_{k}Res\left[F(z),z_k\right]  \quad\quad\quad\quad\quad\quad$$
%               $\mbox{即:\quad\quad}x(n)$是围线$C$ 内所有极点的留数之和。
%         \item [b] 留数的求法:\par
%               \begin{enumerate}
%                 \item 对于单阶极点:$z=z_0$,有
%                       $$Res\left[X(z)\cdot z^{n-1},z_0\right] = X(z)\cdot z^{n-1}(z-z_0)|_{z=z_0}$$
%                 \item 对于m阶高阶极点:$z=z_0$,有
%                       $$Res\left[X(z)\cdot z^{n-1},z_0\right] =
%                           \frac{1}{(m-1)!}\frac{d^{m-1}}{dz^{m-1}}\left[X(z)\cdot z^{n-1}(z-z_0)^{m}\right]_{z=z_0}$$
%               \end{enumerate}
%         \item [c] 留数辅助定理\par
%               \textbf{对于高阶极点,一般难于求解,往往利用留数辅助定理。}\par
%               设被积函数$F(z)=X(z)\cdot z^{n-1}$是有理函数,分母的最高阶次大于等于分子最高阶次$2$次以上,则有:
%               $$\oint_{C\rightarrow \infty}F(z)dz = 0$$
%               $$\therefore\quad\quad\quad \frac{1}{2\pi j}\oint_{C\rightarrow \infty}F(z)dz = 0$$
%               即在整个$Z$平面内,该围线积分积分结果为$0$。
%
%               如果在$X(z)$收敛域内选取一个围线$C$,显然有:
%               $$\sum Res\left[X(z)z^{n-1}\right]_{\mbox{C内诸极点}} +
%                               \sum Res\left[X(z)z^{n-1}\right]_{\mbox{C外诸极点}} =0$$
%               在用留数法求$x(n)$时,如果围线$C$内存在高阶极点,可通过求围线外的一阶极点的留数,再取反即可。\par
%               \textbf{成立条件:}
%               \begin{itemize}
%                 \item 留数辅助定理成立的条件是:$F(z) = X(z)\cdot z^{n-1}$的分母阶次大于等于分子阶次$2$ 次以上。
%                     $$\mbox{设:\quad} X(z) = \frac{P(z)}{Q(z)}\frac{\rightarrow M\mbox{次}}
%                            {\rightarrow N\mbox{次}}  \Longrightarrow  N-M-n+1\geqslant 2  $$
%                     $$\mbox{即:\quad} n< N-M \quad\quad \mbox{注意:此即为判断依据}$$
%               \end{itemize}
%       \end{enumerate}
%\end{frame}
%%%%%%%%%%%%%%%%%%%%%%%%%%%%%%%%%%%%%%%%%%%%%%%%%%%%%%%%%%%%%%%%%%%%%%%%%%%%%%%%%%%%%%%%%%%%%%%%
%
%
%\subsection*{Z变换的性质和定理}
%
%%%%%%%%%%%%%%%%%%%%%%%%%%%%%%%%%%%%%%%%%%%%%%%%%%%%%%%%%%%%%%%%%%%%%%%%%%%%%%%%%%%%%%%%%%%%%%%
%\begin{frame}\frametitle{title}%[allowframebreaks][shrink]
%\begin{enumerate}
%  \item 线性(不讲,但注意收敛域的变化)
%  \item 移位 (非常重要)
%        $$\mbox{设:\quad} x(n)\leftrightarrow X(z),\quad\quad \mbox{则:}x(n-n_{0})\leftrightarrow z^{-n_0}X(z)$$
%        证:
%            \begin{equation*}
%            \begin{split}
%               ZT[x(n)]    &= \sum_{n=-\infty}^{\infty}x(n)z^{-n}\\
%              ZT[x(n-n_0)] &= \sum_{n=-\infty}^{\infty}x(n-n_0)z^{-n}\mbox{\quad\quad 令:}m= n-n_0\\
%                           &= \sum_{m=-\infty}^{\infty}x(m)z^{-(m+n_0)}  = z^{-n_0}\sum_{m=-\infty}^{\infty}x(m)z^{-m}\\
%                           &= z^{-n_0}X(z)
%            \end{split}
%            \end{equation*}
%            \end{enumerate}
%\end{frame}
%%%%%%%%%%%%%%%%%%%%%%%%%%%%%%%%%%%%%%%%%%%%%%%%%%%%%%%%%%%%%%%%%%%%%%%%%%%%%%%%%%%%%%%%%%%%%%%%
%
%
%
%%%%%%%%%%%%%%%%%%%%%%%%%%%%%%%%%%%%%%%%%%%%%%%%%%%%%%%%%%%%%%%%%%%%%%%%%%%%%%%%%%%%%%%%%%%%%%%
%\begin{frame}\frametitle{序列乘以$n$的$ZT$}%[allowframebreaks][shrink]
%序列乘以$n$的$ZT$
%        $$\mbox{设:\quad} x(n)\leftrightarrow X(z),\quad\quad \mbox{则:}nx(n)\leftrightarrow -z\frac{dX(z)}{dz}$$
%        证:从右到左:
%            \begin{equation*}
%            \begin{split}
%               X(z)            &= \sum_{n=-\infty}^{\infty}x(n)z^{-n}\\
%              \frac{dX(z)}{dz} &= -\sum_{n=-\infty}^{\infty}x(n) n z^{-n-1} \\
%                               &= -z^{-1}\sum_{n=-\infty}^{\infty}nx(n) z^{-n} \\
%                               &= -z^{-1}ZT[nx(n)]
%            \end{split}
%            \end{equation*}
%            $$\therefore \quad ZT[nx(n)] = -z\frac{dX(z)}{dz}$$
%\end{frame}
%%%%%%%%%%%%%%%%%%%%%%%%%%%%%%%%%%%%%%%%%%%%%%%%%%%%%%%%%%%%%%%%%%%%%%%%%%%%%%%%%%%%%%%%%%%%%%%%
%
%
%
%%%%%%%%%%%%%%%%%%%%%%%%%%%%%%%%%%%%%%%%%%%%%%%%%%%%%%%%%%%%%%%%%%%%%%%%%%%%%%%%%%%%%%%%%%%%%%%
%\begin{frame}\frametitle{初值定理}%[allowframebreaks][shrink]
%
%        \par 设$x(n)$是因果序列,$x(n)\leftrightarrow X(z)$,则有 $$x(0) = \lim_{z\rightarrow \infty} X(z)$$
%        证明:
%            \begin{equation*}
%            \begin{split}
%               X(z)         &= \sum_{n=0}^{\infty}x(n)z^{-n}= x(0) + x(1)z^{-1}+x(2)z^{-2}+\cdots\\
%               \therefore\  x(0) &= \lim_{z\rightarrow \infty} X(z)
%            \end{split}
%            \end{equation*}
%\end{frame}
%%%%%%%%%%%%%%%%%%%%%%%%%%%%%%%%%%%%%%%%%%%%%%%%%%%%%%%%%%%%%%%%%%%%%%%%%%%%%%%%%%%%%%%%%%%%%%%%
%
%
%
%%%%%%%%%%%%%%%%%%%%%%%%%%%%%%%%%%%%%%%%%%%%%%%%%%%%%%%%%%%%%%%%%%%%%%%%%%%%%%%%%%%%%%%%%%%%%%%
%\begin{frame}\frametitle{终值定理}%[allowframebreaks][shrink]
%
%        \quad 设$x(n)$是因果序列,其$Z$ 变换的极点,除可以有一个一阶极点在$z=1$ 上,其他极点都在单位园内,则:
%        $$\lim_{n\rightarrow \infty}x(n) = \lim_{z\rightarrow 1}(z-1)X(z)$$
%        证明:
%        \begin{equation*}
%        \begin{split}
%            (z-1)X(z) &= zX(z) -X(z) \\
%                      &= ZT[x(n+1)] - ZT[x(n)]\quad\quad\mbox{根据移位性质}\\
%                      &= \sum_{n=-\infty}^{\infty}[x(n+1)-x(n)]z^{-n}  \\
%                      &  \because  x(n)\quad\mbox{因果} \therefore x(n)=0,n<0  \\
%            (z-1)X(z) &= \lim_{n\rightarrow \infty}\left[\sum_{m=-1}^{n}x(m+1)z^{-m}- \sum_{m=0}^{n}x(m)z^{-m}\right]\\
%                      &  \because (z-1)X(z)\mbox{在单位圆上无极点,上式两端对$z=1$取极限}\\
%            \lim_{z\rightarrow 1}(z-1)X(z)
%                      &= \lim_{n\rightarrow \infty}\left[\sum_{m=-1}^{n}x(m+1)- \sum_{m=0}^{n}x(m)\right]\\
%                      &= \lim_{n\rightarrow \infty} \left[(x(0)+x(1)+\cdots +x(n+1))-(x(0)+x(1) + \cdots +x(n))\right] \\
%                      &= \lim_{n\rightarrow \infty}x(n+1)\\
%                      &= \lim_{n\rightarrow \infty}x(n)
%        \end{split}
%        \end{equation*}
%        如$X(z)$在单位圆上没有极点,则$x(\infty)=0$
%\end{frame}
%%%%%%%%%%%%%%%%%%%%%%%%%%%%%%%%%%%%%%%%%%%%%%%%%%%%%%%%%%%%%%%%%%%%%%%%%%%%%%%%%%%%%%%%%%%%%%%%
%
%
%
%%%%%%%%%%%%%%%%%%%%%%%%%%%%%%%%%%%%%%%%%%%%%%%%%%%%%%%%%%%%%%%%%%%%%%%%%%%%%%%%%%%%%%%%%%%%%%%
%\begin{frame}\frametitle{时域卷积定理}%[allowframebreaks][shrink]
%
%设\quad $X(n)\leftrightarrow X(z), \quad\quad h(n)\leftrightarrow H(z),\quad\quad y(n)\leftrightarrow Y(z)$\par
%且\quad $y(n) = x(n)*h(n)$,则有:\quad\quad $Y(z) = X(z)\cdot H(z)$
%
%证明:
%$$ y(n)      = x(n)*h(n) = \sum_{k=-\infty}^{\infty}x(k)h(n-k) $$
%\begin{equation*}
%\begin{split}
%    %y(n)      &= x(n)*h(n) = \sum_{k=-\infty}^{\infty}x(k)h(n-k) \\
%    Y(z)      &= \sum_{n=-\infty}^{\infty}y(n)z^{-n}
%               =\sum_{n=-\infty}^{\infty}\left[\sum_{k=-\infty}^{\infty}x(k)h(n-k)\right]z^{-n} \\
%              &= \sum_{k=-\infty}^{\infty}x(k)\sum_{n=-\infty}^{\infty}h(n-k)z^{-n}
%              = \sum_{k=-\infty}^{\infty}x(k)\cdot z^{-k}\cdot H(z) \\
%              &= H(z) \sum_{k=-\infty}^{\infty}x(k)\cdot z^{-k}  = X(z)\cdot H(z)
%\end{split}
%\end{equation*}
%
%\end{frame}
%%%%%%%%%%%%%%%%%%%%%%%%%%%%%%%%%%%%%%%%%%%%%%%%%%%%%%%%%%%%%%%%%%%%%%%%%%%%%%%%%%%%%%%%%%%%%%%


\section{2.6 利用Z变换分析信号和系统的频域特性}
\subsection*{传输函数与系统函数}
%%%%%%%%%%%%%%%%%%%%%%%%%%%%%%%%%%%%%%%%%%%%%%%%%%%%%%%%%%%%%%%%%%%%%%%%%%%%%%%%%%%%%%%%%%%%%%
\begin{frame}[shrink]\frametitle{传输函数与系统函数}%[allowframebreaks][shrink]
\begin{enumerate}
\item 传输函数

\par $h(n)$为系统的单位脉冲响应,即为系统对单位脉冲序列$\delta(n)$ 的零状态响应:
\begin{equation*}
\begin{split}
\mbox{令:}\quad H(e^{j\omega}) &= FT[h(n)] = \sum_{n=-\infty}^{\infty}h(n)e^{-j\omega n}\\
&= |H(e^{j\omega})|\cdot e^{j\varphi(\omega)}
\end{split}
\end{equation*}

这里$|H(e^{j\omega})|$称为幅频特性函数,$\varphi(\omega)$称为相频特性函数。

\item 系统函数
$$h(n) \leftrightarrow H(z)\quad\quad : H(z)\mbox{称为系统函数}$$
%小结:
%        \newline \newline \newline
\end{enumerate}
\end{frame}
%\begin{frame}[shrink]\frametitle{系统函数$H(z)$与差分方程}
% 系统函数$H(z)$与差分方程
%        $$\mbox{例:\quad 已知}\sum_{k=0}^{N}a_k y(n-k) = \sum_{k=0}^{M}b_k x(n-k),\mbox{求其对应的}H(z).$$
%        对方程两边进行Z变换,可得:
%        \begin{equation*}
%        \begin{split}
%          ZT\left[\sum_{k=0}^{N}a_k y(n-k)\right] &= ZT\left[\sum_{k=0}^{M}b_k x(n-k)\right]\\
%            \mbox{设:}  x(n)\leftrightarrow X(z)       &   \quad \quad y(n)\leftrightarrow Y(z)  \\
%             \sum_{k=0}^{N}a_k z^{-k} Y(z)        &=  \sum_{k=0}^{M} b_k z^{-k} X(z)\\
%             H(z) = \frac{Y(z)}{X(z)} &= \frac{\sum_{k=0}^{M} b_k z^{-k}}{\sum_{k=0}^{N}a_k z^{-k}}
%          \end{split}
%        \end{equation*}
%
%\end{frame}
%%%%%%%%%%%%%%%%%%%%%%%%%%%%%%%%%%%%%%%%%%%%%%%%%%%%%%%%%%%%%%%%%%%%%%%%%%%%%%%%%%%%%%%%%%%%%%%
\begin{frame}[shrink]\frametitle{系统函数$H(z)$与差分方程}%[allowframebreaks][shrink]

$$\mbox{已知}\sum_{k=0}^{N}a_k y(n-k) = \sum_{k=0}^{M}b_k x(n-k),\mbox{求其对应的}H(z).$$
对方程两边进行Z变换, 设:$x(n)\leftrightarrow X(z)$,$y(n)\leftrightarrow Y(z)$, 可得:
\begin{equation*}
\begin{split}
ZT\left[\sum_{k=0}^{N}a_k y(n-k)\right] &= ZT\left[\sum_{k=0}^{M}b_k x(n-k)\right]\\
\sum_{k=0}^{N}a_k z^{-k} Y(z)        &=  \sum_{k=0}^{M} b_k z^{-k} X(z)\\
H(z) = \frac{Y(z)}{X(z)} &= \frac{\sum_{k=0}^{M} b_k z^{-k}}{\sum_{k=0}^{N}a_k z^{-k}}
\end{split}
\end{equation*}

\end{frame}


%%%%%%%%%%%%%%%%%%%%%%%%%%%%%%%%%%%%%%%%%%%%%%%%%%%%%%%%%%%%%%%%%%%%%%%%%%%%%%%%%%%%%%%%%%%%%%
\begin{frame}[allowframebreaks]\frametitle{系统对于单频复指数信号$e^{j\omega n}$ 的响应}%[allowframebreaks][shrink]
%\begin{enumerate}
%  \item [(a)] 系统对于单频复指数信号$e^{j\omega n}$ 的响应\par
%\end{enumerate}


如系统输入信号为$x(n)=e^{j\omega n}$ ,则输出信号为:
$$y(n) = x(n)*h(n) = \sum_{m=-\infty}^{\infty}h(m)x(n-m)=\sum_{m=-\infty}^{\infty}h(m)e^{j\omega (n-m)}$$
$$ = e^{j\omega n}\sum_{m=-\infty}^{\infty}h(m)e^{-j\omega m} =H(e^{j\omega})e^{j\omega n}\quad
\quad\quad\quad\quad\quad $$
$$\mbox{即:\quad} y(n) = H(e^{j\omega})e^{j\omega n} = |H(e^{j\omega})|e^{j(\omega n + \varphi(\omega))}  \quad
\quad\quad\quad\quad\quad $$


% $$\mbox{即:\quad} y(n) = H(e^{j\omega})e^{j\omega n} = |H(e^{j\omega})|e^{j(\omega n + \varphi(\omega))}  \quad
%         \quad\quad\quad\quad\quad $$

可见,系统的输入为单频复指数函数$e^{j\omega n}$时,输出仍为单频复指数函数,只不过幅度放大$|H(e^{j\omega})|$ 倍,相移$\varphi(\omega)$。
\newpage
\textbf{ 所以,$H(e^{j\omega})$表示系统对特征序列$e^{j\omega n}$的响应特性,这也是$H(e^{j\omega})$的物理意义。}
\end{frame}


\begin{frame}[shrink]\frametitle{系统对正弦信号$x(n)=cos(\omega n)$ 的响应}%[allowframebreaks][shrink]
%\begin{enumerate}
%  \item [(b)] 系统对正弦信号$x(n)=cos(\omega n)$ 的响应 %\par
%\end{enumerate}
\begin{equation*}
\begin{split}
x(n)      &= cos(\omega n) = \frac{1}{2}(e^{j\omega n}+e^{-j\omega n}) \\
y(n)      &= x(n)*h(n)\\
&= \frac{1}{2}\left(H(e^{j\omega})e^{j\omega n}+ H(e^{j(-\omega)})e^{-j\omega n}\right)\\
&= \frac{1}{2}\left(|H(e^{j\omega})|e^{j\varphi(\omega)}e^{j\omega n}+|H(e^{-j\omega})|e^{j\varphi(-\omega)}e^{-j\omega n}\right) \\
&\quad  \quad\quad\quad\quad\mbox{设}  h(n)\in R\quad H^{*}(e^{j\omega})=H(e^{-j\omega}) \\
&\quad  \quad\quad\therefore |H(e^{j\omega})| = |H(e^{-j\omega})|\quad \varphi(-\omega) = -\varphi(\omega) \\
&= \frac{1}{2}|H(e^{j\omega})|\left(e^{j(\varphi(\omega)+\omega n)}+e^{-j(\varphi(\omega)+\omega n)}\right)\\
&= |H(e^{j\omega})|cos(\omega n +\varphi(\omega))
\end{split}
\end{equation*}
\end{frame}


\begin{frame}[shrink]\frametitle{系统对正弦信号$x(n)=cos(\omega n)$ 的响应}%[allowframebreaks][shrink]
对比:
\begin{equation*}
\begin{split}
x(n)      &= cos(\omega n) = \frac{1}{2}(e^{j\omega n}+e^{-j\omega n}) \\
y(n)
&= \frac{1}{2}|H(e^{j\omega})|\left(e^{j(\varphi(\omega)+\omega n)}+e^{-j(\varphi(\omega)+\omega n)}\right)\\
&= |H(e^{j\omega})|cos(\omega n +\varphi(\omega))
\end{split}
\end{equation*}

可见,线性时不变系统对单频正弦信号$cos(\omega n)$的响应为同频正弦信号,其幅度放大$|H(e^{j\omega n})|$ 倍,相移增加$\varphi(\omega)$

\begin{shuoming}
对于一般序列x(n),可通过傅立叶变换分解为一系列正弦函数的加权和。此时可通过$H(e^{j\omega n})$ 对不同的频率成分进行加权处理。
\end{shuoming}
\end{frame}
%%%%%%%%%%%%%%%%%%%%%%%%%%%%%%%%%%%%%%%%%%%%%%%%%%%%%%%%%%%%%%%%%%%%%%%%%%%%%%%%%%%%%%%%%%%%%%%


\subsection*{系统的因果性和稳定性}
%%%%%%%%%%%%%%%%%%%%%%%%%%%%%%%%%%%%%%%%%%%%%%%%%%%%%%%%%%%%%%%%%%%%%%%%%%%%%%%%%%%%%%%%%%%%%%
\begin{frame}[shrink]\frametitle{因果系统的收敛域特点}%[allowframebreaks][shrink]
%\begin{enumerate}
%\item 因果系统的收敛域特点
\begin{enumerate}
\item 回顾:
$$\mbox{系统因果}\Longleftrightarrow h(n)=0,n<0\quad\mbox{(第一章讨论过)}$$
\item 从Z变换的角度看,则有
\begin{equation*}
\begin{split}
\mbox{系统因果} &\longleftrightarrow   \mbox{$H(z)$的收敛域包含$\infty$点,或$|z|>R_{x^{-}}$}\\
&\quad\quad\quad       \mbox{即极点在某个圆内,收敛域在圆外}
\end{split}
\end{equation*}
\end{enumerate}
\end{frame}


\begin{frame}[allowframebreaks]\frametitle{证明}%[allowframebreaks][shrink]
%    \textbf{证明:} \par
1 充分性 \quad($\Longleftarrow$),设$H(z)$在$z=\infty$ 处收敛,往证系统因果。
\begin{equation}
\begin{split}
H(z) &=  \sum_{n=-\infty}^{\infty}h(n)z^{-n}\\
&=  \underbrace{\sum_{n=-\infty}^{-1}h(n)z^{-n}}_{\mbox{正项级数}} \quad
+ \quad\underbrace{\sum_{n=0}^{\infty}h(n)z^{-n}}_{\mbox{负项级数}}
\end{split}
\end{equation}
如$z=\infty$处收敛,则必有$n\leqslant -1$时,$h(n)=0$  \par
即: $n<0$时,$h(n)=0$,所以系统因果。
\newpage
2 必要性 \quad($\Longrightarrow$),设系统因果,往证$H(z)$在$z=\infty$ 处收敛。
\par 因系统因果,则有$h(n)=0,n<0$,
$$H(z)=\sum_{n=0}^{\infty}h(n)z^{-n}= h(0)+h(1)z^{-1}+h(2)z^{-2}+h(3)z^{-3}+\cdots$$
显然,在$z=\infty$处,$H(z)$ 收敛。

\end{frame}





\begin{frame}[allowframebreaks]\frametitle{系统稳定的收敛域的特点}%[allowframebreaks][shrink]
%\item 系统稳定的收敛域的特点

\begin{enumerate}
\item   回顾:
$$\mbox{系统稳定}\longleftrightarrow  \sum_{n=-\infty}^{\infty}|h(n)|<\infty
\quad\mbox{也是$H(e^{j\omega})$存在的条件}$$
\item 从Z变换的角度看,则有
$$\mbox{系统稳定}\Longleftrightarrow  H(z)\mbox{的收敛域包含单位圆}$$

\end{enumerate}
\end{frame}



\begin{frame}[shrink]\frametitle{证明}%[allowframebreaks][shrink]

%            \textbf{证明:} \par
1 充分性 \quad($\Longleftarrow$),已知$H(z)$收敛域包含单位圆,往证系统因果。
\begin{equation*}
\begin{split}
H(z) &=  \sum_{n=-\infty}^{\infty}h(n)z^{-n}\\
&   \mbox{令$z=1$,则有} \\
H(1) &=  \sum_{n=-\infty}^{\infty}h(n)  < \infty\\
\therefore
y(n) &= h(n)*x(n) = \sum_{k=-\infty}^{\infty}h(k)x(n-k)\\
& \because x(n)\mbox{为有界输入},\quad\therefore |x(n)|<B \\
\therefore
y(n) &= h(n)*x(n) = \sum_{k=-\infty}^{\infty}h(k)x(n-k)\leqslant B\sum_{k=-\infty}^{\infty}h(k)<\infty\\
&  \therefore \mbox{系统稳定}
\end{split}
\end{equation*}
\end{frame}

\begin{frame}[shrink]\frametitle{证明}%[allowframebreaks][shrink]
2 必要性 \quad($\Longrightarrow$),已知系统稳定,往证$H(z)$收敛域包含单位圆。
\begin{equation*}
\begin{split}
\sum_{n=-\infty}^{\infty}|h(n)|
&= \sum_{n=-\infty}^{\infty}|h(n)|\cdot|e^{-j\omega n}|\quad\quad(\mbox{显然$|e^{-j\omega n}|=1$})\\
&\geqslant \left|\sum_{n=-\infty}^{\infty}h(n)e^{-j\omega n}\right| =
\left|\left[\sum_{n=-\infty}^{\infty}h(n)z^{-n}\right]\right|_{z=e^{j\omega}}\\
&= \left|H(z)\right|_{z=e^{j\omega}}
\end{split}
\end{equation*}
$$\because \quad\sum_{n=-\infty}^{\infty}|h(n)| <\infty
\quad\quad\quad\therefore \left|H(z)|_{z=e^{j\omega}}\right|<\infty$$
$H(z)$在单位圆上收敛,即$|z|=1$ 是收敛域的一部分。
\end{frame}





\begin{frame}[shrink]\frametitle{因果稳定系统的收敛域、极点分布特点}%[allowframebreaks][shrink]
%\item 因果稳定系统的收敛域、极点分布特点
\begin{equation*}
\begin{split}
\mbox{系统因果} &\Rightarrow  r<|z| \quad \mbox{收敛域在某个圆外(或收敛域包含无穷远点)}\\
\mbox{系统稳定} &\Rightarrow    0 <r<1,     \quad \mbox{收敛域包含单位圆}
\end{split}
\end{equation*}

系统因果稳定,将等价于下面两个条件
\begin{enumerate}
\item [(1)] $|z|>r\quad\mbox{且} 0 <r<1 $ 。
\item [(2)] $H(z)$的极点全部在单位圆内部。
\end{enumerate}

%\end{enumerate}
\end{frame}
%%%%%%%%%%%%%%%%%%%%%%%%%%%%%%%%%%%%%%%%%%%%%%%%%%%%%%%%%%%%%%%%%%%%%%%%%%%%%%%%%%%%%%%%%%%%%%%






%%%%%%%%%%%%%%%%%%%%%%%%%%%%%%%%%%%%%%%%%%%%%%%%%%%%%%%%%%%%%%%%%%%%%%%%%%%%%%%%%%%%%%%%%%%%%%
%\begin{frame}\frametitle{title}%[allowframebreaks][shrink]
%eq:1
%\end{frame}
%%%%%%%%%%%%%%%%%%%%%%%%%%%%%%%%%%%%%%%%%%%%%%%%%%%%%%%%%%%%%%%%%%%%%%%%%%%%%%%%%%%%%%%%%%%%%%%


\end{document}


