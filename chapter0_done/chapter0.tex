\documentclass[notheorems,compress,mathserif,table]{beamer}

\useoutertheme{tree}
\usecolortheme{whale}      % Outer color themes, 其他选择: whale, seahorse, dolphin . 换一个编译看看有什么不同.
\usecolortheme{orchid}     % Inner color themes, 其他选择: lily, orchid
\useinnertheme[shadow]{rounded} % 对 box 的设置: 圆角、有阴影.
\setbeamercolor{sidebar}{bg=blue!50} % sidebar的颜色, 50%的蓝色.
%\setbeamercolor{background canvas}{bg=blue!9} % 背景色, 9%的蓝色. 去掉下一行, 试一试这个.
\setbeamertemplate{background canvas}[vertical shading][bottom=white,top=structure.fg!25] %%背景色, 上25%的蓝, 过渡到下白.
\usefonttheme{serif}  % 字体. 个人偏好有轮廓的字体. 去掉这个设置编译, 就看到不同了.
\setbeamertemplate{navigation symbols}{}   %% 去掉页面下方默认的导航条.
%%------------------------常用宏包---------------------------------------------------------------------
%%注意, beamer 会默认使用下列宏包: amsthm, graphicx, hyperref, color, xcolor, 等等
%\usepackage{CJK}
\usepackage{ctex}
\usepackage{amsmath,amsthm,amsfonts,amssymb,bm}
\usepackage{mathrsfs}
\usepackage{subfigure} %%图形或表格并排排列
\usepackage{xmpmulti}  %%支持文中的 \multiinclude 等命令, 使 mp 文件逐帧出现. 具体讨论见 beamer 手册.
\usepackage{colortbl,dcolumn}     %% 彩色表格
%\logo{\includegraphics[height=0.09\textwidth]{ajln.jpg}}   %左上角科大logo
%%%%%%%%%%%%%%%%%%%%%%%%%%%%%%%%%%%%%%重定义字体、字号命令 %%%%%%%%%%%%%%%%%%%%%%%%%%%%%%%%%%%%%%%%%%%%%%
%\newcommand{\songti}{\CJKfamily{song}}        % 宋体
%\newcommand{\fangsong}{\CJKfamily{fs}}        % 仿宋体
%\newcommand{\kaishu}{\CJKfamily{kai}}         % 楷体
%\newcommand{\heiti}{\CJKfamily{hei}}          % 黑体
%\newcommand{\lishu}{\CJKfamily{li}}           % 隶书
\newcommand{\youyuang}{\CJKfamily{you}}       % 幼圆
\newcommand{\sanhao}{\fontsize{16pt}{\baselineskip}\selectfont}     % 字号设置
\newcommand{\sihao}{\fontsize{14pt}{\baselineskip}\selectfont}      % 字号设置
\newcommand{\xiaosihao}{\fontsize{12pt}{\baselineskip}\selectfont}  % 字号设置
\newcommand{\wuhao}{\fontsize{10.5pt}{\baselineskip}\selectfont}    % 字号设置
\newcommand{\xiaowuhao}{\fontsize{9pt}{\baselineskip}\selectfont}   % 字号设置
\newcommand{\liuhao}{\fontsize{7.875pt}{\baselineskip}\selectfont}  % 字号设置
\newcommand{\qihao}{\fontsize{5.25pt}{\baselineskip}\selectfont}    % 字号设置
%%%%%%%%%%%%%%%%%%%%%%%%%%%%%%%%%%%%%%%%%%%%%%%%%%%%%%%%%%%%%%%%%%%%%%%%%%%%%%%%%%%%%%%%%%%%%%%%%%%%%%%%
%%----------------------- Theorems ---------------------------------------------------------------------
\newtheorem{theorem}{定理}
\newtheorem{definition}{定义}
\newtheorem{lemma}{引理}
\newtheorem{example}{例题}
\newtheorem{answer}{解:}
\newtheorem{dablock}{}
\newtheorem{jytg}{提纲}
\newtheorem{daproof}{证明}
\newtheorem{explain}{说明}
\newtheorem{summary}{小结}

\newtheorem{zhuyi}{注意}
\newtheorem{shuoming}{说明}
\newtheorem{wenti}{问题}
\newtheorem{jielun}{结论}
\newtheorem{yinli}{引理}
%%----------------------------------------------------------------------------------------------------
\title{\heiti 第0章\quad 绪论}
\author[\textcolor{blue}]{{\sihao\kaishu  笪邦友}}
\institute{\sihao\lishu  \textcolor{violet}{中南民族大学~~ 电子信息工程学院}}
\date{\fangsong\today} 

\begin{document}
	%  \begin{CJK*}{GBK}{kai}
\kaishu
\frame{ \titlepage }
	%%---------------------------------------------------------------------------------------------------
%\section*{目录}
%\frame{\kaishu\frametitle{\kaishu 目录}\tableofcontents}
\section*{前言}


%%%%%%%%%%%%%%%%%%%%%%%%%%%%%%%%%%%%%%%%%%%%%%%%%%%%%%%%%%%%%%%%%%%%%%%%%%%%%%%%%%%%%%%%%%%%%%%
\section{引言}


\subsection{信号}
\begin{frame}\frametitle{信号与消息}%[allowframebreaks][shrink]

    现代社会是一个信息社会,一个重要的问题是对信息的认识的和处理。
\begin{enumerate}
  \item 消息
      \begin{itemize}
          \item 一般来说,信息一定要用某种物理的形式表达出来。比如说,用语言、文字、图画、编码等东西。这些东西被称为消息(message)。
      \end{itemize}
  \item 信号
      \begin{itemize}
         \item 而消息依附于某一物理量的变化,就构成了信号(signal)。
      \end{itemize}
\end{enumerate}
%%%%%%%%%%%%%%%%%%%%%%%%%%%%%%%%%%%%%%%%%%%%%%%%%%%%%%%%%%%%%%%%%%%%%%%%%%%%%%%%%%%%%%%%%%%%%%%
%
%
%
%
%
%
%%%%%%%%%%%%%%%%%%%%%%%%%%%%%%%%%%%%%%%%%%%%%%%%%%%%%%%%%%%%%%%%%%%%%%%%%%%%%%%%%%%%%%%%%%%%%%%
\begin{dablock}
   广义上说,信号是一种随时间(或空间)变化的物理量。或者说,信号可表示为时间变量或空间变量的函数。
\end{dablock}
\end{frame}
%%%%%%%%%%%%%%%%%%%%%%%%%%%%%%%%%%%%%%%%%%%%%%%%%%%%%%%%%%%%%%%%%%%%%%%%%%%%%%%%%%%%%%%%%%%%%%%
%
%
%
%
%
%
%%%%%%%%%%%%%%%%%%%%%%%%%%%%%%%%%%%%%%%%%%%%%%%%%%%%%%%%%%%%%%%%%%%%%%%%%%%%%%%%%%%%%%%%%%%%%%%
\begin{frame}[shrink]\frametitle{信号的形式}%[allowframebreaks][shrink]
\begin{dablock}
信号为我们所熟知的是其时域形式,引入傅里叶变换的概念后,信号同时存在频域形式。
\end{dablock}
这里傅里叶变换的相关知识,是信号处理的核心概念与核心内容,包括: 傅里叶级数、傅里叶变换、拉普拉斯变换等等。


\end{frame}
%%%%%%%%%%%%%%%%%%%%%%%%%%%%%%%%%%%%%%%%%%%%%%%%%%%%%%%%%%%%%%%%%%%%%%%%%%%%%%%%%%%%%%%%%%%%%%%
%
%
%
%
%
%
%%%%%%%%%%%%%%%%%%%%%%%%%%%%%%%%%%%%%%%%%%%%%%%%%%%%%%%%%%%%%%%%%%%%%%%%%%%%%%%%%%%%%%%%%%%%%%%
\subsection{信号与系统}
\begin{frame}[shrink]\frametitle{系统的概念}%[allowframebreaks][shrink]


    有了信号的概念后,我们就考虑对信号进行处理,需引入了系统的概念,

\begin{dablock}
    可认为这种对信号的处理构成了一个系统。在这里系统的观点着重于输入输出间的关系。
\end{dablock}
    为了考察对系统的认识,我们需了解系统的特性。

    系统的不同特性如下
    \begin{dablock}
    系统的不同特性如下
    \begin{enumerate}
      \item 线性:   %\par    \qquad 所谓线性就是同时具有齐次性和加性。
      \item 时不变性  %\par    \qquad         系统的性质不随时间变化。
      \item 因果性  %\par     \qquad         符合因果律的系统,也就是物理上的可实现性。
      \item 稳定性  %\par    \qquad          有限的输入产生有限的输出。
    \end{enumerate}
    \end{dablock}
\end{frame}
%%%%%%%%%%%%%%%%%%%%%%%%%%%%%%%%%%%%%%%%%%%%%%%%%%%%%%%%%%%%%%%%%%%%%%%%%%%%%%%%%%%%%%%%%%%%%%%
%
%
%
%
%
%
%%%%%%%%%%%%%%%%%%%%%%%%%%%%%%%%%%%%%%%%%%%%%%%%%%%%%%%%%%%%%%%%%%%%%%%%%%%%%%%%%%%%%%%%%%%%%%%
\subsection{数字信号处理的引入}

\begin{frame}\frametitle{模拟系统}%[allowframebreaks][shrink]


    既然信号是一种物理量,那么模拟信号的处理也就是利用物理器件直接对数字信号进行处理。

例如
\begin{enumerate}
    \item 一个电容和一个电阻将构成一个模拟高通滤波器。
    \item 一组光学组件也可构成一种低通滤波器。
\end{enumerate}
\begin{dablock}
    一般来说,一个模拟系统是通过一些模拟器件(比如说,晶体管、运算放大器、电阻、电容、电感等器件)组成的网络来进行处理。
\end{dablock}
%    一个典型的、简单的模拟高通滤波器如下图所示:
%    \newline\newline\newline\newline\newline\newline
%    \newline\newline\newline\newline\newline\newline
\end{frame}
%%%%%%%%%%%%%%%%%%%%%%%%%%%%%%%%%%%%%%%%%%%%%%%%%%%%%%%%%%%%%%%%%%%%%%%%%%%%%%%%%%%%%%%%%%%%%%%
%
%
%
%
%
%
%%%%%%%%%%%%%%%%%%%%%%%%%%%%%%%%%%%%%%%%%%%%%%%%%%%%%%%%%%%%%%%%%%%%%%%%%%%%%%%%%%%%%%%%%%%%%%%
\begin{frame}\frametitle{模拟滤波器的缺点}%[allowframebreaks][shrink]

\begin{dablock}
    \begin{itemize}
      \item [(1)] 不够灵活。
            \begin{itemize}
              \item 一但任务改变,其网络结构难以改变,如要将高通滤波器改为低通滤波器,就需要更换电路元器件。
            \end{itemize}
      \item [(2)]  不稳定。
            \begin{itemize}
              \item 其主要由电信号或光信号组成,存在物理信号的相关问题。这个问题非常复杂,如高精度设备里面的电磁信号耦合问题。
            \end{itemize}
      \item [(3)]  精度不高。
            \begin{itemize}
              \item 由于物理限制,模拟系统无法达到太高的精度。物理限制一般是无法克服的障碍,而数字系统通常可以达到极高精度。
            \end{itemize}
    \end{itemize}
\end{dablock}
\end{frame}
%%%%%%%%%%%%%%%%%%%%%%%%%%%%%%%%%%%%%%%%%%%%%%%%%%%%%%%%%%%%%%%%%%%%%%%%%%%%%%%%%%%%%%%%%%%%%%%
%
%
%
%
%
%
%%%%%%%%%%%%%%%%%%%%%%%%%%%%%%%%%%%%%%%%%%%%%%%%%%%%%%%%%%%%%%%%%%%%%%%%%%%%%%%%%%%%%%%%%%%%%%%
\begin{frame}[shrink]\frametitle{模拟滤波器的缺点}%[allowframebreaks][shrink]
\begin{dablock}

模拟滤波器难以处理人工智能问题



\end{dablock}
    \textbf{举例说明}:
    \begin{enumerate}
      \item 雷达测速,雷达测距
      \item 图像导航
      \item 机器人(智能车)导航
    \end{enumerate}


     模拟系统只能用于对信号的“简单”处理,某些应用需要对信号的“智能”处理,我们只能用数字信号处理的方式来完成。为了要利用数字计算机对信号进行处理,我们必须将模拟信号数字化,对其进行数字信号处理。
\end{frame}
%%%%%%%%%%%%%%%%%%%%%%%%%%%%%%%%%%%%%%%%%%%%%%%%%%%%%%%%%%%%%%%%%%%%%%%%%%%%%%%%%%%%%%%%%%%%%%%
%
%
%
%
%
%
%%%%%%%%%%%%%%%%%%%%%%%%%%%%%%%%%%%%%%%%%%%%%%%%%%%%%%%%%%%%%%%%%%%%%%%%%%%%%%%%%%%%%%%%%%%%%%%
\section{数字信号处理简介}
\subsection{数字信号处理的概念}
\begin{frame}[shrink]\frametitle{什么是数字信号处理}%[allowframebreaks][shrink]
\begin{definition}
数字信号处理就是利用\textbf{数值计算方法},按预定的规则进行运算,对数字序列进行各种处理,把信号变换为需要的某种形式,便于分析、识别和使用。
\end{definition}

\begin{zhuyi}
信号处理的实质是\textbf{“运算”},即通过对信号的运算达到各种应用目的,包括对信号的检测,滤波、谱分析、调制等。
\end{zhuyi}

\end{frame}
%%%%%%%%%%%%%%%%%%%%%%%%%%%%%%%%%%%%%%%%%%%%%%%%%%%%%%%%%%%%%%%%%%%%%%%%%%%%%%%%%%%%%%%%%%%%%%%
%
%
%
%
%
%
%%%%%%%%%%%%%%%%%%%%%%%%%%%%%%%%%%%%%%%%%%%%%%%%%%%%%%%%%%%%%%%%%%%%%%%%%%%%%%%%%%%%%%%%%%%%%%%
\begin{frame}[shrink]\frametitle{  数字信号处理的典型应用 }%[allowframebreaks][shrink]
\begin{dablock}
\begin{enumerate}
    \item 一维数字信号处理:如语音识别、语音合成。
    \item 二维数字信号处理:如工业上有图像处理。例如人脸识别、机器视觉、卫星侦察、巡航导弹的电视制导等。
\end{enumerate}
\end{dablock}
\end{frame}
%%%%%%%%%%%%%%%%%%%%%%%%%%%%%%%%%%%%%%%%%%%%%%%%%%%%%%%%%%%%%%%%%%%%%%%%%%%%%%%%%%%%%%%%%%%%%%%
%
%
%
%
%
%
%%%%%%%%%%%%%%%%%%%%%%%%%%%%%%%%%%%%%%%%%%%%%%%%%%%%%%%%%%%%%%%%%%%%%%%%%%%%%%%%%%%%%%%%%%%%%%%
\begin{frame}[shrink]\frametitle{  数字信号处理的实施方法 }%[allowframebreaks][shrink]


\quad\quad \emph{数字信号处理的研究对象是数字信号,处理方式是数值运算。因此数字信号处理必然与计
算机联系在一起。}\newline

数字信号处理的具体实施方法有三种:
\begin{dablock}
\begin{enumerate}
    \item  利用通用计算机,软件实现(c语言,Matlab语言等),优点是灵活性好,但速度慢。
    \item  利用专用硬件实现,如利用FPGA等可编程逻辑阵列,开发专用芯片等,其实时性好,易于集成。
    \item  利用通用可编程DSP芯片,以软硬件结合的方式实现,其灵活性好,实时性好。
\end{enumerate}
\end{dablock}
\end{frame}
%%%%%%%%%%%%%%%%%%%%%%%%%%%%%%%%%%%%%%%%%%%%%%%%%%%%%%%%%%%%%%%%%%%%%%%%%%%%%%%%%%%%%%%%%%%%%%%
%
%
%
%
%
%
%%%%%%%%%%%%%%%%%%%%%%%%%%%%%%%%%%%%%%%%%%%%%%%%%%%%%%%%%%%%%%%%%%%%%%%%%%%%%%%%%%%%%%%%%%%%%%%
\begin{frame}[shrink]\frametitle{  学科发展历史 }%[allowframebreaks][shrink]

\quad\quad 很久以来,人们对数字信号处理的重要性早有认识,但限于当时的科学发展水平,未能将之实用化。
\newline

\quad\quad 70年代以前,对信号的处理绝大多数是用模拟方法实现的,因为数字信号处理运算量太大,当时的计算能力远远不够,无法应用到实际工作中。
\newline

\quad\quad 近几十年来,现代科技水平的发展克服了数字信号处理的缺点,使得数字信号处理技术,成为当前,乃至未来信号处理的主流方法。
\newline

\end{frame}
%%%%%%%%%%%%%%%%%%%%%%%%%%%%%%%%%%%%%%%%%%%%%%%%%%%%%%%%%%%%%%%%%%%%%%%%%%%%%%%%%%%%%%%%%%%%%%%
%
%
%
%
%
%
%%%%%%%%%%%%%%%%%%%%%%%%%%%%%%%%%%%%%%%%%%%%%%%%%%%%%%%%%%%%%%%%%%%%%%%%%%%%%%%%%%%%%%%%%%%%%%%
\begin{frame}[shrink]\frametitle{  学科发展历史 }%[allowframebreaks][shrink]
数字信号处理的快速发展主要归功于两个因素:
\begin{enumerate}
    \item [(1)]快速算法的研究降低了计算复杂度
        \begin{itemize}
            \item 1965年图基和库利发表了第一篇快速傅里叶变换论文后,将运算速度提高了2-3个数量级,使得计算复杂度大大降低,将该学科的发展向实际应用大大推动了一步。也就是说,在普通的应用场景下,仅算法的改进,就把速度提高了约1000倍。
        \end{itemize}
    \item [(2)]现代芯片技术的发展导致计算能力急剧提高
    \begin{itemize}
        \item  例如: 最初的电脑计算能力仅为5000次/秒 ,而现在的高端个人电脑的计算能力几乎可以
        达到5000万次/秒,而超级电脑甚至达到了数万亿次/秒。也就是说,计算能力提高了几乎
        一万万倍。\par
        这意味着,对信号处理算法,如果以前需要算1年的话,现在只需要0.0003秒。
    \end{itemize}

\end{enumerate}
\end{frame}
%%%%%%%%%%%%%%%%%%%%%%%%%%%%%%%%%%%%%%%%%%%%%%%%%%%%%%%%%%%%%%%%%%%%%%%%%%%%%%%%%%%%%%%%%%%%%%%
%
%
%
%
%
%
%%%%%%%%%%%%%%%%%%%%%%%%%%%%%%%%%%%%%%%%%%%%%%%%%%%%%%%%%%%%%%%%%%%%%%%%%%%%%%%%%%%%%%%%%%%%%%%
\begin{frame}[shrink]\frametitle{  学科发展历史 }%[allowframebreaks][shrink]

正是快速算法的提出,与计算能力的提高,使得数字信号处理学科得到了飞速的发展。成为今后信号处理的主流方法。


可以说凡是有信号处理需要的领域都离不开数字信号处理技术,在很多领域里,数字信号处理技术都是一项必不可少的工具。


当然,实际科研、生产等应用需要的牵引,是数字信号处理学科发展的最大动力。


\end{frame}
%%%%%%%%%%%%%%%%%%%%%%%%%%%%%%%%%%%%%%%%%%%%%%%%%%%%%%%%%%%%%%%%%%%%%%%%%%%%%%%%%%%%%%%%%%%%%%%

%%%%%%%%%%%%%%%%%%%%%%%%%%%%%%%%%%%%%%%%%%%%%%%%%%%%%%%%%%%%%%%%%%%%%%%%%%%%%%%%%%%%%%%%%%%%%%%
\section{学科内容}

\begin{frame}[shrink]\frametitle{学科内容}%[allowframebreaks][shrink]


本学科包括三方面内容:
\begin{enumerate}
  \item
      一维数字信号处理,为本书的基本内容。
  \item
      多维数字信号处理,如图像处理为典型的二维信号处理,下学期我们将学习数字图像处理。
  \item
      用超大规模集成电路(VLSI)及硬件设备实现各种处理算法。
\end{enumerate}
\end{frame}
%%%%%%%%%%%%%%%%%%%%%%%%%%%%%%%%%%%%%%%%%%%%%%%%%%%%%%%%%%%%%%%%%%%%%%%%%%%%%%%%%%%%%%%%%%%%%%%
%
%
%
%
%
%
%%%%%%%%%%%%%%%%%%%%%%%%%%%%%%%%%%%%%%%%%%%%%%%%%%%%%%%%%%%%%%%%%%%%%%%%%%%%%%%%%%%%%%%%%%%%%%%
\begin{frame}[shrink]\frametitle{主要内容}%[allowframebreaks][shrink]


本课程仅限于一维数字信号处理的一些最基础知识。其主要内容集中于两个中心问题:
\begin{enumerate}
  \item
      DFT(离散傅里叶变换(1-4章)
      \begin{enumerate}
        \item 时域离散信号与系统的概念与频域分析
        \item DFT — \qquad 一种新的、可适用于计算机处理的傅里叶变换
        \item FFT — \qquad 一种DFT的快速算法。
      \end{enumerate}
  \item
      数字滤波器(5-7章)
      \begin{enumerate}
        \item 时域离散系统的网络结构
        \item IIR—DF的设计  (无限冲击响应数字滤波器)
        \item FIR—DF的设计  (有限冲击响应数字滤波器)
      \end{enumerate}

\end{enumerate}
\end{frame}
%%%%%%%%%%%%%%%%%%%%%%%%%%%%%%%%%%%%%%%%%%%%%%%%%%%%%%%%%%%%%%%%%%%%%%%%%%%%%%%%%%%%%%%%%%%%%%%
%
%
%
%
%
%
%%%%%%%%%%%%%%%%%%%%%%%%%%%%%%%%%%%%%%%%%%%%%%%%%%%%%%%%%%%%%%%%%%%%%%%%%%%%%%%%%%%%%%%%%%%%%%%
\section{教学安排与特点}
\begin{frame}[shrink]\frametitle{教材与参考书}%[allowframebreaks][shrink]
以本书为主,按照教材编撰线索讲授。
\begin{enumerate}
  \item 教材
        \begin{itemize}
          \item 《数字信号处理》,西电,高西全,丁玉美
        \end{itemize}
  \item 参考书
          \begin{itemize}
            \item 《学习指导》      丁玉美
            \item 《数字信号处理》,华中科大出版社,  姚天任
            \item 《信号与系统》,高教出版社,管致中
          \end{itemize}
\end{enumerate}
\end{frame}
%%%%%%%%%%%%%%%%%%%%%%%%%%%%%%%%%%%%%%%%%%%%%%%%%%%%%%%%%%%%%%%%%%%%%%%%%%%%%%%%%%%%%%%%%%%%%%%
%
%
%
%
%
%
%%%%%%%%%%%%%%%%%%%%%%%%%%%%%%%%%%%%%%%%%%%%%%%%%%%%%%%%%%%%%%%%%%%%%%%%%%%%%%%%%%%%%%%%%%%%%%%
\begin{frame}[shrink]\frametitle{教学安排}%[allowframebreaks][shrink]


\begin{itemize}
  \item
      本课程属于专业课性质,是一定难度。
  \item
      在学习中,要注意和《信号与系统》中所学习过的模拟信号与系统的一些结论对比,两者在
      许多地方相类似,甚至有些结论可直接引申过来,当然,两者之间也存在许多不同之处。
      同学们在学习中,要注意其不同之处。
\end{itemize}

\end{frame}
%%%%%%%%%%%%%%%%%%%%%%%%%%%%%%%%%%%%%%%%%%%%%%%%%%%%%%%%%%%%%%%%%%%%%%%%%%%%%%%%%%%%%%%%%%%%%%
%\begin{frame}\frametitle{title}%[allowframebreaks][shrink]
%
%\end{frame}
%%%%%%%%%%%%%%%%%%%%%%%%%%%%%%%%%%%%%%%%%%%%%%%%%%%%%%%%%%%%%%%%%%%%%%%%%%%%%%%%%%%%%%%%%%%%%%%







\end{document}

